% กำหนดคลาสและขนาดฟอนต์ของเอกสาร
\documentclass[10pt,twoside]{book}

% ตั้งค่าขนาดกระดาษและขอบ
\usepackage[paperwidth=165mm, paperheight=230mm, margin=2cm, top=2cm,bottom=2cm,outer=2cm,inner=2.2cm,columnsep=30pt]{geometry}

% ใช้งานแพ็กเกจสำหรับฟอนต์
\usepackage{fontspec}

% ตั้งค่าการแบ่งบรรทัดสำหรับภาษาไทย
\usepackage{polyglossia}
\setmainlanguage[numerals=thai]{thai}
\setotherlanguage{english}
\XeTeXlinebreaklocale "th"
\XeTeXlinebreakskip = 0pt plus 2pt minus 1pt

% แพคเกจอื่น ๆ
\usepackage{needspace}

% ตั้งค่าฟอนต์ Laksaman or Saraban
\setmainfont[%
    ItalicFont={Laksaman-Italic.otf},%
    BoldFont={Laksaman-Bold.otf},%
    BoldItalicFont={Laksaman-BoldItalic.otf},%
    Script=Thai,%
    Scale=MatchLowercase,%
    WordSpace=1.25,%
    Mapping=tex-text,%
]{Laksaman.otf}

% ตั้งค่าหัวเรื่องและการจัดหน้า
\usepackage{multicol}
\usepackage[bf,center]{titlesec}

% ปกรอง 
\newcommand{\booktitle}{ปทานุกรม}
\newcommand{\subtitle}{พฺรหฺมชาลสุตฺตํ}

% ตั้งค่าหัวกระดาษและท้ายกระดาษ
\usepackage{fancyhdr}
\setlength{\headheight}{15pt}

% ตัวแปรอักษร
\newcommand{\currentfirstchar}{}
\fancyhead[LE,RO]{\booktitle} % หน้าคู่ซ้าย, หน้าคี่ขวา
\fancyhead[LO,RE]{\subtitle}  % หน้าคี่ซ้าย, หน้าคู่ขวา
\renewcommand{\headrulewidth}{0pt}

% รูปแบบเนื้อหา
\usepackage{needspace}
\usepackage{scrextend}

\newcommand{\entry}[5]{
    \needspace{2cm}
    \begin{addmargin}[1.2em]{0em}
        \hspace{-1.2em}\textbf{#1} {\textit{#2}} {#3}\\ % ศัพท์ และรายละเอียด
        \begin{minipage}{\columnwidth}
            {#4}\\
            \vspace{0.5em}{\footnotesize{#5}} 
        \end{minipage}                    
    \end{addmargin}
    \par\vspace{0.1em}
}


\begin{document}
    \frontmatter
    
    \frontmatter
    \pagestyle{empty}

    % Title page
    \begin{titlepage}
        \centering
        
        ~
        
        \vspace{24pt}
        {\scshape\Huge \booktitle\par}
        \vspace{6pt}
        {\scshape\large \subtitle\par}
        \vspace{\stretch{1.25}}
    \end{titlepage}
    \cleardoublepage
    

    \mainmatter
    \pagestyle{fancy}

    \needspace{2cm}
    \begin{addmargin}[1.2em]{0em}
        \hspace{-1.2em}\textbf{something} {\textit{something}} [{something}]\\
        {something}\\
        \vspace{0.5em}{\footnotesize{something}}
    \end{addmargin}
    \par\vspace{0.1em}

    \needspace{5cm} \section*{\Huge อ}
\noindent\rule{\textwidth}{0.4pt}
\begin{multicols}{2}
\entry{อกมฺปิตฺถ} {ขฺยา} {[กปิ / กมฺป + อา]} {หวั่นไหวแล้ว อุ. โลกธาตุ อกมฺปิตฺถ.(ที. ๑/๑๔๘/๔๓)\&nbsp; โลกธาตุหวั่นไหวแล้ว\, แผ่นดินไหว \* เป็นอาขยาตกริยา รูปอดีตกาลของ กมฺปติ(กปิ+อ+ติ) สำเร็จรูปโดยลง อ อาคมหน้าธาตุ ลง อิอาคมท้ายธาตุ แปลง อา เป็น ตฺถ\, บางแห่งไม่ลง อ อาคมหน้าธาตุ ใช้เป็นรูปว่า กมฺปิตฺถ ก็มี (ที.อฏฺ. ๒/๕๗)} {อกมฺปิตฺถ}
\entry{อกิลนฺตกาย} {ติ} {[อกิลนฺต + กาย]} {ผู้มีร่างที่ไม่เหนื่อยไม่เพลีย} {อกิลนฺตกายา}
\entry{อกิลนฺตจิตฺต} {ติ} {[อกิลนฺต + จิตฺต]} {ผู้มีจิตใจที่ไม่เหนื่อยไม่เพลีย} {อกิลนฺตจิตฺตา}
\entry{อกุสล} {ติ} {[น + กุสล]} {๑. อันเป็นไปกับด้วยโทษ\, อันมีโทษ\, อันมีภัย (ม.) ๒. ผู้ไม่เฉลียวฉลาด (ม.) ๓. ผู้ไม่มีปัญญา (ขุ.สุ.) ๔. อาบัติ (อํ.) ๕. อันให้ผลเป็นโทษและเป็นทุกข์\, อันเป็นปฏิปักข์กับกุศล (อภิ.อ.) ๖. อันไม่ปราศจากโรค\, อันมีโรค (ขุ.ปฏิ.) ๗. อันเกิดจากความไม่ฉลาดกล่าวคือโมหะ (อํ.)} {อกุสลํ, อกุสเลหิ}
\entry{อกฺข} {ปุ\, นปุ} {[(๑) อก + ข\, อกฺข + ณ]} {๑. เพลารถ\, เพลาเกวียน (วิ\, มิ\, สํ\, ขุ.ชา) ๒. ลูกเต๋า (วิ\, ขุ.ชา)\, การทอดลูกเต๋า (การเล่นไฮโล) ๓. อินทรีย์มีจักษุเป็นต้น (สํ\, อภิ.) ๔. ศาล\, โรงวินิจฉัยคดี\, คดี (สารตฺถ.ฏี.) ๕. (ก) ต้นสมอ(ต้นไม้สมุนไพรรักษาโรคชนิดหนึ่ง (อภิธาน.)      ข. น้ำหนักของที่หนักเท่ากับเมล็ดสมอ (๒อักขะ=๕ มาสก) (อภิธานฎีกา) [ข้อ ๑\,๒ เป็น ปุง และ นปุง\, ข้อ ๓ เป็น นปุง\, ข้อ ๔\,๕ เป็น ปุง]} {อกฺขํ, อกฺขานํ}
\entry{อกฺขริกา} {อิตฺ} {[อกฺขร + ณิก]} {การเล่นปริศนาไขว้อักษร\, การเล่นอักษรแปรรูป\, การเล่นทายอักษร(วิ.)} {อกฺขริกํ}
\entry{อกฺขาต} {ติ} {[อา + ขา + ต]} {อันถูกกล่าวแสดงไว้\, ที่ถูกเรียกว่า} {อกฺขาตา}
\entry{อกฺขายติ} {ขฺยา} {[อา + ขา + ย + ติ]} {อันบุคคลย่อมกล่าว-แสดง-เรียก} {อกฺขายติ}
\entry{อคาร} {นปุ} {[อค + อาร]} {บ้าน\, เรือน\, ครัวเรือน การครองเรือน ชีวิตฆราวาส (คำนี้ใช้ความหมายเดียวกันกับคำว่า \`อาคาร\` } {อคารสฺมา}
\entry{อคฺคพีช} {นปุ} {[อคฺค + พีช]} {พันธุ์ไม้ประเภทเด็ดยอดปลูก กล้าไม้ที่หักยอดก่อนแล้วจึงปลูก (วิ.)} {อคฺคพีชํ}
\entry{อคฺคิโหม} {ปุ\, นปุ} {[อคฺคิ + โหม]} {การบูชาไฟ\, การบำเรอไฟ} {อคฺคิโหมํ}
\entry{องฺค} {ติ} {[องฺค + ณ]} {๑. ใช้เป็นคำอาลปนะ(คำร้องเรียก) (อภิธาน.) ๒. องค์\, ส่วน ๓. สาเหตุ\, ส่วนที่เป็นพื้นฐาน ๔. ชิ้นส่วน\, ส่วนประกอบ (อภิ.ธ.) ๕. องค์ ๙ มี สุตฺตะ เคยฺย เป็นต้น (วิ.อฏฺ.) ๖. คุณสมบัติ (อํ)<br>๗. ส่วนที่ทำให้ครบองค์ประกอบแห่งโทษ (สํ.) ๘. นิมิตหมาย\, เครื่องหมาย (ขุ.เถร.อฏฺ.) ๙. กาย\, ร่างกาย (ชา.) ๑๐. (ก) อวัยวะของร่างกาย (วิ.)        (ข) พี่กับน้อง (พี่ชายน้องชาย) (ชา.)       (ค) เนื้อหนังมังสาอันเป็นส่วนหนึ่งของร่างกาย (ชา.)      (ฆ) อวัยวะของนกมีปีกเป็นต้น (ชา.อฏฺ.)      (ง) อวัยวะของเต่ามีหัวและขาเป็นต้น (สํ.) ๑๑. เครื่องจดจำ\, ลักษณะ\, ตำหนิ (ขุ.มหานิ.อฏฺ.) ๑๒. หมู่\, กอง (วิ.) ๑๓. อังคปกรณ์\, คัมภีร์โหราศาสตร์ (สำหรับสังเกตจากโหงวเฮ้งแล้วทำนายผลดีผลเสีย) (หมายเหตุ \: (๑) องฺค ศัพท์ที่สำเร็จรูปเป็นบทตัทธิตแล้วจะมีการใช้เป็นลิงค์ที่ต่างจาก องฺค ที่เป็นคำดั้งเดิม เช่น หากลง ณ ปัจจัย ใน อัสสัตถิตัทธิต ก็จะใช้เป็นไตรลิงค์ และมีความหมายว่่า ผู้มีอวัยวะ\, ผู้มีสัดส่วนของร่างกายที่งดงาม หากใช้เป็นตัทธิตดังกล่าว แต่หากใช้เป็นปุลลิงค์อย่างเดียวก็จะหมายถึง (๑) พระปัจเจกพุทธเจ้า ผู้มีพระนามว่าอังคะ (๒) พระเถระนามว่าอังคะ (โลมสกังคิยเถระ) (๓) พระราชานามว่าอังคะ  (๒) องฺค ศัพท์ที่คำเดิมเป็นปุลลิงค์พหูพจน์ จะใช้ในความหมายว่าแคว้น\, มหาชนบท\, หรือประเทศ นั่นคือแคว้นอังคะซึ่งเป็นหนึ่งใน ๑๖ แคว้น ของอินเดียโบราณ (๓) หากเป็นอังคะศัพท์ที่แปลว่า แคว้น มาทำเป็นบทตัทธิตโดยการลง ณ ปัจจัย ท้าย องฺค (องฺค + ณ) ก็จะใช้เป็นคำคุณบท ไตรลิงค์ หมายถึงบุคคลหรือสิ่งของที่เกี่ยวข้องกับแคว้นอังคะ เช่น คนที่เกิดในแคว้นอังคะ คนที่อาศัยอยู่ในแคว้นอังคะ ชาวแคว้นอังคะ พระราชาผู้ปกครองแคว้นอังคะ สินค้าที่มาจากแคว้นอังคะ } {องฺคํ}
\entry{องฺควิชฺชา} {อิตฺ} {[องฺค + วิชฺชา]} {อังควิชชาศาสตร์ (๑) ศาสตร์ที่ทำนายชีวิตด้วยการดูรูปลักษณ์ของคน (ดูโหงวเฮ้ง) แล้วร่ายมนต์ประกอบคำทำนาย (๒) ศาสตร์ว่าด้วยการหยั่งรู้ชะตาชีวิตของผู้คนด้วยการสังเกตลักษณะที่บริบูรณ์ของร่างกายมีเท้าและมือเป็นต้นแล้วทำนายว่าจะเป็นผู้ถึงพร้อมไปด้วยบุญ โชค วาสนา เป็นต้น อย่างไร  (๓) ศาสตร์ที่ล่วงรู้ลักษณะของอวัยวะส่วนของร่างกายทั้งของหญิงและชาย ว่าจะมีบุญหรือกาลกิณี (ผู้มีวาสนาหรือไร้วาสนา) (ขุ.มหานิ.) } {องฺควิชฺชา}
\entry{อจฺจย} {ปุ} {[อติ + อิ + อ]} {การล่วงไป} {อจฺจเยน}
\entry{อจฺฉริย} {ติ} {[อจฺฉริย + อ]} {ช่างน่าอัศจรรย์จริงหนอ!} {อจฺฉริยํ}
\entry{อชยุทฺธ} {นปุ} {[อช + ยุทฺธ]} {การแข่งขันชนแพะ} {อชยุทฺธํ}
\entry{อชลกฺขณ} {นปุ} {[อช + ลกฺขณ]} {ศาสตร์ว่าด้วยการทำนายลักษณะของแพะ} {อชลกฺขณํ}
\entry{อชานตา} {อิตฺ} {[อชานนฺต + ตา]} {ผู้ไม่รู้} {อชานตํ, ชานตา}
\entry{อชินปฺปเวณี} {อิตฺ} {[อชิน + ปเวณี]} {เครื่องลาดที่ทำด้วยหนังเสือดำ} {อชินปฺปเวณิํ}
\entry{อเชฬกปฏิคฺคหณ} {นปุ} {ไม่มีการประกอบคำ} {จากการรับแพะและแกะ} {อเชฬกปฏิคฺคหณา}
\entry{อชฺฌตฺต} {ติ} {ไม่มีการประกอบคำ} {อันเป็นไปภายในขันธสันดานของตน} {อชฺฌตฺตํ}
\entry{อญฺชน} {ปุ\, นปุ} {[อญฺช + ยุ]} {การแต้มตาให้สวยงาม} {อญฺชนํ}
\entry{อญฺญ} {ติ} {[นา + ญา + อ]} {อันอื่นจากจูฬศีล มัชฌิมศีล มหาศีลนั่นเทียว} {อญฺเญ}
\entry{อญฺญตร} {ติ} {[อญฺญ + ตร]} {กันและกัน} {อญฺญตเรน}
\entry{อญฺญตฺร} {พฺยย} {ไม่มีการประกอบคำ} {ด้วยเหตุใดเหตุหนึ่ง} {อญฺญตฺร}
\entry{อญฺญถา} {พฺยย} {[อญฺญ + ถา]} {โดยประการอื่น} {อญฺญถา}
\entry{อญฺญทตฺถุทส} {ติ} {[อญฺญทตฺถุ + ทส]} {ผู้เห็นอย่างถ่องแท้ทุกอย่าง} {อญฺญทตฺถุทโส}
\entry{อญฺญมญฺญ} {ติ} {[อญฺญ + อญฺญ]} {กันและกัน} {อญฺญมญฺญํ, อญฺญมญฺญมฺหิ, อญฺญมญฺญสฺส}
\entry{อฏฺฐปท} {นปุ} {[อฏฺฐ + ปท]} {การพนันหมากรุกที่มีแถวละ ๘ ตา} {อฏฺฐปทํ}
\entry{อฏฺฐาห} {นปุ} {[อฏฺฐ + อห]} {๘ ประการ} {อฏฺฐหิ}
\entry{อตกฺกาวจร} {ติ} {[น + ตกฺก + อวจร]} {เป็นธรรมชาติที่ไม่สามารถรู้ได้ด้วยจินตนาการ} {อตกฺกาวจรา}
\entry{อตจฺฉ} {ติ} {[น + ตจฺฉ + อตฺถ]} {เป็นคำพูดที่ไม่ถูกต้อง} {อตจฺฉํ}
\entry{อติเวล} {ติ} {[อติ + เวลา]} {นานเกินควร} {อติเวลํ}
\entry{อตฺต} {ปุ} {ไม่มีการประกอบคำ} {อัตตา} {อตฺตนา, อตฺตา, อตฺตานํ}
\entry{อตฺตมน} {ติ} {[ (ก) อตฺต + มน]} {ผู้มีใจเป็นของตน[เป็นผู้มีใจร่าเริงยินดี]} {อตฺตมนา}
\entry{อตฺถ} {ขฺยา} {[อส + อ + ถ]} {ย่อมเป็น} {อตฺถ}
\entry{อตฺถงฺคม} {ปุ} {[อตฺถ + คมุ + อ]} {การดับลง} {อตฺถงฺคมํ, อตฺถงฺคมา}
\entry{อตฺถชาล} {ติ} {[อตฺถ + ชาล]} {ข่ายกล่าวคือประโยชน์} {อตฺถชาลํ}
\entry{อตฺถวาที} {ติ} {[อตฺถ + วท + ณี]} {เป็นผู้มีปกิตกล่าวเฉพาะคำพูดที่มีประโยชน์} {อตฺถวาที}
\entry{อตฺถสญฺหิต} {ติ} {[อตฺถ + สํหิต]} {อันถึงพร้อมด้วยความหมาย} {อตฺถสญฺหิตํ}
\entry{อตฺถิ} {ขฺยา} {[อส (=ภุวิ) + ติ]} {มีอยู่} {อตฺถิ}
\entry{อถ} {พฺยย} {[ธาน]} {ครั้นไม่นานนัก} {อถ}
\entry{อเถน} {ติ} {[น + เถน]} {ไม่ลักขโมย} {อเถเนน}
\entry{อทินฺนาทาน} {นปุ} {[อทินฺน + อาทาน]} {การถือเอาสิ่งของที่ผู้อื่นไม่ได้ให้ด้วยกายหรือวาจา[การลักทรัพย์]} {อทินฺนาทานํ, อทินฺนาทานา}
\entry{อทุกฺขมสุข} {ติ} {[อทุกฺข + อสุข]} {อันมีสภาพเป็นทุกข์ก็ไม่ใช่เป็นสุขก็ไม่ใช่} {อทุกฺขมสุขํ}
\entry{อทุกฺขมสุขี} {ติ} {[อทุกฺขมสุข + อี]} {เป็นธรรมชาติที่ไม่มีทั้งทุกข์และสุข} {อทุกฺขมสุขี}
\entry{อทฺทสาม} {ขฺยา} {[อา + ทิส + ม]} {ย่อมเห็น} {อทฺทสาม}
\entry{อทฺธ} {ปุ\, นปุ} {[ (๑) อสติ เขเปติ สมุทายนฺติ อทฺโธ\, โต]} {ตลอดกาลนาน} {อทฺธานํ}
\entry{อทฺธานมคฺคปฺปฏิปนฺน} {ติ} {[อทฺธานมคฺค + ปฺปฏิปนฺน]} {ผู้เดินทางไกล} {อทฺธานมคฺคปฺปฏิปนฺโน}
\entry{อทฺธุว} {ติ} {[น + ธุว]} {ผู้ไม่ยั่งยืน} {อทฺธุวา, อทฺธุโว}
\entry{อธิจฺจสมุปฺปนฺน} {ติ} {[อธิจฺจ + สมุปฺปนฺน]} {เป็นธรรมชาติที่เกิดขึ้นโดยไม่มีเหตุ} {อธิจฺจสมุปฺปนฺนํ, อธิจฺจสมุปฺปนฺโน}
\entry{อธิจฺจสมุปฺปนฺนวาท} {ปุ} {[อธิจฺจสมุปฺปนฺน + วาท]} {อธิจจสมุปปันนทิฏฐิ [บุคคลผู้มีความเห็นผิดคิดว่า อัตตาและโลกเกิดขึ้นโดยไม่มีเหตุ ๒ จำพวก]} {อธิจฺจสมุปฺปนฺนวาท}
\entry{อธิจฺจสมุปฺปนฺนิก} {ติ} {[อธิจฺจสมุปฺปนฺน + อิก]} {ผู้มีความเห็นผิดคิดว่า อัตตาและโลกเกิดขึ้นโดยไม่มีเหตุ} {อธิจฺจสมุปฺปนฺนิกา}
\entry{อธิวุตฺติปท} {นปุ} {[อธิวุตฺติ + ปท]} {สมมุติบัญญัติ ท.} {อธิมุตฺติปทานิ}
\entry{อโธวิเรจน} {นปุ} {[อโธ + วิเรจน]} {การทำให้ลมขับลงข้างล่างร่างกาย} {อโธวิเรจนํ}
\entry{อนคาริย} {อิตฺ\, นปุ} {[ (๑) น + อคาริย]} {=ปพฺพชฺช การบรรพชาอันเป็นเพศที่ไม่มีการครองเรือน} {อนคาริยํ}
\entry{อนงฺคณ} {ติ} {[น + องฺคณ]} {ปราศจากกิเลสแล้ว} {อนงฺคเณ}
\entry{อนตฺตมน} {ติ} {[น + อตฺตมน]} {เป็นผู้ไม่พอใจ} {อนตฺตมนา}
\entry{อนนฺต} {ติ} {[น + อนฺต]} {ธรรมชาติที่ไม่มีที่สิ้นสุด} {อนนฺตํ, อนนฺโต}
\entry{อนนฺตวาหน} {ติ} {[อนนฺต + วาหน]} {พรหมผู้ไม่มีที่สิ้นสุด} {อนนฺตวา}
\entry{อนนฺตสญฺญี} {ติ} {[อนนฺต + สญฺญา + อี]} {ผู้ไม่มีความเชื่อมั่น} {อนนฺตสญฺญี}
\entry{อนภิภูต} {ติ} {[น + อภิภูต]} {เป็นผู้ไม่มีผู้ใดยิ่งใหญ่เหนือกว่า} {อนภิภูโต}
\entry{อนภิรติ} {อิตฺ} {[น + อภิรติ]} {ความไม่เพลิดเพลิน(หรือความว้าเหว่ใจ)} {อนภิรติ}
\entry{อนภิรทฺธิ} {ติ} {[น + อภิรทฺธิ]} {ความไม่ยินดี} {อนภิรทฺธิ}
\entry{อนิจฺจ} {ติ} {[ (๑) น + นิจฺจ]} {เป็นผู้ไม่เที่ยงแท้} {อนิจฺจา}
\entry{อนิยฺยาน} {นปุ} {[น + นิยฺยาน]} {การไม่เสด็จออก} {อนิยฺยานํ}
\entry{อนีกทสฺสน} {นปุ} {[อนิก + ทสฺสน]} {การตรวจกองทัพ} {อนีกทสฺสนํ}
\entry{อนุกสฺสามิ} {ขฺยา} {[อนุ + กร + อ + ติ]} {ไม่มีคำแปล} {อนุสฺสรามิ}
\entry{อนุตฺตร} {ติ} {[น + อุตฺตร]} {อนุตตรสังคามวิชัย [ตำราพิชัยสงครามชนะมาร]} {อนุตฺตโร}
\entry{อนุปาทาวิมุตฺต} {ติ} {[อนุปาทาย + วิมุตฺต]} {เป็นผู้หลุดพ้นแล้วเพราะไม่มีการยึดมั่นถือมั่นในสังขารใดๆ} {อนุปาทาวิมุตฺโต}
\entry{อนุปฺปตฺต} {ติ} {[อนุ + ป + อป + ต]} {เสด็จมาถึง} {อนุปฺปตฺโต}
\entry{อนุปฺปทาตุ} {ติ} {[อนุ + ป + ทา + ตุ]} {เป็นผู้ให้การสนับสนุน} {อนุปฺปทาตา}
\entry{อนุปฺปทาน} {นปุ} {[อนุ + ป + ทา + ยุ]} {การให้} {อนุปฺปทานํ}
\entry{อนุพนฺธ} {ติ} {[อนุ + พนฺธ + อ]} {ผู้ติดตาม} {อนุพนฺธา}
\entry{อนุยุตฺต} {อิตฺ} {[อนุ + ยุช + ต]} {เป็นผู้กระทำ} {อนุยุตฺตา}
\entry{อนุโยค} {ปุ} {[อนุ + ยุช + ณ]} {ความเพียรอย่างสม่ำเสมอ} {อนุโยคํ}
\entry{อนุโยคปริเชคุจฺฉา} {อิตฺ} {[อนุโยค + ปริเชคุจฺฉา]} {เพราะรังเกียจที่จะให้ผู้อื่นซักถาม} {อนุโยคปริเชคุจฺฉา}
\entry{อนุโยคภย} {นปุ} {[อนุโยค + ภย]} {กลัวการซักถามจากผู้อื่น} {อนุโยคภยา}
\entry{อนุสฺสรติ} {ขฺยา} {[อนุ + สร + อ + ติ]} {ย่อมระลึกได้} {อนุสฺสรติ}
\entry{อเนก} {ติ} {[น + เอก]} {มิใช่หนึ่ง} {อเนกานิ}
\entry{อเนกปริยาย} {ปุ} {[อเนก + ปริยาย]} {ข้อกล่าวหาต่างๆ นานา} {อเนกปริยาเยน}
\entry{อเนกวิหิต} {ติ} {[อเนก + วิหิต]} {อันมีประการต่าง ๆ} {อเนกวิหิตํ, อเนกวิหิตานิ}
\entry{อนฺตรา} {พฺยย} {[อนฺต + อิ + อ]} {ในระหว่าง} {อนฺตรา}
\entry{อนฺตรากถา} {อิตฺ} {[อนฺตรา + กถา]} {อันตรากถา} {อนฺตรากถา}
\entry{อนฺตราย} {ปุ} {[อนฺตร + อา + ยา + อ]} {อันตราย} {อนฺตราโย}
\entry{อนฺตลิกฺขจร} {ติ} {[อนฺตลิกฺข + จร + อ]} {ผู้ท่องเที่ยวไปในอากาศ} {อนฺตลิกฺขจรา, อนฺตลิกฺขจโร}
\entry{อนฺตสญฺญี} {ติ} {[อนฺต (๒) + สญฺญา + อี]} {เป็นผู้มีความเชื่อมั่นว่ามีที่สุดในโลก} {อนฺตสญฺญี}
\entry{อนฺตานนฺต} {ติ} {[อนฺต + อนนฺต]} {ความมีที่สุดและไม่มีที่สุด} {อนฺตานนฺตํ}
\entry{อนฺตานนฺตวาท} {ปุ} {[อนฺตานนฺต (๑) + วาท]} {อันตานันตทิฏฐิ ๔ [บุคคลผู้มีความเห็นผิดคิดว่า โลกมีที่สุดและไม่มีที่สุด ๔ จำพวก]} {อนฺตานนฺตวาท}
\entry{อนฺตานนฺติก} {ติ} {[อนฺตานนฺต + อิก]} {ผู้มีความเห็นผิดคิดว่า โลกมีที่สุดและไม่มีที่สุด} {อนฺตานนฺติกา}
\entry{อนฺเตวาสี} {ปุ} {[อนฺเต + วส + ณี]} {ผู้เป็นศิษย์} {อนฺเตวาสินา, อนฺเตวาสี}
\entry{อนฺโตชาลีกต} {ติ} {[อนฺโตชาล + อี + กต]} {จักตั้งอยู่ภายในแหนี้} {อนฺโตชาลีกตา}
\entry{อนฺนกถา} {อิตฺ} {[อนฺน + กถา]} {คำพูดที่ไร้สาระเกี่ยวกับอาหาร} {อนฺนกถํ}
\entry{อนฺนสนฺนิธิ} {ปุ} {[อนฺน + สนฺนิธิ]} {การเก็บสะสมอาหาร} {อนฺนสนฺนิธิํ}
\entry{อนฺวาย} {พฺยย} {[อนุ + อิ + ตฺวา]} {อาศัยแล้ว} {อนฺวาย}
\entry{อปยาน} {นปุ} {[อป + ยา + ยุ]} {ล่าถอย} {อปยานํ}
\entry{อปรนฺต} {ปุ} {[อปร + อนฺต]} {ส่วนแห่งขันธ์ ๕ ที่เป็นอนาคต} {อปรนฺตํ}
\entry{อปรนฺตกปฺปิก} {ติ} {[อปรนฺต + กปฺป + ณิ + ก]} {ผู้มีความเห็นผิดคิดเรื่องที่เกี่ยวกับส่วนแห่งขันธ์ ๕ ที่เป็นอนาคตไปตามอำนาจแห่งตัณหาและทิฏฐิของตนด้วย} {อปรนฺตกปฺปกา, อปรนฺตกปฺปิกา}
\entry{อปรนฺตานุทิฏฺฐิ} {ติ} {[อปรนฺต + อนุ + ทิฏฺฐิ]} {ผู้มีความเห็นผิดคิดเรื่องที่เกี่ยวกับส่วนแห่งขันธ์ ๕ ที่เป็นอนาคตอยู่เนืองนิตย์} {อปรนฺตานุทิฏฺฐิโน}
\entry{อปรามสนฺต} {ติ} {[น + ปรามสนฺต]} {ผู้ไม่หลงยึดติดโดยผิดเพี้ยน} {อปรามสโต}
\entry{อปริยนฺต} {ติ} {[น + ปริยนฺต]} {ไม่มีขอบเขต} {อปริยนฺโต}
\entry{อปสฺสต} {ติ} {[อป + สิ + ต]} {ผู้ไม่เห็น} {อปสฺสตํ}
\entry{อปาส} {ติ} {[น + ปาส]} {ไม่มีคำแปล} {อาปาสึ}
\entry{อปิ} {พฺยย} {[น + ปิ + กฺวิ\, ธาน]} {ก็ดี} {อปิ}
\entry{อปฺปจฺจย} {ปุ} {[น + ปจฺจย]} {(๑) อันไม่ใช่เหตุ\, อันมิได้เป็นเหตุ
(๒) (ติ) 
     (ก) อันไม่มีเหตุ(ปราศจากเหตุ) ในการเกิดขึ้น\, ในการตั้งอยู่
 อันไม่มีเหตุ กล่าวคือ กรรม จิต อุตุ และอาหาร  
อุ. อปฺปจฺจยา ธมฺมา.(อภิ.ธ. ๔)\,  นตฺถิ เอเตสํ อุปฺปาเท วา ฐิติยํ วา ปจฺจโยติ อปฺปจฺจยา.(อภิ.อฏฺ. ๑/๑๙๐)
     (ข) ผู้ที่ไม่มีเหตุ\, ผู้ที่ปราศจากเหตุ 
อุ. อเหตู อปฺปจฺจยา สตฺตา สํกิลิสฺสนฺติ.(ที. ๑/๕๐\, ม.๒/๗๐\,๑๘๕ ; สํ.๒/๑๗๑)
     (ค) อันไม่มีเชื้อ\, อันปราศจากเชื้อ
     (ฆ) อันไม่มีเหตุ เช่น ตัณหา เป็นต้น
     (ง) อันไม่มีสิ่งเกื้อหนุน
(๓) (ปุํ) ธรรมอันมิได้เป็นเหตุนำไปสู่ความภาคภูมิใจ นั่นคือ โทมนัส
     อนึ่ง ในทีฆนิกายฎีกา(ที.ฏี.๑/๒๐๐) อธิบายว่า คำว่า อปฺปจฺจย โทมนสฺส และ เจตสิกทุกฺข ทั้งสามคำนี้เป็นไวพจน์กัน} {อปฺปจฺจโย}
\entry{อปฺปชานนฺต} {ติ} {[น + ปชานนฺต]} {ผู้ไม่รู้\, ผู้ไม่รู้ไม่เห็น\, ผู้ไม่รู้เห็นด้วยปัญญา
อุ. ยถาภูตํ อปฺปชานนฺโต. ผู้ไม่รู้ตามความเป็นจริง(ที.๑/๒๔)} {อปฺปชานนฺโต}
\entry{อปฺปทุฏฺฐจิตฺต} {ติ} {[น + ปทุฏฺฐ + จิตฺต]} {ผู้มีจิตไม่ถูกทำร้าย\, ผู้ไม่มีจิตที่ถูกร้าย\, ผู้มีจิตไม่ถูกประทุษร้าย\, ผู้ไม่มีจิตที่ถูกทำร้าย.
อุ. เต อญฺญมญฺญํ อปฺปทุฏฺฐจิตฺตา. เธอเหล่านั้น ผู้มีจิตไม่ประทุษร้ายต่อกันและกัน (ที.๑/๑๙)} {อปฺปทุฏฺฐจิตฺตา}
\entry{อปฺปมตฺตก} {ติ} {[อปฺป + มตฺตา + ก]} {อันมีปริมาณเล็กน้อย\, อันเล็กน้อย

หากเป็นศัพท์ในอิตถีลิงค์ ใช้รูปว่า อปฺปมตฺติกา
อุ. อปฺปมตฺตกํ โข ปเนตํ ภิกฺขเว โอรมตฺตกํ สีลมตฺตกํ. ดูก่อนภิกษุ ท. ศีลนี้ เป็นเพียงคุณธรรมเล็กน้อยเท่านั้น\, เป็นเพียงคุณธรรมขั้นต้นเท่านั้น\, เป็นเพียงข้อบัญญัติเล็กน้อยเท่านั้น} {อปฺปมตฺตกํ}
\entry{อปฺปมาณสญฺญี} {ติ} {[ (๑) อปฺปมาณ + สญฺญา + อี]} {เป็นธรรมชาติที่มีสัญญาอันไพบูลย์} {อปฺปมาณสญฺญี}
\entry{อปฺปมาท} {ปุ} {[น + ปมาท]} {ความไม่ประมาท} {อปฺปมาทํ}
\entry{อปฺปายุกตร} {ติ} {[อปฺปายุก + ตร]} {เป็นผู้มีอายุสั้นกว่า} {อปฺปายุกตรา}
\entry{อปฺปายุกา} {ติ} {ไม่มีการประกอบคำ} {ผู้มีอายุสั้น} {อปฺปายุกา}
\entry{อปฺเปสกฺขตร} {ติ} {[อปฺเปสกฺข + ตร]} {เป็นผู้มีฤทธิ์เดชน้อยกว่าพรหมผู้เกิดก่อนด้วย} {อปฺเปสกฺขตรา}
\entry{อพฺภนฺตร} {นปุ} {[อภิ + อนฺตร]} {ภายใน} {อพฺภนฺตรานํ}
\entry{อพฺภุชฺชลน} {นปุ} {[อภิ + อุชฺชลน]} {การร่ายมนต์พ่นไฟ} {อพฺภุชฺชลนํ}
\entry{อพฺภุต} {ติ} {[น + ภูต]} {ช่างน่าพิศวงจริงหนอ!} {อพฺภุตํ}
\entry{อพฺรหฺมจริย} {นปุ} {[อพฺรหฺม + จริย\, อพฺรหฺมานํ นิหีนานํ อพฺรหฺมํ วา นิหีนํ จริยํ วุตฺติ อพฺรหฺมจริยํ\, เมถุนธมฺโม]} {การประพฤติที่ไม่ประเสริฐ[การเสพเมถุน]} {อพฺรหฺมจริยํ}
\entry{อภิญฺญา} {พฺยย} {[อภิ + ญา + ตฺวา]} {รู้แล้ว} {อภิญฺญา}
\entry{อภินนฺทุํ} {ขฺยา} {[อภิ + นนฺท + อุํ]} {อนุโมทนายอมรับแล้ว} {อภินนฺทุํ}
\entry{อภิภู} {ปุ} {[อภิ + ภู + กฺวิ]} {เป็นเจ้าเหนือหัวของเรา ท.} {อภิภู}
\entry{อภิวทนฺติ} {ขฺยา} {[อภิ + วท + อ + อนฺติ]} {ย่อมกล่าวประกาศ} {อภิวทนฺติ}
\entry{อภูต} {ติ} {[น + ภูต]} {คำพูดที่ไม่เป็นจริง} {อภูตํ}
\entry{อมนสิการ} {ปุ} {[น + มนสิการ]} {การไม่ใส่ใจหรือพิจารณา} {อมนสิการา}
\entry{อมราวิกฺเขป} {ปุ} {[อมรา + วิกฺเขป]} {อันมีลักษณะซัดส่ายหรือปฏิเสธคัดค้านโดยไม่มีที่สิ้นสุด} {อมราวิกฺเขปํ}
\entry{อมราวิกฺเขปวาท} {ปุ} {[อมราวิกฺเขป + วาท]} {คำพูดอันมีลักษณะซัดส่ายหรือปฏิเสธคัดค้านโดยไม่มีที่สิ้นสุด} {อมราวิกฺเขปวาท}
\entry{อมราวิกฺเขปิก} {ติ} {[ (๑) อมราวิกฺเขป + อิก]} {ผู้มีความเห็นและวาจาที่ซัดส่ายหรือปฏิเสธคัดค้านโดยไม่มีที่สิ้นสุด} {อมราวิกฺเขปิกา}
\entry{อมุตฺร} {พฺยย} {[อมุ + ตฺร]} {ฝ่ายโน้น} {อมุตฺร}
\entry{อมฺพปิณฺฑิย} {ปุ} {[อมฺพปิณฺฑิ + อิย]} {พวงแห่งมะม่วง} {อมฺพปิณฺฑิยา}
\entry{อมฺพลฏฺฐิกา} {อิตฺ} {[อมฺพ + ลฏฺฐิกา]} {พระราชอุทยานอันมีนามเรียกขานว่า อัมพลัฏฐิกา} {อมฺพลฏฺฐิกายํ}
\entry{อมฺห} {อ} {ไม่มีการประกอบคำ} {ย่อมเป็น} {อมฺห, อมฺหากํ, อหํ, โน, มํ, มม, มมํ, มยํ, มยา, เม}
\entry{อมฺหิ} {ขฺยา} {[อส + อ + มิ]} {ย่อมเป็น} {อมฺหิ}
\entry{อรหโต} {ปุ} {ไม่มีการประกอบคำ} {ผู้ทรงกำจัดศัตรูกล่าวคือกิเลส (หรือผู้สมควรแก่การบูชาและรับทักษิณาทาน เป็นต้น)} {อรหตา}
\entry{อริย} {ติ} {[ (๑) อารกา-สทฺทากิุ นิรุตฺตินญฺญะผฺรง့ อริย-ปฺรุ]} {พระอริยะ ท. (หรือสัตบุรุษ ท. มีพระสัมมาสัมพุทธเจ้า เป็นต้น)} {อริยา}
\entry{อรูปี} {ติ} {[ (๑) น + รูปี]} {เป็นธรรมชาติที่ไม่มีรูปด้วย} {อรูปี}
\entry{อโรค} {ติ} {[น + โรค]} {เป็นธรรมชาติที่เที่ยงแท้ไม่เสื่อมสลาย} {อโรโค}
\entry{อวจ} {ติ} {[อว + จิ + อ¿]} {กล่าวแล้ว} {อวจ}
\entry{อวณฺณ} {ติ} {[น + วณฺณ]} {คำติเตียน} {อวณฺณํ}
\entry{อวิจาร} {ติ} {[น + วิจาร]} {อันไม่มีวิจาร[การพิจารณาไต่สวน]} {อวิจารํ}
\entry{อวิตกฺก} {ติ} {[น + วิตกฺก]} {อันไม่มีวิตก[การตรึกนึกคิด]} {อวิตกฺกํ}
\entry{อวิปริณามธมฺม} {ติ} {[อวิปริณาม + ธมฺม]} {จักเป็นผู้ไม่มีการเปลี่ยนแปลงสถานภาพ} {อวิปริณามธมฺมา, อวิปริณามธมฺโม}
\entry{อวิสํวาทก} {ติ} {[น + วิสํวาทก\, ถี-]} {เป็นผู้ไม่หลอกลวง} {อวิสํวาทโก}
\entry{อโวจ} {ขฺยา} {[อ + วจ + อา]} {ได้ตรัสแล้ว} {อโวจ}
\entry{อโวจุํ} {ขฺยา} {[อ + วจ + อุํ]} {กราบทูลแล้ว} {อโวจุํ}
\entry{อสญฺญสตฺต} {ปุ} {[อสญฺญ + สตฺต]} {อสัญญสัตตพรหม [พรหมผู้ไม่มีสัญญา]} {อสญฺญสตฺตา}
\entry{อสญฺญิมตฺต} {ติ} {[น + สญฺญิมตฺต]} {ไม่มีคำแปล} {อสญฺญิมตฺตานํ}
\entry{อสญฺญิวาท} {ปุ} {[อสญฺญี + วาท]} {ผู้มีความเห็นผิดคิดว่า ชีวิตหลังความตายไม่มีสัญญา} {อสญฺญีวาท, อสญฺญีวาทา}
\entry{อสญฺญี} {ติ} {[น + สญฺญา + อี]} {ธรรมชาติที่ไม่มีสัญญา} {อสญฺญี}
\entry{อสมฺโมส} {ปุ} {[น + สมฺโมส]} {การไม่หลงลืม} {อสมฺโมสา}
\entry{อสสฺสต} {ปุ} {[น + สสฺสต]} {ว่าเป็น สิ่งไม่เที่ยงแท้} {อสสฺสตํ, อสสฺสโต}
\entry{อสหิต} {ติ} {[น + สหิต]} {ไม่สละสลวย ไม่มีประโยชน์ ไร้เหตุผล} {อสหิตํ}
\entry{อสิ} {ขฺยา} {[อส + อ + อี]} {ย่อมเป็น} {อสิ}
\entry{อสิํ} {ขฺยา} {[อส + อ + อึ]} {การสะพายดาบอันคมกริบ} {อสิํ}
\entry{อสิลกฺขณ} {นปุ} {[อสิ + ลกฺขณ]} {ศาสตร์ว่าด้วยการทำนายลักษณะของดาบ} {อสิลกฺขณํ}
\entry{อสฺมิ} {พฺยย} {[อส + มิ]} {ย่อมเป็น} {อสฺมิ}
\entry{อสฺสตฺถร} {ปุ} {[อสฺส + อตฺถร]} {เครื่องลาดบนหลังม้า} {อสฺสตฺถรํ}
\entry{อสฺสยุทฺธ} {นปุ} {[อสฺส + ยุทฺธ]} {การแข่งขันชนม้า} {อสฺสยุทฺธํ}
\entry{อสฺสรถ} {ปุ} {[อสฺส + รถ]} {รถม้า} {อสฺสถ}
\entry{อสฺสลกฺขณ} {นปุ} {[อสฺส + ลกฺขณ]} {ศาสตร์ว่าด้วยการทำนายลักษณะของม้า} {อสฺสลกฺขณํ}
\entry{อสฺสาท} {ติ} {[ (๑) น + สาท\, (๒) อา + สท + อ]\, อสฺสาท (ปุ) [อา + สท + ณ]} {อัสสาทะ} {อสฺสาทํ}
\entry{อหนฺตฺวา} {พฺยย} {[น + หนฺตฺวา]} {ไม่มีคำแปล} {อนฺตวา}
\entry{อหิวิชฺชา} {อิตฺ} {[อหิ + วิชฺชา]} {ศาสตร์ว่าด้วยการรักษาพิษงูหรือศาสตร์ว่าด้วยเวทมนต์เรียกงู} {อหิวิชฺชา}
\entry{อหีนินฺทฺริย} {ติ} {[อหีน + อินฺทฺริย]} {อันมีอินทรีย์ที่ไม่บกพร่อง} {อหีนินฺทฺริโย}
\entry{อหุตฺวา} {พฺยย} {[น + หุตฺวา]} {ไม่มี} {อหุตฺวา}
\entry{อหุมฺห} {ขฺยา} {[อ + หู + มฺห]} {ได้เป็นแล้ว} {อหุมฺหา}
\entry{อโห} {พฺยย} {[น + หา + โอ]} {ก็คงจะวิเศษทีเดียว} {อโห}
\entry{อโหสิ} {ขฺยา} {[อ + หู + อี]} {ได้มีแล้ว} {อโหสิ}
\end{multicols}
\newpage
\needspace{5cm} \section*{\Huge อา}
\noindent\rule{\textwidth}{0.4pt}
\begin{multicols}{2}
\entry{อากาส} {ปุ\, นปุ} {[อากาส + กีฬน]} {การพนันหมากรุกในอากาศ} {อากาสํ, อากาโส}
\entry{อากาสานญฺจายตน} {ติ} {[อากาสานญฺจ + อายตน]} {ซึ่งอากาสานัญจายตนฌาน[อารมณ์ที่เป็นอากาศอันได้มาด้วยการเพิกกสิณที่ไม่มีขอบเขตแห่งการเกิดขึ้นตั้งอยู่และดับไป]} {อากาสานญฺจายตนํ}
\entry{อากาสานญฺจายตนุปค} {ติ} {[อากาสานญฺจายตน + อุปค (=อุป + คมุ + กฺวิ)]} {อันเข้าถึงอากาสานัญจายตนภูมิ(หรือผู้บังเกิดในอากาสานัญจายตนภูมิ)} {อากาสานญฺจายตนูปโค}
\entry{อากิญฺจญฺญายตน} {ติ} {[อากิญฺจญฺญ + อายตน]} {อากิญจัญญายตนฌาน} {อากิญฺจญฺญายตนํ}
\entry{อากิญฺจญฺญายตนูปค} {ติ} {[อากิญฺจญฺญายตน + อุป + คมุ + กฺวิ]} {อันเข้าถึงอากิญจัญญายตนภูมิ} {อากิญฺจญฺญายตนูปโค}
\entry{อาคจฺฉติ} {ขฺยา} {[อา + คมุ + อ + ติ]} {ย่อมมา} {อาคจฺฉติ}
\entry{อาคจฺเฉยฺยุํ} {ขฺยา} {[อา + คมุ + อ + เอยฺยุํ]} {พึงมา} {อาคจฺเฉยฺยุํ}
\entry{อาคต} {ติ} {[อา + คมุ + ต]} {ได้มาแล้ว} {อาคตา, อาคโต}
\entry{อาคมฺม} {พฺยย} {[อา + คมุ + ตฺวา]} {อาศัยแล้ว} {อาคมฺม}
\entry{อาฆาต} {ปุ} {[อา + หน + ต]} {ความผูกโกรธ} {อาฆาโต}
\entry{อาจมน} {นปุ} {[อา + จมุ + ยุ]} {การพ่นน้ำมนต์[การล้างปากให้หมดจดด้วยน้ำมนต์]} {อาจมนํ}
\entry{อาจริยนฺเตวาสี} {ปุ} {[อาจริย + อนฺเตวาสี]} {อาจารย์และศิษย์ ท.} {อาจริยนฺเตวาสี}
\entry{อาจิกฺขนฺติ} {ติ} {[อา + จิกฺข + อ + อนฺต]} {ย่อมกล่าวสรรเสริญ} {อาจิกฺขนฺติ}
\entry{อาชานามิ} {ขฺยา} {[อา + ญา + นา + มิ]} {อาชานามิ} {อาชานามิ}
\entry{อาชานาสิ} {ขฺยา} {[อา + ญา + นา + สิ]} {ย่อมเข้าใจ} {อาชานาสิ}
\entry{อาชานิสฺสสิ} {ขฺยา} {[อา + ญา + นา + สฺสสิ]} {จักเข้าใจ} {อาชานิสฺสสิ}
\entry{อาตปฺป} {นปุ} {[อา + ตป + ณฺย]} {ความเพียรอย่างยิ่งยวด} {อาตปฺปํ}
\entry{อาทาส} {ปุ} {[อา + ทิส + ณ]} {การส่องกระจก} {อาทาสํ}
\entry{อาทาสปญฺห} {ติ} {[อาทาส + ปญฺห]} {การเป็นหมอดูลูกแก้ว[การเรียกเทวดามาถามปัญหาทางกระจก]} {อาทาสปญฺหํ}
\entry{อาทิจฺจุปฏฺฐาน} {นปุ} {[อาทิจฺจ + อุปฏฺฐาน]} {การนั่งบูชาดวงอาทิตย์} {อาทิจฺจุปฏฺฐานํ}
\entry{อาทีนว} {ปุ} {[ (๑) อาทีนว + ทสฺสก\, โนกปุทกฺเย]} {อาทีนวะ} {อาทีนวํ}
\entry{อานนฺท} {ปุ} {[อา + นนฺท + อ]} {ดูก่อนอานนท์} {อานนฺท, อานนฺโท}
\entry{อานนฺที} {ติ} {[อา + นนฺท + ณี]} {ผู้มีความยินดี} {อานนฺทิโน}
\entry{อาปชฺชติ} {ขฺยา} {[อา + ปท + ย + ติ]} {ย่อมถึง} {อาปชฺชติ}
\entry{อาปชฺชนฺติ} {ขฺยา} {[อา + ปท + ย + อนฺติ]} {ย่อมถึง} {อาปชฺชนฺติ}
\entry{อาภสฺสรกาย} {ปุ} {[อาภสฺสร + กาย]} {พรหมโลกชั้นอาภัสสรภูมิ} {อาภสฺสรกายา}
\entry{อาภสฺสรสํวตฺตนิก} {ติ} {[อาภสฺสร + สํวตฺตนิก]} {เกิดเป็นพรหมในชั้นอาภัสสรภูมิ} {อาภสฺสรสํวตฺตนิกา}
\entry{อาโภค} {ปุ} {[อา + ภุช + ณ]} {ความประสงค์\, การเอาใจใส่} {อาโภโค}
\entry{อามกธญฺญปฏิคฺคหณ} {นปุ} {[อามกธญฺญ + ปฏิคฺคหณ]} {การรับธัญญพืช} {อามกธญฺญปฏิคฺคหณา}
\entry{อามกมํสปฏิคฺคหณ} {นปุ} {[อามกมํส + ปฏิคฺคหณ]} {การรับเนื้อดิบ
อามนฺเตสิ ตรัสถามแล้ว} {อามกมํสปฏิคฺคหณา}
\entry{อามนฺเตสิ} {ขฺยา} {[อา + มนฺต เณ + สิ]} {ตรัสถามแล้ว} {อามนฺเตสิ}
\entry{อามิสสนฺนิธิ} {ปุ} {[อามิส + สนฺนิธิ]} {การเก็บสะสมอามิส} {อามิสสนฺนิธิํ}
\entry{อายสฺมา} {ปุ} {ไม่มีการประกอบคำ} {ผู้มีอายุ} {อายสฺมา}
\entry{อายุกฺขย} {ปุ} {[อายุ + ขย]} {การสิ้นอายุเป็นเหตุ} {อายุกฺขยา}
\entry{อายุปริยนฺต} {ปุ} {[อายุ + ปริยนฺต]} {ไม่มีคำแปล} {อายุปริยนฺโต}
\entry{อารพฺภ} {พฺยย} {[อา + รภ + ตฺวา]} {อิงแล้ว} {อารพฺภ}
\entry{อาราจารี} {ติ} {[อารา + จร + ณี]} {ผู้มีปกติประพฤติธรรมห่างไกลจากการเสพเมถุน} {อาราจารี}
\entry{อาโรคฺย} {นปุ} {[อโรค + ณฺย]} {สุขภาพพลานามัยที่ดี} {อาโรคฺยํ}
\entry{อาโรปิต} {ติ} {[อา + รุห + เณ + ต]} {ตำหนิ} {อาโรปิโต}
\entry{อาวหน} {นปุ} {[อา + วห + ยุ]} {การดูฤกษ์อาวาหมงคล[พิธีที่เจ้าบ่าวนำเจ้าสาวเข้ามาในบ้าน]} {อาวาหนํ}
\entry{อาวุธลกฺขณ} {นปุ} {[อาวุธ + ลกฺขณ]} {ศาสตร์ว่าด้วยการทำนายลักษณะของอาวุธ} {อาวุธลกฺขณํ}
\entry{อาวุโส} {พฺยย} {[รหนะอขฺยงะขฺยงะ โขโวรา]} {ข้าแต่ท่าน ท.} {อาวุโส}
\entry{อาสน} {นปุ} {[อาส + ยุ]} {อาสนะ} {อาสเน}
\entry{อาสนฺที} {อิตฺ} {[อา + สท + อ + อี]} {ตั่งหรือเก้าอี้อันมีขาสูงยาวเกินขนาด} {อาสนฺทิํ}
\entry{อาห} {ขฺยา} {[พฺรู + อ + ติ]} {ย่อมกล่าว} {อาห}
\entry{อาหร} {ขฺยา} {[อา + หร + อ + หิ]} {จงนำมา} {อาหร}
\end{multicols}
\newpage
\needspace{5cm} \section*{\Huge อิ}
\noindent\rule{\textwidth}{0.4pt}
\begin{multicols}{2}
\entry{อิติ} {พฺยย} {[อิ + อ + ติ]} {ด้วยประการฉะนี้} {อิติ}
\entry{อิติภวาภวกถา} {อิตฺ} {[อิติ + ภว + อภว + กถา]} {คำพูดที่ไร้สาระเกี่ยวกับโลกเจริญเพราะเหตุนี้และโลกเสื่อมเพราะเหตุนี้} {อิติภวาภวกถํ}
\entry{อิติห} {ติ} {[อิติห (=อิติ+ห)+ณ]\, อิติห (พฺย) [อิติ+ห]} {=เอวํ ด้วยประการฉะนี้} {อิติห}
\entry{อิโต} {พฺยย} {[ (๑) อิม + โต]} {จากที่นี้} {อิโต}
\entry{อิตฺถตฺต} {นปุ} {[อิตฺถํ + ตฺต]} {มนุษยโลกนี้} {อิตฺถตฺตํ}
\entry{อิตฺถิกถา} {อิตฺ} {[อิตฺถี + กถา]} {คำพูดที่ไร้สาระเกี่ยวกับสตรี} {อิตฺถิกถํ}
\entry{อิตฺถิกุมาริกปฏิคฺคหณ} {นปุ} {[อิตฺถี + กุมาริกา + ปฏิคฺคหณ]} {การรับสตรีและหญิงสาว} {อิตฺถิกุมาริกาปฏิคฺคหณา}
\entry{อิตฺถิลกฺขณ} {นปุ} {[อิตฺถี + ลกฺขณ]} {ศาสตร์ว่าด้วยการทำนายลักษณะของสตรี} {อิตฺถิลกฺขณํ}
\entry{อิธ} {ปุ} {[อิธ + อ]} {(โลเก) ในโลกนี้} {อิธ}
\entry{อิม} {นปุ} {ไม่มีการประกอบคำ} {นี้} {อยํ, อสฺส, อิทํ, อิมํ, อิมสฺมิํ, อิมสฺส, อิมินา, อิเม, อิเมสํ, อิเมหิ}
\entry{อิสฺสร} {ติ} {[ (๑) อีส + อร]} {ผู้ครอบครองโลก} {อิสฺสโร}
\end{multicols}
\newpage
\needspace{5cm} \section*{\Huge อุ}
\noindent\rule{\textwidth}{0.4pt}
\begin{multicols}{2}
\entry{อุกฺกาปาต} {ปุ\, นปุ} {[อุกฺกาปาต + ณ]} {การตกแห่งลูกอุกกาบาต} {อุกฺกาปาโต}
\entry{อุคฺคมนํ} {นปุ} {[อุ + คมุ + ยุ]} {การขึ้น} {อุคฺคมนํ}
\entry{อุจฺจาสยนมหาสยน} {นปุ} {[อุจฺจาสยน + มหาสยน]} {การใช้สอยเสนาสนะอันสูงใหญ่และเสนาสนะอันประณีต} {อุจฺจาสยนมหาสยนํ, อุจฺจาสยนมหาสยนา}
\entry{อุจฺฉาทน} {นปุ} {[อุ + ฉท + ยุ]} {การอบผิวด้วยผงไม้จันทน์} {อุจฺฉาทนํ}
\entry{อุจฺฉิชฺชติ} {ขฺยา} {[อุ + ฉิทิ + ย + ติ]} {ย่อมขาดสูญ} {อุจฺฉิชฺชติ}
\entry{อุจฺฉินฺนภวเนตฺติก} {ติ} {[อุ + ฉิท (ฉิทิ) + ต]} {มีภวตัณหา[ตัณหาอันเปรียบเสมือนเชือกดึงสัตว์ไปสู่ภพ] อันถูกตัดขาดแล้วด้วยดาบกล่าวคืออรหัตมรรค} {อุจฺฉินฺน}
\entry{อุจฺฉินฺนภวเนตฺติโก} {ติ} {ไม่มีการประกอบคำ} {มีภวตัณหา[ตัณหาอันเปรียบเสมือนเชือกดึงสัตว์ไปสู่ภพ] อันถูกตัดขาดแล้วด้วยดาบกล่าวคืออรหัตมรรค} {อุจฺฉินฺนภวเนตฺติโก}
\entry{อุจฺเฉท} {ปุ} {[อุ + ฉิท (=ฉิทิ) + ณ]} {การขาดสูญ} {อุจฺเฉทํ}
\entry{อุจฺเฉทวาท} {ปุ} {[ (๑) อุจฺเฉท + วาท]} {อุจเฉททิฏฐิ ๗ [บุคคลผู้มีความเห็นผิดคิดว่า ตายแล้วสูญ ๗ จำพวก]} {อุจฺเฉทวาท, อุจฺเฉทวาทา}
\entry{อุชุวิปจฺจนีกวาท} {ติ} {[อุชุวิปจฺจนีก + วาท]} {ผู้มีคำพูดขัดแย้งที่ตรงกันข้าม} {อุชุวิปจฺจนีกวาทา}
\entry{อุณฺหีส} {นปุ} {[อุณฺหีส + อ]} {=อุณฺหีสธารณํ การสวมกรอบหน้า} {อุณฺหีสํ}
\entry{อุตฺตริตร} {ติ} {[อุตฺตร + ตร]} {ธรรมที่ประเสริฐกว่า} {อุตฺตริตรํ}
\entry{อุตฺราสี} {ติ} {[อุ + ตฺรส + ณี]} {ไม่มีคำแปล} {อมุตฺราสิํ}
\entry{อุทกทห} {ปุ} {[อุทก + ทห]} {หนองน้ำ} {อุทกทหํ, อุทกทเห}
\entry{อุทปาทึ} {ขฺยา} {[อุ + ปท + อึ]} {เกิดแล้ว} {อุทปาทิํ}
\entry{อุทปาทิ} {ขฺยา} {[อุ + ปท + อี]} {ได้เกิดขึ้นแล้ว} {อุทปาทิ}
\entry{อุทฺทโลมี} {ติ} {[อุทฺทโลม + อี]} {เครื่องลาดขนแกะมีชาย ๒ ด้าน} {อุทฺธโลมิํ}
\entry{อุทฺธํ} {พฺยย} {[สตฺตมฺยตฺถนิปาตปุท]} {(โลกสฺมึ) อโธ โลกสฺมึ อนฺตสฺี เป็นผู้มีความสำคัญผิดคิดว่า โลกส่วนบนและโลกส่วนล่างมีที่สุด} {อุทฺธํ}
\entry{อุทฺธํวิเรจน} {นปุ} {[อุทฺธํ + วิเรจน]} {การทำให้ลมเรอขึ้นข้างบนร่างกาย} {อุทฺธวิเรจนํ}
\entry{อุทฺธมาฆาตน} {ปุ\, นปุ} {[ (๑) อุทฺธํ + อาฆาตน]\, [อุทฺธมาฆาตน (น) [อุทฺธํ + อาฆาตน]} {อันมีสภาพที่ไม่ตายภายหลังจุติ} {อุทฺธมาฆาตนํ}
\entry{อุทฺธมาฆาตนิก} {ติ} {[อุทฺธมาฆาตน + อิก]} {ผู้มีความเห็นผิดคิดว่า ภายหลังจากจุติจิตดับแล้วชีวิตจะเป็นอมตะ} {อุทฺธมาฆาตนิกา}
\entry{อุปคญฺฉิ} {ขฺยา} {[อุป + คมุ + อี]} {เข้าถึงแล้ว} {อุปคญฺฉิ}
\entry{อุปฏฺฐาน} {นปุ} {[อุป + ฐา + ยุ]} {ไม่มีคำแปล} {อุปฏฺฐานํ}
\entry{อุปนิชฺฌายนฺต} {ติ} {[อุป + นิ + เฌ + อนฺต]} {เพ่งโทษ} {อุปนิชฺฌายนฺตา}
\entry{อุปนิชฺฌายนฺติ} {ขฺยา} {[อุป + นิ + เฌ + อ + อนฺติ]} {ย่อมเพ่งโทษ} {อุปนิชฺฌายนฺติ}
\entry{อุปนิชฺฌายิมฺหา} {ขฺยา} {[อุป + นิ + เฌ + มฺหา]} {เพ่งโทษแล้ว} {อุปนิชฺฌายิมฺหา}
\entry{อุปปชฺชติ} {ขฺยา} {[อุป + ปท + ย + ติ]} {ย่อมเกิดขึ้น} {อุปปชฺชติ}
\entry{อุปปชฺชนฺติ} {ขฺยา} {[อุป + ปท + ย + อนฺติ]} {ย่อมเข้าถึง} {อุปปชฺชนฺติ}
\entry{อุปปนฺน} {ติ} {[อุป + ปท + ต]} {เกิดแล้ว} {อุปปนฺนํ, อุปปนฺนา, อุปปนฺโน}
\entry{อุปยาน} {นปุ} {[อุป + ยา + ยุ]} {การเข้าประชิด} {อุปยานํ}
\entry{อุปสงฺกมิ} {ขฺยา} {[อุป + สํ + กมุ + อี]} {เสด็จเข้าไปแล้ว} {อุปสงฺกมิ}
\entry{อุปสงฺกมิตฺวา} {พฺยย} {ไม่มีการประกอบคำ} {ครั้นเสด็จเข้าไปแล้ว} {อุปสงฺกมิตฺวา}
\entry{อุปสมฺปชฺช} {ขฺยา} {[ (๑) อุป + สํ + ปท + ตฺวา]} {เข้าถึงแล้ว} {อุปสมฺปชฺช}
\entry{อุปาทาน} {นปุ} {[อุป + อา + ทา + ยุ]} {อุปาทาน} {อุปาทานํ}
\entry{อุปาทานปจฺจย} {ปุ} {[อุปาทน + ปจฺจย]} {มีอุปาทานเป็นอุปนิสสยปัจจัยและสหชาตปัจจัย} {อุปาทานปจฺจยา}
\entry{อุปาทานปริเชคุจฺฉา} {อิตฺ} {[อุปาทาน + ปริเชคุจฺฉา]} {รังเกียจอุปาทาน} {อุปาทานปริเชคุจฺฉา}
\entry{อุปาทานภยา} {นปุ} {[อุปทาน + ภย]} {กลัวอุปาทาน} {อุปาทานภยา}
\entry{อุเปกฺขก} {ติ} {[อุป + อิกฺข + ณฺวุ]} {มีความนิ่งเฉย} {อุเปกฺขโก}
\entry{อุเปกฺขาสติปาริสุทฺธิ} {อิตฺ} {[อุเปกฺขา + สติปาริสุทฺธิ]} {มีความบริสุทธิ์แห่งสติซึ่งบังเกิดขึ้นด้วยอำนาจแห่งอุเบกขา} {อุเปกฺขาสติปาริสุทฺธิํ}
\entry{อุปฺปชฺชนฺติ} {ขฺยา} {[อุ + ปท + ย + อนฺติ]} {ย่อมเกิดขึ้น} {อุปฺปชฺชนฺติ}
\entry{อุปฺปถคมน} {นปุ} {[อุปฺปถ + คมน]} {การโคจรนอกเส้นทาง} {อุปฺปถคมนํ}
\entry{อุปฺปาต} {ปุ} {[อุ + ปต + ณ]} {ศาสตร์ว่าด้วยการทำนายโชคลางจากเหตุการณ์ที่เกิดขึ้น} {อุปฺปาตํ}
\entry{อุพฺพิลาวิต} {ติ} {[อุปฺปิ (พฺพิ)ลาว + อิต]} {ผู้ฟูใจ} {อุพฺพิลาวิตา}
\entry{อุพฺพิลาวิตตฺต} {นปุ} {[อุปฺปิ (พฺพิ)ลาวิต + ตฺต]} {ความฟูขึ้น} {อุพฺพิลาวิตตฺตํ}
\entry{อุภ} {ติ} {[อุ + ภ]} {ทั้งสอง} {อุโภ}
\entry{อุภโตโลหิตกูปธาน} {ติ} {[อุภโตโลหิตก + อุปธาน]} {เตียงอันมีหมอนหนุนสีแดงอยู่ทั้ง ๒ ด้าน คือ ทางด้านหัวนอนและปลายเท้า} {อุภโตโลหิตกูปธานํ}
\entry{อุมฺมุชฺชนฺติ} {ขฺยา} {[อุ + มุชฺช + อ + อนฺติ]} {ย่อมผุดขึ้น} {อุมฺมุชฺชนฺติ}
\entry{อุมฺมุชฺชมาน} {ติ} {[อุ + มุชฺช + อ + มาน]} {จมลงอยู่} {อุมฺมุชฺชมานา}
\entry{อุยฺโยธิก} {ติ} {[อุ + ยุธ + ณฺวุ]} {การรบหรือสมรภูมิ} {อุยฺโยธิกํ}
\entry{อุสภยุทฺธ} {นปุ} {[อุสภ + ยุทฺธ]} {การแข่งขันชนวัว} {อุสภยุทฺธํ}
\entry{อุสภลกฺขณ} {นปุ} {[อุสภ + ลกฺขณ]} {ศาสตร์ว่าด้วยการทำนายลักษณะของโคอุสภะ} {อุสภลกฺขณํ}
\entry{อุสุลกฺขณํ} {นปุ} {[อุสุ + ลกฺขณ]} {ศาสตร์ว่าด้วยการทำนายลักษณะของศร} {อุสุลกฺขณํ}
\end{multicols}
\newpage
\needspace{5cm} \section*{\Huge เอ}
\noindent\rule{\textwidth}{0.4pt}
\begin{multicols}{2}
\entry{เอก๑} {ติ} {[อิ + ณฺวุ. อิ + ก. อิ คติยํ. ณฺวุ. อิสฺเส อโลโป จ. เอโก. สทิสรหิตตาย วา เอกัภาเว ติฏฺฐตีติ เอโก. โก.]} {หนึ่ง} {เอเก}
\entry{เอก} {ปุ} {[เอก + โย]} {หนึ่ง} {เอกํ}
\entry{เอกจฺจ} {ติ} {[เอก + จฺจ\,  เอก + อจฺจ]} {บางอย่าง} {เอกจฺจํ, เอกจฺโจ}
\entry{เอกจฺจอสสฺสติก} {ติ} {[เอกจฺจอสสฺสต + อิก]} {ผู้มีความเห็นผิดคิดว่า อัตตาและโลกบางอย่างเป็นสิ่งไม่เที่ยงแท้} {เอกจฺจอสสฺสติกา}
\entry{เอกจฺจสสฺสตวาท} {ปุ} {[เอกจฺจสสฺสต + วาท]} {เอกัจจสัสสตทิฏฐิ ๔ [บุคคลผู้มีความเห็นผิดคิดว่า อัตตาและโลกบางอย่างเป็นสิ่งเที่ยงแท้และบางอย่างเป็นสิ่งไม่เที่ยงแท้ ๔ จำพวก]} {เอกจฺจสสฺสตวาท}
\entry{เอกจฺจสสฺสติก} {ติ} {[เอกจฺจสสฺสต + อิก]} {ผู้มีความเห็นผิดคิดว่า อัตตาและโลกบางอย่างเป็นสิ่งเที่ยงแท้} {เอกจฺจสสฺสติกา}
\entry{เอกตฺตสญฺญี} {ติ} {[เอกตฺตสญฺญา + อี]} {เป็นธรรมชาติที่มีสัญญาเดียว(หรือธรรมชาติที่มีสัญญาอันเป็นไปในอารมณ์เดียว)} {เอกตฺตสญฺญี}
\entry{เอกนฺตทุกฺขี} {ติ} {[เอกนฺตทุกฺข + อี]} {บุคคลผู้มีความทุกข์อย่างเดียว} {เอกนฺตทุกฺขี}
\entry{เอกนฺตโลมี} {ติ} {[เอกนฺต + โลมี]\, เอก + อนฺต + โลม]\, เอกนฺตโลม + อี]} {เครื่องลาดขนแกะมีชายด้านเดียว} {เอกนฺตโลมิํ}
\entry{เอกนฺตสุขี} {ติ} {[เอกนฺต + สุขี]} {พรหมผู้มีความสุขอย่างเดียว} {เอกนฺตสุขี}
\entry{เอกภตฺติก} {ติ} {[เอกภตฺต + อิก]} {ผู้มีปกติฉันภัตตาหารเพียงมื้อเดียวเท่านั้น} {เอกภตฺติโก}
\entry{เอกรตฺติวาส} {ปุ} {เอกรตฺติ + วาส} {การพักแรมตลอดหนึ่งคืน} {เอกรตฺติวาสํ}
\entry{เอโกทิภาว} {ติ} {[เอโกทิ + ภาว]} {อันยังสมาธิอันเลิศให้บังเกิด} {เอโกทิภาวํ}
\entry{เอต} {ติ} {ไม่มีการประกอบคำ} {นี้} {เอเต, เอเตน, เอเตสํ}
\entry{เอตรหิ} {พฺยย} {ไม่มีการประกอบคำ} {ในเวลานี้} {เอตรหิ}
\entry{เอตา} {อิตฺ} {ไม่มีการประกอบคำ} {นั่น} {เอตํ, เอโส}
\entry{เอตฺตาวตา} {พฺยย} {ไม่มีการประกอบคำ} {ด้วยเหตุนี้} {เอตฺตาวตา}
\entry{เอตฺถ} {พฺยย} {ไม่มีการประกอบคำ} {ในธรรมเทศนานี้} {เอตฺถ}
\entry{เอว} {พฺยย} {ไม่มีการประกอบคำ} {นั่นแล} {เอว}
\entry{เอวํ} {พฺยย} {ไม่มีการประกอบคำ} {ด้วยประการฉะนี้} {เอวํ}
\entry{เอวํอภิสมฺปราย} {ติ} {[เอวํ + อภิสมฺปราย]} {ย่อมเป็นเหตุให้ไปสู่ปรโลกอันเป็นทุคติอย่างใดอย่างหนึ่ง} {เอวํอภิสมฺปรายา}
\entry{เอวํคติก} {ติ} {[เอวํ + คติก]} {ย่อมเป็นเหตุให้ได้คติภพอันเป็นทุคติอย่างใดอย่างหนึ่ง} {เอวํคติกา}
\entry{เอวํคหิต} {ติ} {[เอวํ + คหิต]} {อันบุคคลยึดติดแล้วอย่างนี้} {เอวํคหิตา}
\entry{เอวํทิฏฺฐี} {ติ} {[เอวํ + ทิฏฺฐี\, เอวํทิฏฺฐิ + อี]} {เป็นผู้มีทิฏฐิ[ความยึดมั่น]อย่างนี้} {เอวํทิฏฺฐิ}
\entry{เอวํปรามฏฺฐ} {ติ} {[เอวํ + ปรามฏฺฐ]} {อันบุคคลหลงยึดติดแล้วอย่างนี้} {เอวํปรามฏฺฐา}
\entry{เอวํวณฺณ} {ติ} {[เอวํ + วณฺณ]} {ผู้มีผิวพรรณอย่างนี้} {เอวํวณฺโณ}
\entry{เอวํวาที} {ติ} {[เอวํวาท + อี. เอวํ + วท + ณี]} {เป็นผู้มีวาทะ[ความเห็น]อย่างนี้} {เอวํวาที}
\entry{เอวํวิปาก} {ติ} {[เอวํ + วิปาก]} {มีผลเช่นนี้} {เอวํวิปากํ}
\entry{เอวํสุขทุกฺขปฺปฏิสํเวที} {ติ} {[เอวํ + สุขทุกฺข+ ปฏิสํเวที]} {เป็นผู้เสวยสุขทุกข์อย่างนี้} {เอวํสุขทุกฺขปฺปฏิสํเวที}
\entry{เอวมาหาร} {ติ} {[เอวํ + อาหาร]} {เป็นผู้มีอาหารอย่างนี้} {เอวมาหาโร}
\entry{เอวรูป} {ติ} {[เอวํ + รูป]} {อันมีสภาพที่ไม่เหมาะสมเช่นนี้} {เอวรูปํ, เอวรูปา, เอวรูปาย}
\entry{เอสิกฏฺฐายิฏฺฐิต} {ติ} {[เอสิกฏฺฐายี + ฐิต]} {เป็นธรรมชาติที่ตั้งมั่นอยู่ดุจเสาเขื่อนอันมั่นคง} {เอสิกฏฺฐายิฏฺฐิโต}
\end{multicols}
\newpage
\needspace{5cm} \section*{\Huge โอ}
\noindent\rule{\textwidth}{0.4pt}
\begin{multicols}{2}
\entry{โอคมน} {นปุ} {[อว + คมุ + ยุ]} {การตก} {โอคมนํ}
\entry{โอตฺถเรยฺย} {ขฺยา} {[อว + ถร + เอยฺย]} {พึงครอบ} {โอตฺถเรยฺย}
\entry{โอปปาติก} {ติ} {[อุปปาต + ณิก\, อุปปาติก + ณ]} {ผู้เกิดมาในภพใหม่ด้วยการตายจากภพเก่า} {โอปปาติกา}
\entry{โอรมตฺตก} {ติ} {[โอร + มตฺตา + ก]} {เป็นเพียงคุณธรรมขั้นต้นเท่านั้น} {โอรมตฺตกํ}
\entry{โอฬาริก} {ติ} {[อุฬาร + ณิก]} {ของหยาบ} {โอฬาริกํ, โอฬาริกา}
\end{multicols}
\newpage
\needspace{5cm} \section*{\Huge ก}
\noindent\rule{\textwidth}{0.4pt}
\begin{multicols}{2}
\entry{กึ} {ปุ} {ไม่มีการประกอบคำ} {อะไร} {กึ}
\entry{กจฺฉปลกฺขณ} {นปุ} {[กจฺฉป + ลกฺขณ]} {ศาสตร์ว่าด้วยการทำนายลักษณะของเต่า} {กจฺฉปลกฺขณํ}
\entry{กฏฺฏิสฺส} {นปุ} {[โกเสยฺย + กฏฺฏิสฺส]\, กฏฺฏิสฺส  (น) กฏฺฏิสฺส+ ณ]} {เครื่องลาดที่ทอด้วยไหมเจือด้วยใยทองและปักประดับด้วยรัตนะไปรอบๆ} {กฏฺฐิสฺสํ}
\entry{กณโหม} {นปุ} {[กณ + โหม]} {ศาสตร์ว่าด้วยการทำนายตามลักษณะของรำข้าวที่ใช้ในการบูชาไฟ} {กณโหมํ}
\entry{กณฺณชปฺปน} {นปุ} {[กณฺณ + ชปฺปน]} {การร่ายมนต์ทำให้หูอื้อหรือการทำให้เทวดากระซิบบอกทางหู} {กณฺณชปฺปนํ}
\entry{กณฺณเตล} {นปุ} {[กณฺณ + เตล]} {การเคี่ยวน้ำมันหยอดหู} {กณฺณเตลํ}
\entry{กณฺณสุข} {ติ} {[กณฺณ + สุข]} {ก่อให้เกิดความสบายหู} {กณฺณสุขา}
\entry{กณฺณิกาลกฺขณ} {นปุ} {[กณฺณิกา + ลกฺขณ]} {ศาสตร์ว่าด้วยการทำนายลักษณะของตุ้มหูหรือช่อฟ้าหรือศาสตร์ว่าด้วยการทำนายลักษณะของกวาง เก้ง แรด} {กณฺณิกาลกฺขณํ}
\entry{กตปรปฺปวาท} {ติ} {[กตปรปฺปวาท + กตปรปฺปวาท]} {ผู้เชี่ยวชาญในลัทธิฝ่ายอื่น} {กตปรปฺปวาทา}
\entry{กตม} {ติ} {ไม่มีการประกอบคำ} {มีอะไรบ้าง} {กตมํ}
\entry{กตฺตา} {ปุ} {[กตฺตุ + สิ]} {ผู้สร้าง} {กตฺตา}
\entry{กถา} {อิตฺ} {[กถ + อ + อา]\, กถา (อี) [กถํ + กถา]} {ไม่มีคำแปล} {กถาย}
\entry{กทลิมิคปวรปจฺจตฺถรณ} {ติ} {[กทลีมิคปวรปจฺจตฺถรณ + อตฺถต]} {เครื่องลาดที่ทำด้วยหนังชะมดอันประณีต} {กทลิมิคปวรปจฺจตฺถรณํ}
\entry{กทาจิ} {พฺยย} {ไม่มีการประกอบคำ} {ในบางครั้งบางคราว} {กทาจิ}
\entry{กปฺเปนฺติ} {ขฺยา} {[กปฺป=เณ + อนฺติ]} {ย่อมเลี้ยง} {กปฺเปนฺติ}
\entry{กพฬีการาหารภกฺข} {ติ} {[กพฬีการาหาร + ภกฺข]} {การบริโภคกพฬีการาหาร} {กพฬีการาหารภกฺโข}
\entry{กมฺม} {นปุ} {[กมุ + รมฺม]\, กมฺม (น) [กร + รมฺม]} {กรรม} {กมฺมานํ}
\entry{กยวิกฺกย} {ปุ} {[กย + วิกฺกย]} {การซื้อและการขาย} {กยวิกฺกยา}
\entry{กรณีย} {ติ} {[กร + อนีย]} {พึงกระทำ} {กรณียํ, กรณียา}
\entry{กรหจิ} {พฺยย} {ไม่มีการประกอบคำ} {ในบางครั้งบางคราว} {กรหจิ}
\entry{กาม} {ปุ\, นปุ} {[กาม + ภว]\, กาม (ปุ\,น) [กมุ + ณ]} {กามคุณ} {กามา, กาเมหิ}
\entry{กามคุณ} {ปุ} {[กาม + คุณ]} {กามคุณ} {กามคุเณหิ}
\entry{กามาวจร} {ติ} {[กาม + อวจร]\, กามาวจร (ติ) [กามาวจร + ณ]\, กามาวจร (ติ) [กามาวจร + อวจร]} {กามาวจร} {กามาวจโร}
\entry{กาย} {ปุ} {[ก + อย + ณ]} {กาย} {กายํ, กายสฺส, กายา, กาเยน, กาโย}
\entry{กาล} {ปุ} {[กล + ณ]\, กาล (ปุ) [กาล + กาล]} {กาลเวลา} {กาเลน}
\entry{กาลวาที} {ติ} {[กาล + วท + ณี]} {ผู้มีปกติกล่าวในกาลอันสมควร} {กาลวาที}
\entry{กาเวยฺย} {อิตฺ\, นปุ} {[กุ + ณฺย]} {วิชาการแต่งกาพย์กลอน} {กาเวยฺยํ}
\entry{กิํ} {ติ} {ไม่มีการประกอบคำ} {อะไร} {กา, กาย, กิํ}
\entry{กิญฺจิ} {พฺยย} {[กึ + จิ]} {นิดเดียว} {กิญฺจิ}
\entry{กิลนฺตกาย} {ติ} {[กิลนฺต + กาย]} {ผู้เหนื่อยกาย} {กิลนฺตกายา}
\entry{กิลนฺตจิตฺต} {ติ} {[กิลนฺต + จิตฺต]} {ผู้เหนื่อยใจ} {กิลนฺตจิตฺตา}
\entry{กุกฺกุฏยุทฺธ} {นปุ} {[กุกฺกุฏ + ยุทฺธ]} {การแข่งขันชนไก่} {กุกฺกุฏยุทฺธํ}
\entry{กุกฺกุฏลกฺขณ} {นปุ} {[กุกฺกุฏ + ลกฺขณ]} {ศาสตร์ว่าด้วยการทำนายลักษณะของไก่} {กุกฺกุฏลกฺขณํ}
\entry{กุกฺกุฏสูกรปฏิคฺคหณ} {นปุ} {[กุกฺกุฏสูกร + ปฏิคฺคหณ]} {การรับไก่และสุกร} {กุกฺกุฏสูกรปฏิคฺคหณา}
\entry{กุตฺตก} {นปุ} {[กุตฺต + ก]} {เครื่องลาดขนแกะขนาดใหญ่ที่สามารถรองรับนักฟ้อน ๑๖ คนได้} {กุตฺตกํ}
\entry{กุปิต} {ติ} {[กุป + ต]} {ผู้โกรธ} {กุปิตา}
\entry{กุมาร} {ปุ} {[กุมาร + อ]} {พระราชกุมาร} {กุมารานํ}
\entry{กุมารลกฺขณ} {นปุ} {[กุมาร + ลกฺขณ]} {ศาสตร์ว่าด้วยการทำนายลักษณะของเด็กชายหรือชายหนุ่ม} {กุมารลกฺขณํ}
\entry{กุมาริกปญฺห} {ติ} {[กุมาริกา + ปญฺห]} {การใช้หญิงสาวเป็นคนทรง[การเรียกเทวดามาถามปัญหาโดยผ่านร่างหญิงสาว]} {กุมาริปญฺหํ}
\entry{กุมาริลกฺขณ} {นปุ} {[กุมารี + ลกฺขณ]} {ศาสตร์ว่าด้วยการทำนายลักษณะของเด็กหญิงหรือหญิงสาว} {กุมาริลกฺขณํ}
\entry{กุมฺภฏฺฐานกถา} {อิตฺ} {[กุมฺภ + ฐาน + กถา]} {คำพูดที่ไร้สาระเกี่ยวกับแหล่งน้ำ ท่าน้ำ หรือนางกุมภทาสี} {กุมฺภฏฺฐานกถํ}
\entry{กุมฺภถูณ} {ปุ\, นปุ} {[กุมฺภ + ถูณ]} {การตีกลองสี่เหลี่ยม} {กุมฺภถูณํ}
\entry{กุสล} {ติ} {[กุสล + ณ]\, กุสล (น) [กุ + สล + อ]} {กุศล} {กุสลํ}
\entry{กุหก} {ติ} {[กุห + ณฺวุ]\, กุหก (ปุ) [กุห + ก]} {คนโกหกหลอกลวง} {กุหกา}
\entry{กุหนลปนา} {อิตฺ} {[กุหน + ลปนา]} {การพูดโกหกหลอกลวงและการพูดเอาดีใส่ตัว} {กุหนลปนา}
\entry{กูฏฏฺฐ} {ติ} {[กูฏ + ฐา + อ]\, กูฏฏฺฐ (ปุ) [กูฏ + อตฺถ]} {ธรรมชาติที่ตั้งอยู่เหนือกาลเวลา} {กูฏฏฺโฐ}
\entry{เกจิ} {พฺยย} {[เก (=กึ + โย) + จิ]} {ใดๆ} {เกจิ}
\entry{เกวฏฺฏ} {ปุ} {[กีวฏฺฏ + ณ]} {ชาวประมง} {เกวฏโฏ}
\entry{เกวฏฺฏนฺเตวาสี} {ปุ} {[เกวฏฺฏ + อนฺเตวาสี]} {ลูกน้องชาวประมง} {เกวฏฺฏนฺเตวาสี}
\entry{โกนาม} {ปุ} {[กึ + นาม]} {มีนามว่าอะไร} {โกนาโม}
\entry{โกเสยฺย} {นปุ} {[โกส + เณยฺย]} {เครื่องลาดที่ทอด้วยไหมล้วนและปักด้วยรัตนะไปรอบๆ} {โกเสยฺยํ}
\end{multicols}
\newpage
\needspace{5cm} \section*{\Huge ข}
\noindent\rule{\textwidth}{0.4pt}
\begin{multicols}{2}
\entry{ขคฺค} {ปุ\, นปุ} {[ขคฺค + อ]\, ขคฺค (ปุ\,น) [ขคฺค + ณ]} {ไม่มีคำแปล} {ขคฺคํ}
\entry{ขตฺตวิชฺชา} {อิตฺ} {[ขตฺต + วิชฺชา]} {ศาสตร์ว่าด้วยการปกครอง} {ขตฺตวิชฺชา}
\entry{ขตฺติย} {ปุ} {[ขตฺต + อิย]} {พระราชาผู้ได้รับการอภิเษก} {ขตฺติยานํ}
\entry{ขนฺธพีช} {นปุ} {[ขนฺธ + พีช]} {พันธุ์ไม้จำพวกลำต้น} {ขนฺธพีชํ}
\entry{ขลิกา} {อิตฺ} {[ขล + อิก + อา]} {การพนันทอดลูกบาศก์[ลูกเต๋า]บนกระดานสกา} {ขลิกํ}
\entry{ขิฑฺฑาปโทสิก} {ติ} {[ขิฑฺฑาปโทสิก + มูล]\, ขิฑฺฑาปโทสิก (ติ) [ขิฑฺฺฑา + ป + ทุส + ณี + ก]} {เทวดาจำพวกขิฑฑาปโทสิกะ} {ขิฑฺฑาปโทสิกา}
\entry{เขตฺตวตฺถุปฏิคฺคหณ} {นปุ} {[เขตฺตวตฺถุ + ปฏิคฺคหณ]} {การรับไร่นาและเรือกสวน} {เขตฺตวตฺถุปฏิคฺคหณา}
\entry{เขม} {ติ} {[เขม + ณ]\, เขม (ปุ\,น) [ขี + ม]} {ความสงบสุข} {เขมํ}
\entry{โข} {พฺยย} {[โขสทฺทากาะ นิปาต]} {เล็กน้อย} {โข}
\end{multicols}
\newpage
\needspace{5cm} \section*{\Huge ค}
\noindent\rule{\textwidth}{0.4pt}
\begin{multicols}{2}
\entry{คจฺฉ} {ปุ\, นปุ} {[คมุ + ณ]\, คจฺฉ (กฺริ) [คมุ + อ+ หิ]} {จงไป} {คจฺฉ}
\entry{คณน} {อิตฺ\, นปุ} {[คณ + ยุ]} {วิชาคำนวณนับด้วยการใช้ระบบตัวเลข[คณิตศาสตร์]} {คณนา}
\entry{คนฺธกถา} {อิตฺ} {[คนฺธ + กถา]} {คำพูดที่ไร้สาระเกี่ยวกับของหอม} {คนฺธกถํ}
\entry{คนฺธสนฺนิธิ} {ปุ} {[คนฺธ + สนฺนิธิ]} {การเก็บสะสมของหอม[เครื่องประทินผิว]} {คนฺธสนฺนิธิํ}
\entry{คมฺภีร} {ติ} {[คม + กีร]} {ธรรมชาติที่ลึกซึ้ง} {คมฺภีรา}
\entry{คหปติก} {ปุ} {[คหปติ + ก]} {เศรษฐี} {คหปติกานํ}
\entry{คามกถา} {อิตฺ} {[คาม + กถา]} {คำพูดที่ไร้สาระเกี่ยวกับหมู่บ้าน} {คามกถํ}
\entry{คามธมฺม} {ปุ} {[คาม + ธมฺม]} {ธรรมเนียมของชาวบ้าน} {คามธมฺมา}
\entry{คีต} {นปุ} {[เค + ต]} {การขับร้อง} {คีตํ}
\entry{โคตม} {ปุ} {[รุฬฺหีนามปุท]\, โคตม (ปุ) [โคตม + ณ]} {ผู้มีนามว่าโคดม} {โคตโม}
\entry{โคธาลกฺขณ} {นปุ} {[โคธ + ลกฺขณ]} {ศาสตร์ว่าด้วยการทำนายลักษณะของเหี้ย} {โคธาลกฺขณํ}
\entry{โคลกฺขณ} {นปุ} {[โค + ลกฺขณ]} {ศาสตร์ว่าด้วยการทำนายลักษณะของโคทั่วไป} {โคลกฺขณํ}
\end{multicols}
\newpage
\needspace{5cm} \section*{\Huge ฆ}
\noindent\rule{\textwidth}{0.4pt}
\begin{multicols}{2}
\entry{ฆฏิกา} {อิตฺ} {[ฆฏ + อิก + อา]\, ฆฏิกา (ถี) [ฆฏี + ก + อา]\, ฆฏิกา (ถี) [ฆฏ = ณฺวุ + อา]} {การพนันตีหึ่ง} {ฆฏิกํ}
\entry{ฆาน} {นปุ} {[ฆา + ยุ]} {ฆานะ} {ฆานํ}
\end{multicols}
\needspace{5cm} \section*{\Huge จ}
\noindent\rule{\textwidth}{0.4pt}
\begin{multicols}{2}
\entry{จ} {พฺยย} {[จิ + อ]\, จ  (พฺย) [จิ + อ]\, จ จ อกฺขรา.} {ด้วย} {จ}
\entry{จกฺขุ} {นปุ} {[จกฺขุ + อ]\, จกฺขุ (น) [จกฺขุ + ภูต]\, จกฺขุ (น) [จกฺขุ + ภาว ]\, จกฺขุ (น) [จกฺข + อุ]} {จักษุ} {จกฺขุํ}
\entry{จณฺฑาล} {ปุ\, นปุ} {[จณฺฑาล + ณ]\, จณฺฑาล (น) [จณฺฑ + อล]\, จณฺฑาล (ปุ) [จฑิ + อล]} {การเล่นโยนลูกเหล็กหรือการเล่นล้างศาลาของคนจัณฑาล} {จณฺฑาลํ}
\entry{จตุ} {ปุ} {ไม่มีการประกอบคำ} {๔} {จตสฺโส, จตูหิ, จตฺตาริ}
\entry{จตุจตฺตารีสา} {อิตฺ} {[จตุ + จตฺตารีสา]} {๔๔} {จตุจตฺตารีสาย}
\entry{จตุตฺถ} {ติ} {[จตุ + ถ]\, จตุตฺถ (ติ) [จตุ + อ]} {ที่ ๔} {จตุตฺถํ, จตุตฺเถ}
\entry{จตฺตาลีส} {อิตฺ} {[จตฺตาลีสา + อ]} {๔๐} {จตฺตาลีสํ}
\entry{จนฺทคฺคาห} {ปุ\, นปุ} {[จนฺทคฺคาห + ณ]\, จนฺทคฺคาห (ปุ) [จนฺท + คาห]} {จันทรคราส} {จนฺทคฺคาโห}
\entry{จนฺทิมสูริย} {ปุ} {[จนฺทิมนฺตุ + สูริย]\,  จนฺมิสูริย (ปุ) [จนฺทิมสูริย + ณ]} {ดวงจันทร์และดวงอาทิตย์ ท.} {จนฺทิมสูริยานํ}
\entry{จนฺทิมสูริยนกฺขตฺต} {นปุ} {[จนฺทิมสูริย + นกฺขตฺต]} {ดวงจันทร์ดวงอาทิตย์และดวงดาว ท.} {จนฺทิมสูริยนกฺขตฺตานํ}
\entry{จร} {ปุ} {[จร + อ]\, จร (กฺวิ) [จร + หิ]} {จงเที่ยวไป} {จร}
\entry{จรนฺติ} {ขฺยา} {[จร + อ + อนฺติ]} {ย่อมเที่ยวไป} {จรนฺติ}
\entry{จวนธมฺม} {ติ} {[อจวน + ธมฺม]} {ผู้มีการจุติเป็นธรรมดา} {จวนธมฺมา}
\entry{จวนฺต} {ติ} {[จุ + อนฺต]} {เที่ยวไปอยู่} {จวนฺติ}
\entry{จวิตฺวา} {พฺยย} {[จุ + ตฺวา.]} {จุติแล้ว} {จวิตฺวา}
\entry{จาตุมหาภูติก} {ติ} {[จตุมหาภูต + อิก]} {สำเร็จมาจากมหาภูตรูป ๔} {จาตุมหาภูติโก}
\entry{จิงฺคุลิก} {นปุ} {[จิงฺคุล + อิก]} {การพนันกังหัน} {จิงฺคุลิกํ}
\entry{จิตฺตก} {ปุ\, นปุ} {[จิติ + ณฺวุ]\, จิตฺตก (จิตฺต + ก)} {เครื่องลาดขนแกะที่เย็บเป็นลวดลายวิจิตร} {จิตฺตกํ}
\entry{จิตฺตา} {อิตฺ} {[จิตฺต + ณ + อา]\, จิตฺตา (ถี) [จิตฺต + อา]} {จิต ท.} {จิตฺตํ, จิตฺตานิ, จิตฺเต}
\entry{จิตฺรุปาหน} {ติ} {[จิตฺร + อุปาหน]} {เขียงเท้าอันประดับให้วิจิตรด้วยวัสดุมีค่า} {จิตฺรุปาหนํ}
\entry{จิร} {ติ} {[จิ + รก]} {นาน} {จิรํ}
\entry{จุต} {ติ} {[จุ + ต + ]} {จุติแล้ว} {จุตา}
\entry{จุติ} {อิตฺ} {[จุ + เณ + ติ]\, จุติ (ถี) [จุ + ติ]} {การจุติ} {จุโต}
\entry{จูฬสีล} {นปุ} {[จูฬ + สีล]} {จูฬศีล} {จูฬสีล, จูฬสีลํ}
\entry{เจ} {พฺยย} {[จิ + เอ]} {หากว่า} {เจ}
\entry{เจต} {ติ} {[เจต + ณ]\,  เจต (ปุ) [จิต + ณ]} {ใจ} {เจตํ}
\entry{เจตส} {ปุ} {[เจต + ณ]} {ใจ} {เจตโส}
\entry{เจโตสมาธิ} {ปุ} {[เจต + สมาธิ]} {เจโตสมาธิ} {เจโตสมาธิํ}
\entry{โจรกถา} {อิตฺ} {[โจร + กถา]} {คำพูดที่ไร้สาระเกี่ยวกับโจร} {โจรกถํ}
\end{multicols}
\newpage
\needspace{5cm} \section*{\Huge ฉ}
\noindent\rule{\textwidth}{0.4pt}
\begin{multicols}{2}
\entry{ฉ} {ติ} {[โฉ + อ]} {๖} {ฉนฺนํ, ฉหิ}
\entry{ฉตฺต} {นปุ} {[ฉตฺต + อุปาหน + สิกฺขาปท]\, ฉตฺต (น) [ฉท + ต]} {=ฉตฺตธารณํ การใช้ร่มอันเย็บด้วยผ้าเบญจพรรณ[ผ้าหลากหลายสี] และมีลวดลายวิจิตรด้วยรูปต่างๆ มีฟันมังกร เป็นต้น} {ฉตฺตํ}
\entry{ฉนฺท} {ปุ\, นปุ} {[ฉท + อ]\, ฉนฺท [ปุ] [ฉนฺท + อ]} {ฉันทะ} {ฉนฺโท}
\entry{เฉทนวธพนฺธนวิปราโมสอาโลปสหสาการ} {ปุ} {[เฉทน +  วธ + พนฺธน + วิปราโมส + อาโลป + สาหส + อาการ]} {การตัดแขนขา การฆ่า การจองจำ การลอบชิงทรัพย์ การปล้นสดมภ์ และการข่มขู่กรรโชก} {เฉทนวธพนฺธนวิปราโมสอาโลปสหสาการา}
\end{multicols}
\needspace{5cm} \section*{\Huge ช}
\noindent\rule{\textwidth}{0.4pt}
\begin{multicols}{2}
\entry{ชนปทกถา} {อิตฺ} {[ชนปท + กถา]} {คำพูดที่ไร้สาระเกี่ยวกับรัฐหรือประเทศ} {ชนปทกถํ}
\entry{ชย} {ปุ\, นปุ} {[ชน + อ]} {ชัยชนะ} {ชโย}
\entry{ชรามรณ} {นปุ} {[ชรา + มรณ]} {ชราและมรณะ} {ชรามรณํ}
\entry{ชาตรูปรชตปฺปฏิคฺคหณ} {นปุ} {[ชาตรูปรชต + ปฏิคฺคหณ]} {การรับทองคำและเงิน} {ชาตรูปรชตปฏิคฺคหณา}
\entry{ชาติ} {อิตฺ} {[ชน + ติ]} {ชาติ[ขันธ์ ๕ พร้อมกับความเปลี่ยนแปลง]} {ชาติ, ชาติํ, ชาติโย}
\entry{ชาติปจฺจย} {ปุ} {[ชาติ + ปจฺจย]} {ชาติ เป็นอุปนิสสยปัจจัย} {ชาติปจฺจยา}
\entry{ชาติสต} {นปุ} {[ชาติ + สต]} {ชาติ ๑๐๐ ชาติ} {ชาติสตํ, ชาติสตานิ}
\entry{ชาติสตสหสฺส} {นปุ} {[ชาติสห + สหสฺส]} {ชาติ ๑๐๐\,๐๐๐ ชาติ} {ชาติสตสหสฺสํ, ชาติสตสหสฺสานิ}
\entry{ชาติสหสฺส} {นปุ} {[ชาติ + สหสฺส]} {ชาติ ๑\,๐๐๐ ชาติ} {ชาติสหสฺสํ, ชาติสหสฺสานิ}
\entry{ชานามิ} {ขฺยา} {[ญา + นา + มิ]} {จึงรู้} {ชานามิ}
\entry{ชานาสิ} {ขฺยา} {[ญา + นา + สิ]} {ย่อมรู้} {ชานาสิ}
\entry{ชาล} {ปุ\, นปุ} {[ชล + ณ]\, ชาล (ปุ\,น) [ชล + ณ]} {แห} {ชาเลน}
\entry{ชิวฺหานิพนฺธน} {นปุ} {[ชิวฺหา + นิพนฺธน]} {การร่ายมนต์ทำให้ลิ้นแข็ง} {ชิวฺหานิพนฺธนํ}
\entry{ชีวิก} {ติ} {[ชีว + อก]\, ชีวิก (ติ) [ชีวิกา + ณ]} {ไม่มีคำแปล} {ชีวิกํ}
\entry{ชีวิต} {นปุ} {[ชีว + ต]} {ชีวิต} {ชีวิตํ}
\entry{ชีวิตปริยาทาน} {นปุ} {[ชีวิต + ปริยาทาน]} {ชีวิตดับลงแล้วโดยสิ้นเชิง ไม่มีปฏิสนธิใหม่อีกต่อไป} {ชีวิตปริยาทานา}
\entry{ชุหน} {นปุ} {[หุ + ยุ]} {การบูชาไฟ} {ชุหนํ}
\entry{ชูตปฺปมาทฏฺฐานานุโยค} {ปุ} {[ชูตปฺปมทฏฺฐาน + อนุโยค]} {การเล่นการพนันอันเป็นเหตุนำมาซึ่งความประมาท ซึ่งการบรรพชาอันเป็นเพศที่ไม่มีการครองเรือน ด้วยเดรัจฉานวิชานี้} {ชูตปฺปมาทฏฺฐานานุโยคํ, ชูตปฺปมาทฏฺฐานานุโยคา}
\end{multicols}
\newpage
\needspace{5cm} \section*{\Huge ฌ}
\noindent\rule{\textwidth}{0.4pt}
\begin{multicols}{2}
\entry{ฌาน} {นปุ} {[เฌ + ยุ]} {ฌาน} {ฌานํ}
\end{multicols}
\needspace{5cm} \section*{\Huge ญ}
\noindent\rule{\textwidth}{0.4pt}
\begin{multicols}{2}
\entry{ญสฺสติ} {ขฺยา} {[ญา + สฺสติ]} {จักรู้} {ญสฺสติ}
\entry{ญาติกถา} {อิตฺ} {[ญาติ + กถา]} {คำพูดที่ไร้สาระเกี่ยวกับญาติพี่น้อง} {ญาติกถํ}
\end{multicols}
\needspace{5cm} \section*{\Huge ฐ}
\noindent\rule{\textwidth}{0.4pt}
\begin{multicols}{2}
\entry{ฐสฺสติ} {ขฺยา} {[ฐา + อ + สฺสติ]} {จักดำรงอยู่} {ฐสฺสติ}
\entry{ฐสฺสนฺติ} {ขฺยา} {[ฐา + อ + สฺสนฺติ]} {จักดำรงอยู่} {ฐสฺสนฺติ}
\entry{ฐาน} {นปุ} {[ฐา + ยุ]} {ที่ตั้ง} {ฐานํ}
\end{multicols}
\newpage
\needspace{5cm} \section*{\Huge ต}
\noindent\rule{\textwidth}{0.4pt}
\begin{multicols}{2}
\entry{ต} {ติ} {ไม่มีการประกอบคำ} {นั้น} {ตํ, ตมฺหา, ตสฺส, เตน, เตสํ, นํ, สา, โส}
\entry{ตกฺกปริยาหต} {นปุ} {[ตกฺก (๑) + ปริยาหต (=ปริ + อา + \`หร\` + ต)]} {ตามอารมณ์ความนึกคิด} {ตกฺกปริยาหตํ}
\entry{ตกฺกี} {ติ} {[ตกฺก + ณี]} {เป็นผู้มีปกติคิด[นักทฤษฎี]} {ตกฺกี}
\entry{ตจฺฉ} {นปุ} {[ตถ + ย]} {คำพูดที่ถูกต้อง} {ตจฺฉํ}
\entry{ตณฺฑุลโหม} {ปุ\, นปุ} {[ตณฺฑุล + โหม]} {ศาสตร์ว่าด้วยการทำนายตามลักษณะของข้าวสารที่ใช้ในการบูชาไฟ} {ตณฺฑุลโหมํ}
\entry{ตณฺหา} {อิตฺ} {[ตส + ณฺหา]\, ตณฺหา (ถี) [ตส + ณฺห + อา]} {ตัณหา[ความอยาก]} {ตณฺหา}
\entry{ตณฺหาคต} {นปุ} {[ตณฺหา + คต]} {=ตณฺหาทิฏฺิคตานํ ผู้เต็มไปด้วยตัณหาและทิฏฐิ} {ตณฺหาคตานํ}
\entry{ตณฺหาปจฺจย} {ปุ} {[ตณฺหา + ปจฺจย]} {มีตัณหาเป็นอุปนิสสยปัจจัยและสหชาตปัจจัย} {ตณฺหาปจฺจยา}
\entry{ตติย} {ติ} {[ติ + ติย]} {ลำดับที่ ๓} {ตติยํ, ตติเย}
\entry{ตโต} {พฺยย} {[ต + โต]} {จากนั้น} {ตโต}
\entry{ตตฺถ} {พฺยย} {[นิปาตมตฺต]} {นั้น} {ตตฺถ}
\entry{ตตฺร} {พฺยย} {[วากฺยารมฺภ-อุปนฺยาส-จ-ปน-นฺหง့ อนกตู นิปาตปุท.]} {ใน... ท.เหล่านั้น} {ตตฺร}
\entry{ตถา} {ติ} {[ตถ + ณ]} {ฉันนั้น} {ตถา}
\entry{ตถาคต} {ปุ} {[ตถาคต + คุณ]\, ตถาคต (ติ) [ตถา + อาคต]} {ตถาคต} {ตถาคตสฺส, ตถาคโต}
\entry{ตถารูป} {ติ} {[ตถา + รูป]} {อันมีความตั้งมั่นแห่งจิตอย่างแน่วแน่ เช่นนั้นเป็นสภาพ} {ตถารูปํ, ตถารูปิํ}
\entry{ตทนฺวย} {ติ} {[ต + อนฺวย]} {ย่อมเป็นผลที่ติดสอยห้อยตามขั้วนั้นไป} {ตทนฺวยานิ}
\entry{ตึส} {อิตฺ} {[ติ + ทส ]} {๓๐} {ตึสํ}
\entry{ตสฺมาติห} {พฺยย} {ไม่มีการประกอบคำ} {เพราะเหตุนั้นแล} {ตสฺมาติห}
\entry{ตา} {อิตฺ} {ไม่มีการประกอบคำ} {นั้น} {ตานิ}
\entry{ตาว} {พฺยย} {[ปฐม\, กม-จโส อนกรฺหิ นิปาตปุท]} {สิ้นกาลเพียงนั้น} {ตาว}
\entry{ติ} {นปุ} {ไม่มีการประกอบคำ} {๓} {ติสฺโส, ตีณิ}
\entry{ติฏฺฐติ} {ขฺยา} {[ฐา + อ + ติ]} {ย่อมตั้งอยู่} {ติฏฺฐติ}
\entry{ติฏฺฐนฺติ} {ขฺยา} {[ฐา + อ + อนฺติ]} {ย่อมดำรงอยู่} {ติฏฺฐนฺติ}
\entry{ติรจฺฉานกถา} {อิตฺ} {[ติรจฺฉานา + กถา]} {คำพูดอันเป็นเดรัจฉานกถา} {ติรจฺฉานกถํ, ติรจฺฉานกถาย}
\entry{ติรจฺฉานวิชฺชา} {อิตฺ} {[ติรจฺฉานา + วิชฺชา]} {เดรัจฉานวิชา} {ติรจฺฉานวิชฺชาย}
\entry{ติริยํ} {พฺยย} {[สตฺตมฺยนฺต-นิปาต]\, ติริยํ (กฺวิ\,วิ) [ตร + ตฺวา]} {ในแนวราบ} {ติริยํ}
\entry{ตุมฺห} {อ} {[สพฺพนาม ๒๗ (รู]} {เธอ} {ตยา, ตุมฺหํ, ตุเมฺห, ตุเมฺหหิ, เต, ตฺวํ}
\entry{ตุลากูฏกํสกูฏมานกูฏา} {อิตฺ\, นปุ} {[ตุลากูฏ + กํสกูฏ + มานกูฏ]} {การโกงด้วยตาชั่ง การโกงด้วยถาดทองคำและการโกงด้วยการตวงวัด} {ตุลากูฏกํสกูฏมานกูฏา}
\entry{ตูลิกา} {อิตฺ} {[ตูล + ณิก + อา]} {เครื่องลาดอันยัดนุ่น} {ตูลิกํ}
\entry{เตลโหม} {ปุ\, นปุ} {[เตล + โหม]} {ศาสตร์ว่าด้วยวิธีการเติมน้ำมันบูชาไฟ} {เตลโหมํ}
\end{multicols}
\newpage
\needspace{5cm} \section*{\Huge ถ}
\noindent\rule{\textwidth}{0.4pt}
\begin{multicols}{2}
\entry{ถุสโหม} {ปุ\, นปุ} {[ถุส + โหม]} {ศาสตร์ว่าด้วยการทำนายตามลักษณะของแกลบที่ใช้ในการบูชาไฟ} {ถุสโหมํ}
\entry{เถต} {ติ} {[เถต + ณ]} {ผู้มีถ้อยคำที่หนักแน่น} {เถโต}
\end{multicols}
\needspace{5cm} \section*{\Huge ท}
\noindent\rule{\textwidth}{0.4pt}
\begin{multicols}{2}
\entry{ทกฺข} {ติ} {[ทล + ข]} {ผู้ชำนาญการ} {ทกฺโข}
\entry{ทกฺขนฺติ} {ขฺยา} {[ทิส + อ + อนฺติ]\, ทกฺขนฺติ (กฺริ) [ทิส + อ + สฺสนฺติ]} {จักได้เห็น} {ทกฺขนฺติ}
\entry{ทณฺฑ} {ติ} {[ทฑิ=ทณฺฑ + อ]\, ทณฺฑ (ติ) [ทณฺฑ + อี]\, ทณฺฑ (น) [ทณฺฑ + ทาน]} {การใช้ไม้เท้าที่ประดับตกแต่งแล้ว} {ทณฺฑํ}
\entry{ทณฺฑยุทฺธ} {นปุ} {[ทณฺฑ + ยุทฺธ]} {การแข่งขันตีกระบอง} {ทณฺฑยุทฺธํ}
\entry{ทณฺฑลกฺขณ} {นปุ} {[ทณฺฑ + ลกฺขณ]} {ศาสตร์ว่าด้วยการทำนายลักษณะของไม้เท้า} {ทณฺฑลกฺขณํ}
\entry{ทพฺพิโหม} {ปุ\, นปุ} {[ทพฺพิ + โหม]} {ศาสตร์ว่าด้วยการทำนายตามลักษณะของทัพพีที่ใช้ในการบูชาไฟ} {ทพฺพิโหมํ}
\entry{ทยาปนฺน} {ติ} {[ทยา + อาปนฺน]} {ผู้ถึงพร้อมด้วยเมตตาจิตต่อเหล่าสัตว์ทั้งปวง} {ทยาปนฺโน}
\entry{ทส} {ติ} {[สพฺพนาม มหุตโส พหุวุจ สงฺขฺยาสทฺทา]\, ทส (ติ) [ทส + สหสฺส]\, ทส (ติ) [ทส + ทส]\, ทส (ติ) [ทิส + อ]\, ทส (ติ) [ทส + อ]} {๑๐} {ทส}
\entry{ทสปท} {นปุ} {[ทส + ปท]} {การพนันหมากรุกที่มีแถวละ ๑๐ ตา} {ทสปทํ}
\entry{ทสสหสฺสี} {อิตฺ} {[ทสสหสฺส + อี]} {จำนวนหนึ่งหมื่นจักรวาล} {ทสสหสฺสี}
\entry{ทารกติกิจฺฉา} {อิตฺ} {[ทารก + ติกิจฺฉา]} {การรักษาเด็ก} {ทารกติกิจฺฉา}
\entry{ทาสลกฺขณ} {นปุ} {[ทาส + ลกฺขณ]} {ศาสตร์ว่าด้วยการทำนายลักษณะของทาสชาย} {ทาสลกฺขณํ}
\entry{ทาสิทาสปฏิคฺคหณ} {นปุ} {[ทาสิทาส + ปฏิคฺคหณ]} {การรับทาสหญิงและทาสชาย} {ทาสิทาสปฏิคฺคหณา}
\entry{ทาสิลกฺขณ} {นปุ} {[ทาสี + ลกฺขณ]} {ศาสตร์ว่าด้วยการทำนายลักษณะของทาสหญิง} {ทาสิลกฺขณํ}
\entry{ทิฏฺฐธมฺมนิพฺพานวาท} {ปุ} {[ทิฏฺฐธมฺมนิพฺพาน + วาท]} {ทิฏฐธัมมนิพพานทิฏฐิ ๕ [บุคคลผู้มีความเห็นผิดคิดว่าการดับทุกข์ในชาติปัจจุบันคือนิพพาน ๕ จำพวก]} {ทิฏฺฐธมฺมนิพฺพานวาท, ทิฏฺฐธมฺมนิพฺพานวาทา}
\entry{ทิฏฺฐิคต} {นปุ} {[ทิฏฺฐิ + คต]} {ทิฏฐิ} {ทิฏฺฐิคตานิ}
\entry{ทิฏฺฐิคติกาธิฏฺฐานวฏฺฏกถา} {อิตฺ} {[ทิฏฺฐิคติก + อธิฏฺฐาน + วฏฺฏกถา]} {ทิฏฐิคติกาธิฏฐานวัฏฏกถา ตอนว่าด้วยวัฏทุกข์อันเป็นที่ตั้งของบุคคลผู้มีมิจฉาทิฏฐิ]} {ทิฏฺฐิคติกาธิฏฺฐานวฏฺฏกถา}
\entry{ทิฏฺฐิชาล} {ปุ\, นปุ} {[ทิฏฺฐิ + ชาล]} {ทิฏฐิชาละ[ข่ายกล่าวคือทิฏฐิ]} {ทิฏฺฐิชาลํ}
\entry{ทิฏฺฐิฏฺฐาน} {นปุ} {[ทิฏฺฐิ + ฐาน]} {ไม่มีคำแปล} {ทิฏฺฐิฏฺฐานา}
\entry{ทินฺนปาฏิกงฺขี} {ติ} {[ทินฺน + ปาฏิกงฺขี]} {ผู้ต้องการเฉพาะสิ่งที่ผู้อื่นให้แล้วเท่านั้น} {ทินฺนปาฏิกงฺขี}
\entry{ทินฺนาทายี} {ติ} {[ทินฺน + อาทายี]} {ผู้มีปกติถือเอาสิ่งของที่ผู้อื่นให้แล้วเท่านั้น} {ทินฺนาทายี}
\entry{ทิพฺพ} {ติ} {ไม่มีการประกอบคำ} {เป็นทิพย์[ธรรมชาติที่เกิดในสวรรค์]} {ทิพฺโพ}
\entry{ทิสาฑาห} {ปุ} {[ทิสา + ฑาห= ทาห]} {ภาวะความปั่นป่วนร้อนระส่ำแห่งทิศ} {ทิสาฑาโห}
\entry{ทีฆ} {ติ} {ไม่มีการประกอบคำ} {ยาว} {ทีฆํ, ทีฆสฺส}
\entry{ทีฆทส} {ติ} {[ทีฆา + ทสา]} {มีชายยาว} {ทีฆทสานิ}
\entry{ทีฆรตฺต} {นปุ} {[ทีฆา + รตฺติ]} {ระยะเวลานานแสนนาน} {ทีฆรตฺตํ}
\entry{ทีฆายุกตร} {ติ} {[ทีฆ + อายุก + ตร]} {ผู้มีอายุยืน} {ทีฆายุกตโร}
\entry{ทุกฺข} {ติ} {[ทุกฺข +  ทุกฺข]\, ทุกฺข (ติ) [ทุกฺข + ณ]\, ทุกฺข (น) [ทุกฺข + อ]} {ทุกข์} {ทุกฺขสฺส, ทุกฺขา}
\entry{ทุติย} {ติ} {[ทฺวิ + ติย]\, ทุติย (ติ) [ทฺวิ  + ติย]\, ทุติย (ติ) [ทุติย + อ]} {ที่ ๒} {ทุติยํ, ทุติเย}
\entry{ทุติยภาณวาร} {ปุ} {[ทุติย + ภาณวาร]} {ภาณวารที่ ๒} {ทุติยภาณวโร, ทุติยภาณวาโร}
\entry{ทุทฺทส} {ติ} {[ทุ + ทิส + ข=อ]} {ธรรมชาติที่เห็นได้ยากนั่นเทียว} {ทุทฺทสา}
\entry{ทุพฺพณฺณตร} {ติ} {[ทุพฺพณฺณ + ตร]} {ผู้มีผิวพรรณหยาบกว่า} {ทุพฺพณฺณตรา}
\entry{ทุพฺพุฏฺฐิกา} {อิตฺ} {[ทุพฺพุฏฺฐิ + ก + อา]} {ฝนแล้ง} {ทุพฺพุฏฺฐิกา}
\entry{ทุพฺภคกรณ} {นปุ} {[ทุพฺภค + กรณ]} {การดูฤกษ์ประกอบงานที่ไม่เป็นสิริมงคลหรือการทำให้คนชัง} {ทุพฺภคกรณํ}
\entry{ทุพฺภาสิต} {ติ} {[ทุ + ภาสิต]} {ความหมายของคำที่เป็นทุพภาษิต[คำพูดที่ไม่ดี]} {ทุพฺภาสิตํ}
\entry{ทุพฺภิกฺข} {ปุ\, นปุ} {[ทุ + ภิกฺขา.]} {ข้าวยากหมากแพง} {ทุพฺภิกฺขํ}
\entry{ทุรนุโพธ} {ติ} {[ทุ + อนุโพธ]} {อันเป็นธรรมชาติที่หยั่งรู้ได้ยาก} {ทุรนุโพธา}
\entry{ทูเตยฺยปหิณคมนานุโยค} {ปุ} {[ทูเตยฺยปหิณคมน + อนุโยค]} {การทำหน้าที่เป็นทูตและหน้าที่เดินส่งข่าว} {ทูเตยฺยปหิณคมนานุโยคํ, ทูเตยฺยปหิณคมนานุโยคา}
\entry{เทว} {ปุ} {[ทิวุ + ณ + ]\, เทว (น) [เทว + ณ + อ]} {เทวดา} {เทวา}
\entry{เทวทุนฺทุภิ} {อิตฺ} {[เทว + ทุนฺทุภิ]} {การคำรามผิดกาลแห่งเมฆ} {เทวทุนฺทุภิ}
\entry{เทวปญฺห} {ติ} {[เทว + ปญฺห.]} {การใช้นางทาสเป็นคนทรง[การเรียกเทวดามาถามปัญหาโดยผ่านร่างหญิงรับใช้]} {เทวปญฺหํ}
\entry{เทวมนุสฺส} {ปุ} {[เทว + มนุสฺส]} {เทวดาและมนุษย์ ท.} {เทวมนุสฺสา}
\entry{โทส} {ปุ} {[ทุส + ณ]} {โทสะ[ความโกรธ]} {โทโส}
\entry{ทฺวาสฏฺฐิ} {ติ} {[ทฺวิ + สฏฺฐิ.]} {๖๒} {ทฺวาสฏฺฐิยา}
\entry{ทฺวิ} {ติ} {ไม่มีการประกอบคำ} {๒} {ทฺวีหิ, เทฺว}
\end{multicols}
\newpage
\needspace{5cm} \section*{\Huge ธ}
\noindent\rule{\textwidth}{0.4pt}
\begin{multicols}{2}
\entry{ธนุก} {นปุ} {[ธนุ + ก]} {การพนันของเล่นที่เป็นธนูเล็กๆ} {ธนุกํ}
\entry{ธนุลกฺขณ} {นปุ} {[ธนุ + ลกฺขณ]} {ศาสตร์ว่าด้วยการทำนายลักษณะของธนู} {ธนุลกฺขณํ}
\entry{ธมฺม} {ปุ\, นปุ} {[ธร + รมฺม]\, ธมฺม (ปุ)\, [ธมฺม + กาย]\, ธมฺม (ติ) [ธมฺม + ณ]} {พระธรรม} {ธมฺมสฺส, ธมฺเมหิ}
\entry{ธมฺมชาล} {ติ} {[ธมฺม + ชาล]} {ธัมมชาละ[ข่ายกล่าวคือธรรม]} {ธมฺมชาลํ}
\entry{ธมฺมปริยาย} {ปุ} {[ธมฺม + ปริยาย]} {พระธรรมเทศนา} {ธมฺมปริยายํ, ธมฺมปริยาโย}
\entry{ธมฺมวาที} {ติ} {[ธมฺม + วาที]} {ผู้มีปกติกล่าวเฉพาะคำที่เกี่ยวข้องกับโลกุตตรธรรม} {ธมฺมวาที}
\entry{ธมฺมวินย} {ติ} {[ธมฺมวินย + ณ]\, ธมฺมวินย (ปุ) [ธมฺม + วินย]\, ธมฺมวินย (ปุ) [ธมฺม + วินย + ปกาสิต]\, ธมฺมวินย (ุปุ) [ธมฺม + ยุตฺต + วินย]\, ธมฺมวินย (ปุ) [ธมฺม + วินย]} {พระธรรมวินัย} {ธมฺมวินยํ}
\entry{ธมฺมา} {อิตฺ} {[ยทิจฺฉานาม]} {คุณธรรม ท.} {ธมฺมา}
\entry{ธาเรหิ} {ขฺยา} {[ธร + เณ + หิ]} {จงทรงจำไว้} {ธาเรหิ}
\entry{ธุว} {นปุ} {[ธุ + อ]} {ผู้ยั่งยืน} {ธุวา, ธุโว}
\entry{โธวน} {อิตฺ\, นปุ} {[โธวุ + ยุ]} {การละเล่นในงานพิธีล้างอัฐิผู้ตาย} {โธวนํ}
\end{multicols}
\newpage
\needspace{5cm} \section*{\Huge น}
\noindent\rule{\textwidth}{0.4pt}
\begin{multicols}{2}
\entry{น} {พฺยย} {ไม่มีการประกอบคำ} {ไม่} {น}
\entry{นกฺขตฺต} {นปุ} {[น + ขี + อ]} {ดวงดาว} {นกฺขตฺตานํ}
\entry{นกฺขตฺตคฺคาห} {ปุ} {[นกฺขตฺต + คาห]} {นักษัตรคราส} {นกฺขตฺตคฺคาโห}
\entry{นครกถา} {อิตฺ} {[นคร + กถา]} {คำพูดที่ไร้สาระเกี่ยวกับนคร} {นครกถํ}
\entry{นจฺจ} {นปุ} {[นจฺจ + อ]\, นจฺจ (กฺริ) [นต (นฏ + ย + หิ)]} {การฟ้อนรำ} {นจฺจํ}
\entry{นจฺจคีตวาทิตวิสูกทสฺสน} {นปุ} {[นจฺจ + คีต + วาทิต + วิสูก + ทสฺสน]} {การดูซึ่งการฟ้อนรำ การขับร้อง และการดีดสีตีเป่าอันเป็นข้าศึกต่อคำสอน} {นจฺจคีตวาทิตวิสูกทสฺสนา}
\entry{นตฺถิ} {พฺยย} {[เอกวุจ พหุวุจ นฺหจมฺยิุะรโส นิปาตปุท]} {ไม่มี} {นตถิ, นตฺถิ}
\entry{นตฺถุกมฺม} {นปุ} {[นตฺถุ + กมฺม]} {การทำนัตถุกรรม[การปรุงยานัตถุ์]} {นตฺถุกมฺมํ}
\entry{นานตฺตกถา} {อิตฺ} {[นานตฺต (๒) + กถา]} {คำพูดที่ไร้สาระเกี่ยวกับเรื่องสัพเพเหระ} {นานตฺตกถํ}
\entry{นานตฺตสญฺญี} {ติ} {[นานตฺต + สญฺญี]} {ธรรมชาติที่มีสัญญาอันมีสภาพที่หลากหลาย(หรือธรรมชาติที่มีสัญญาอันเป็นไปในอารมณ์ที่หลากหลาย)} {นานตฺตสญฺญานํ, นานตฺตสญฺญี}
\entry{นานาธิมุตฺติกตา} {อิตฺ} {[นานาธิมุตฺติก + ตา]} {ความเป็นผู้มีอัธยาศัยต่างๆ นานา} {นานาธิมุตฺติกตา}
\entry{นาม} {ปุ} {ไม่มีการประกอบคำ} {ชื่อ} {นาม}
\entry{นาฬนฺทา} {อิตฺ} {[ยทิจฺฉานาม]} {นิคมนาฬันทา} {นาฬนฺทํ}
\entry{นาฬิก} {ปุ} {[ยทิจฺฉานาม]} {การสะพายกลักยาที่ประดับตกแต่งแล้ว} {นาฬิกํ}
\entry{นิคมกถา} {อิตฺ} {[นิคม + กถา]} {คำพูดที่ไร้สาระเกี่ยวกับนิคม} {นิคมกถํ}
\entry{นิคฺคหิต} {นปุ} {[นิ + คห + ต รสฺสสรํ นิสฺสาย คยฺหติ\, กรณํ นิคฺคเหตฺวา]} {ผู้อันข้าพเจ้าชนะแล้ว} {นิคฺคหิโต}
\entry{นิจฺจ} {นปุ} {[น + อิ + ต]\, นิจฺจ (น) [นิจฺจ + ภตฺต]} {เที่ยงแท้} {นิจฺจา}
\entry{นิจฺโจ} {ติ} {[นิจฺจ]} {ธรรมชาติที่เที่ยงแท้} {นิจฺโจ}
\entry{นิชิคีสิตุ} {ติ} {[นิ + หร + ส + ตุ]} {ผู้แสวงหา} {นิชิคิํสิตาโร}
\entry{นิฏฺฐิต} {ติ} {[นิ + ฐา + ต]} {จบแล้ว} {นิฏฺฐิตํ}
\entry{นิธานวตี} {ติ} {[นิธาน + วนฺตุ + อี]} {ตรึงใจ} {นิธานวติํ}
\entry{นิปุณ} {ติ} {[นิ + ปุณ + อ]} {ธรรมชาติที่ละเอียดอ่อน} {นิปุณา}
\entry{นิปฺเปสิก} {ติ} {[นิปฺเปสิ + ณิก]} {นักแสวงหาลาภด้วยการโยนความผิดแก่ผู้อื่น} {นิปฺเปสิกา}
\entry{นิพฺพุติ} {อิตฺ} {[นิพฺพุต + โภชน]\, นิพฺพตฺติ (ถี) [นิ + วา + ติ]} {การดับ} {นิพฺพุติ}
\entry{นิพฺพุทฺธ} {ปุ} {[นิ + ยุธ + ต]} {การแข่งขันมวยปล้ำ} {นิพฺพุทฺธํ}
\entry{นิพฺเพเฐตพฺพ} {ติ} {[นิ + เวฐ + เณ + ตพฺพ]} {พึงคัดค้าน} {นิพฺเพเฐตพฺพํ}
\entry{นิพฺเพเฐหิ} {ขฺยา} {[นิ + เวฐ + เณ + หิ]} {จงลบล้าง} {นิพฺเพเฐหิ}
\entry{นิมิตฺต} {นปุ} {[นิ + มา + ต]} {ศาสตร์ว่าด้วยการทำนายตามเครื่องหมายที่ปรากฏเป็นนิมิต} {นิมิตฺตํ}
\entry{นิมฺมาตุ} {ติ} {[นิ + มาตุ]} {ผู้เนรมิต} {นิมฺมาตา}
\entry{นิมฺมิต} {ติ} {[นิ + มา + ต]} {เนรมิตขึ้น} {นิมฺมิตา}
\entry{นิยฺยาน} {นปุ} {[นิ + ยา + ยุ]} {การเสด็จออก} {นิยฺยานํ}
\entry{นิสชฺช} {พฺยย} {[นิ + สท + ตฺวา]\, นิสชฺช (กฺริ) [นิ + สท + ย + อ]} {ประทับนั่งแล้ว} {นิสชฺช}
\entry{นิสีทิ} {ขฺยา} {[นิ + สท + อี]} {ประทับนั่งแล้ว} {นิสีทิ}
\entry{นิสฺสรณ} {นปุ} {[นิ + สร + ยุ (นิครณ-สํ\, ณิสฺสรณ-ปฺรา)]} {การออก (หรือข้อปฏิบัติอันเป็นทางออก)} {นิสฺสรณํ}
\entry{นิหิตทณฺฑ} {ติ} {[นิหิต + ทณฺฑ]} {ผู้มีท่อนไม้อันวางลงแล้ว} {นิหิตทณฺโฑ}
\entry{นิหิตสตฺถ} {ติ} {[นิหิต + สตฺถ]} {ผู้มีศาสตราอันวางลงแล้ว} {นิหิตสตฺโถ}
\entry{นุ} {พฺยย} {ไม่มีการประกอบคำ} {หรือ} {นุ}
\entry{เนตฺตตปฺปน} {นปุ} {[เนตฺต + ตปฺปน]} {การเคี่ยวน้ำมันหยอดตา} {เนตฺตตปฺปนํ}
\entry{เนมิตฺติก} {ติ} {[นิมิตฺต + ณิก]} {นักสร้างภาพ} {เนมิตฺติกา}
\entry{เนล} {ติ} {[น + เอล]} {วาจาที่ไม่มีโทษ} {เนลา}
\entry{เนวสญฺญานาสญฺญายตนุปค} {ติ} {[เนวสญฺญานาสญฺญายตน + อุป + คมุ + กกฺวิ]} {การบัญญัติว่า โลกมีที่สุดก็ไม่ใช่และไม่มีที่สุดก็ไม่ใช่} {เนวสญฺญานาสญฺญายตนูปโค}
\entry{เนวสญฺญีนาสญฺญี} {ติ} {[เนวสญฺญี + นาสญฺญี]} {ธรรมชาติ ที่มีสัญญาก็ไม่ใช่ ไม่มีสัญญาก็ไม่ใช่} {เนวสญฺญีนาสญฺญี}
\entry{นฺหาปน} {นปุ} {[นฺหา + ณาเป + ยุ]} {การรดน้ำมนต์[การให้อาบน้ำมนต์]} {นฺหาปนํ}
\end{multicols}
\newpage
\needspace{5cm} \section*{\Huge ป}
\noindent\rule{\textwidth}{0.4pt}
\begin{multicols}{2}
\entry{ป} {พฺยย} {[ป-สญฺญ อุปสารปุท ๒ว-ติ့ุตฺวง ตปุทผฺรจสญฺญ\, สูขฺยญฺญะ สีะสน့ เนริุะ มรฺหิ\, นามปุท\, อาขฺยาตปุทติ့ุ၏ อจ]} {ไม่มีคำแปล} {ป}
\entry{ปกฺกชฺฌาน} {นปุ} {[ปกฺก + ฌาน]} {ศาสตร์ว่าด้วยการทำนายอายุ[ซึ่งเป็นปัญญาที่แก่กล้าภายใน]} {ปกฺกชฺฌานํ}
\entry{ปงฺคจีร} {นปุ} {[ปงฺก + จีร]} {การพนันเป่าใบไม้} {ปงฺคจีรํ}
\entry{ปจฺจญฺชน} {นปุ} {[ปติ + อญฺชน]} {การปรุงยาป้ายตาที่มีสรรพคุณทำให้ตาเย็น} {ปจฺจญฺชนํ}
\entry{ปจฺจตฺต} {นปุ} {[ปติ + อตฺต]\, ปจฺจตฺต (ติ) [ปจฺจตฺต + ณ]} {ตนเอง} {ปจฺจตฺตํ}
\entry{ปจฺจยิก} {ติ} {[ปจฺจย + ณิก]} {ผู้มีคำพูดอันน่าเชื่อถือ} {ปจฺจยิโก}
\entry{ปจฺจุฏฺฐิต} {ติ} {[ปติ + อุ + ฐา + ต]} {ผู้ลุกขึ้นแล้ว} {ปจฺจุฏฺฐิตานํ}
\entry{ปจฺจูสสมย} {ปุ} {[ปจฺจูส + สมย]} {ในเวลาเช้า} {ปจฺจูสสมยํ}
\entry{ปจฺฉา} {พฺยย} {[\`สมนฺตา\`จสญฺญติ့ุกဲ့สิ့ุ สตฺตมฺยนฺตนิปาตผฺรจสญฺญ]} {ภายหลัง} {ปจฺฉา}
\entry{ปชานน} {อิตฺ\, นปุ} {[ป + ญา + นา + ยุ]} {=ปชานนฺโต จ แม้จะรู้อยู่} {ปชานนํ}
\entry{ปชานนฺต} {ปุ\, อิตฺ} {[ป + ญา + นา + อนฺต]} {แม้จะรู้อยู่} {ปชานนฺโต, ปนานนฺโต}
\entry{ปชานาติ} {ขฺยา} {[ป + ญา + นา + ติ]} {ย่อมรู้} {ปชานาติ}
\entry{ปชานามิ} {ขฺยา} {[ป + ญา + นา + มิ]} {ย่อมรู้} {ปชานามิ}
\entry{ปญฺจ} {ติ} {[ปจิ + อ]} {๕} {ปญฺจ, ปญฺจหิ}
\entry{ปญฺจม} {ติ} {[ปญฺจ + ม]} {ที่ ๕} {ปญฺจมํ}
\entry{ปญฺจมตฺต} {ติ} {[ปญฺจ + มตฺตา]} {ประมาณ ๕} {ปญฺจมตฺเตหิ}
\entry{ปญฺญตฺต} {ติ} {[ป + ญา + ณาเป + ต]} {ที่เขาปูลาดไว้แล้ว(หรือจัดเตรียมไว้แล้ว)} {ปญฺญตฺเต}
\entry{ปญฺญเปนฺติ} {ขฺยา} {[ป + ญา + ณาเป + อนฺติ]} {ย่อมบัญญัติ} {ปญฺญเปนฺติ}
\entry{ปญฺญาคต} {นปุ} {[ปญฺญา + คต]} {มีปัญญา} {ปญฺญาคเตน}
\entry{ปญฺญาส} {อิตฺ} {ไม่มีการประกอบคำ} {๕๐} {ปญฺญาสํ}
\entry{ปญฺห} {ติ} {[ปญฺห + มคฺค]\, ปญฺห (ติ) [ปุจฺฉ + อ]} {ปัญหา} {ปญฺหํ}
\entry{ปฏลิกา} {อิตฺ} {[ปฏล + อิก + อา]} {เครื่องลาดขนแกะอันหรูหราด้วยลายดอกไม้} {ปฏลิกํ}
\entry{ปฏิกา} {อิตฺ} {[ปฏ + ณฺวุ]} {เครื่องลาดขนแกะอันมีสีขาว} {ปฏิกํ}
\entry{ปฏิฆ} {ปุ\, นปุ} {[ปติ + หน + ร]} {ปฏิฆะ[ความกระทบกระทั่ง]} {ปฏิโฆ}
\entry{ปฏิฆสญฺญา} {อิตฺ} {[ปฏิฆ + สมุปฺปนฺนา + สญฺญา]} {ปฏิฆสัญญา} {ปฏิฆสญฺญานํ}
\entry{ปฏิชานิตพฺพ} {ติ} {[ปติ + ญา + นา + ตพฺพ]} {พึงยืนยัน} {ปฏิชานิตพฺพํ}
\entry{ปฏิโมกฺข} {ปุ} {[ปติ + โมกฺข + เณ + อ]} {การพอกยา} {ปฏิโมกฺโข}
\entry{ปฏิวรต} {ติ} {[ปติ + วิ + รมุ + ต]} {ผู้งดเว้น} {ปฏิวิรโต}
\entry{ปฏิสํเวทิสฺสนฺติ} {ขฺยา} {[ปติ + สํ + วิท + ณิ + สฺสนฺติ]} {จักเสวย} {ปฏิสํเวทิสฺสนฺติ}
\entry{ปฏิสํเวเทติ} {ขฺยา} {[ปติ + สํ + วิท + เณ + ติ]\, ปฏิสํเวเทติ (กา\,กฺริ) [ปติ + สํ + วิท + เณ + ติ]} {ย่อมเสวย} {ปฏิสํเวเทติ}
\entry{ปฏิสํเวเทนฺติ} {ขฺยา} {[ปติ + สํ + วิท + เณ + อนฺติ]} {ย่อมเสวย} {ปฏิสํเวเทนฺติ}
\entry{ปฐม} {ติ} {[ปฐ + อม]} {ที่หนึ่ง} {ปฐมํ}
\entry{ปฐมภาณวาร} {ปุ} {[ปฐม + ภาณวาร]} {ภาณวาระที่ ๑} {ปฐมภาณวาโร}
\entry{ปณิธิกมฺม} {นปุ} {[ปณิธิ + กมฺม]} {การทำพิธีแก้บนหรือการทำพิธีบวงสรวงเทพยดา} {ปณิธิกมฺมํ}
\entry{ปณีต} {นปุ} {[ป + นี + ต]} {ประณีต} {ปณีตํ, ปณีตา}
\entry{ปณฺฑิต} {ติ} {[ปณฺฑิต (๑) + ณ]\, ปณฺฑิต (ติ) [(๑) ปฑิ + ต]} {บัณฑิต} {ปณฺฑิตา}
\entry{ปณฺฑิตเวทนีย} {ติ} {[ปณฺฑิต + เวทนีย]} {ธรรมชาติที่ผู้เป็นบัณฑิตกล่าวคือพระพุทธเจ้าเท่านั้นจะพึงรู้ได้} {ปณฺฑิตเวทนียา}
\entry{ปตฺต} {ปุ\, นปุ} {[ปต=ปท + ต]\} ปตฺต (น) [ป + อ +อาป + ต]\} ปตฺต (น) [ปต + ต]} {ถึงแล้ว} {ปตฺโต}
\entry{ปตฺตาฬฺหก} {ปุ\, นปุ} {[ปตฺต + อาฬฺหก]} {การพนันตวงทราย} {ปตฺตาฬฺหกํ}
\entry{ปถวิคมน} {นปุ} {[ปถวี + คมน]} {การโคจรในเส้นทาง} {ปถคมนํ}
\entry{ปทุฏฺฐจิตฺต} {ปุ\, อิตฺ} {[ปทุฏฺฐ + จิตฺต]} {ผู้มีจิตประทุษร้าย} {ปทุฏฺฐจิตฺตา}
\entry{ปทุสฺสนฺติ} {ขฺยา} {[ป + ทุส + ย + อนฺติ]} {ไม่มีคำแปล} {ปทุสฺเสนฺติ}
\entry{ปทูสิมฺหา} {ขฺยา} {[ป + ทุส + เณ + มฺหา]} {ประทุษร้ายแล้ว} {ปโทสิมฺหา}
\entry{ปทูเสนฺติ} {ขฺยา} {[ป + ทุส + เณ + อนฺติ]} {ย่อมประทุษร้าย} {ปโทเสนฺติ}
\entry{ปธาน} {นปุ} {ไม่มีการประกอบคำ} {ความเพียรอย่างไม่ลดละ} {ปธานํ}
\entry{ปน} {พฺยย} {[อุปสารปุท\, นิปาตปุท\, สพฺพนามปุท-จสญฺญติ့ุ]} {ก็} {ปน}
\entry{ปพฺพชติ} {ขฺยา} {[ป + วช + อ + ติ]} {=อุปคจฺฉติ ย่อมเข้าถึง} {ปพฺพชติ}
\entry{ปพฺพชิต} {ปุ} {[ป + วช + ต]} {ผู้เข้าถึง} {ปพฺพชิโต}
\entry{ปมุสฺสติ} {ขฺยา} {[ป + มุส + ย + ติ]} {ไม่มีคำแปล} {ปมุสฺสติ}
\entry{ปร} {ติ} {ไม่มีการประกอบคำ} {อื่น} {ปรํ, ปเร, ปเรสํ, ปโร}
\entry{ปรมทิฏฺฐธมฺมนิพฺพาน} {นปุ} {[ปรม + ทิฏฺฐธมฺม + นิพพาน]} {การดับทุกข์ในชาติปัจจุบันว่าเป็นนิพพานอันล้ำเลิศ} {ปรมทิฏฺฐธมฺมนิพฺพานํ}
\entry{ปรมฺมรณ} {ติ} {ไม่มีการประกอบคำ} {ไม่มีคำแปล} {ปรมฺมรณา}
\entry{ปราชโย} {ปุ} {[ปรา + ชิ + ณ (อ)]} {ความพ่ายแพ้} {ปราชโย}
\entry{ปรามสติ} {ขฺยา} {[ปร + อา (ปรา) + มส + อ + ติ]} {ย่อมหลงยึดติด โดยผิดเพี้ยน} {ปรามสติ}
\entry{ปริจาเรติ} {ขฺยา} {[ปริ + จร + เณ + ติ]} {=อินฺทฺริยานิ อุปเนติ ย่อมหล่อเลี้ยงอินทรีย์ให้สุขสบาย} {ปริจาเรติ}
\entry{ปริณต} {ติ} {[ปริ + นมุ + ต]} {ผู้น้อมมา} {ปริณโต}
\entry{ปริตสฺสน} {อิตฺ\, นปุ} {[ปริ + ตส + ย + ยุ]} {ความปรารถนาเฝ้ารอ} {ปริตสฺสนา}
\entry{ปริตสฺสิตํ} {อิตฺ\, นปุ} {[ปริ + ตส + ย + ต]} {เป็นธรรมชาติที่เป็นไปด้วยอำนาจของตัณหาและทิฏฐิ จึงยังหวั่นไหวไม่มั่นคง} {ปริตสฺสิตํ}
\entry{ปริตสฺสิตวิปฺผนฺทิตวาร} {ปุ} {[ปริตสฺสิต + วิปฺผนฺทิต + วาร]} {ธรรมชาติที่เป็นไปด้วยอำนาจของตัณหาและทิฏฐิ จึงยังหวั่นไหวไม่มั่นคงนั่นเทียว} {ปริตสฺสิตวิปฺผนฺทิตวาร}
\entry{ปริตฺต} {ติ} {[ปริ + อตฺต]} {อันเล็ก} {ปริตฺตํ}
\entry{ปริตฺตสญฺญี} {ติ} {[ปริตฺตสญฺญา + อี]} {ธรรมชาติที่มีสัญญาเล็กน้อย} {ปริตฺตสญฺญี}
\entry{ปริพฺพาชก} {ปุ} {[ปริ + วช (วชฺช) + ณฺวุ]} {ปริพาชก} {ปริพฺพาชกสฺส, ปริพฺพาชโก}
\entry{ปริมทฺทน} {นปุ} {[ปริ + มชฺช + ยุ]} {การนวด} {ปริมทฺทนํ}
\entry{ปริยนฺตวนฺตุ} {ติ} {[ปริยนฺตวตี]} {มีขอบเขต} {ปริยนฺตวติํ}
\entry{ปริยาปนฺน} {ติ} {[ปริ + อา + ปท + ต]} {ผู้รวมอยู่} {ปริยาปนฺนา}
\entry{ปริโยทาต} {ติ} {[ปริ + อว + ทา + ต]} {ผุดผ่องแล้ว} {ปริโยทาเต}
\entry{ปริวฏุม} {ติ} {[ปริ + วฏฺฏ + อุม]} {สัณฐานกลม} {ปริวฏุโม}
\entry{ปริสุทฺธ} {ติ} {[ปริ + สุธ + ต]} {บริสุทธิ์แล้ว} {ปริสุทฺเธ}
\entry{ปริหารปถ} {ปุ} {[ปริหาร + ปถ]} {การพนันดวด} {ปริหารปถํ}
\entry{ปลฺลงฺก} {ปุ} {[ปริ + อติ + ณ]} {เตียงอันมีการแกะสลักรูปสัตว์ร้ายไว้ที่ขา} {ปลฺลงฺกํ}
\entry{ปเวเทติ} {ขฺยา} {[ป + วิท + เณ + ติ]} {ย่อมประกาศให้ผู้อื่นรู้} {ปเวเทติ}
\entry{ปสฺสามิ} {ขฺยา} {[ทิส + อ + มิ]} {ย่อมเห็น} {ปสฺสามิ}
\entry{ปหาน} {นปุ} {[ป + หา + ยุ\, ป + หาน]} {ละ} {ปหานา}
\entry{ปหาย} {พฺยย} {[ป + หา + ตฺวา]} {ละแล้ว} {ปหาย}
\entry{ปโหสิ} {ขฺยา} {[ป + หู + อ + สิ]} {ย่อมสามารถ[กล้าพอ]} {ปโหสิ}
\entry{ปาณ} {ปุ} {[ปณ + ณ]} {สัตว์น้ำ} {ปาณา}
\entry{ปาณาติปาต} {ปุ} {[ปาณ + อติปาต]} {การทำลายชีวิตของผู้อื่น} {ปาณาติปาตํ, ปาณาติปาตา}
\entry{ปาณิสฺสรํ} {ปุ} {[ปาณ + สร]} {การตีฉิ่ง} {ปาณิสฺสรํ}
\entry{ปาตุภวติ} {ขฺยา} {[ปาตุ + ภู +อ + ติ]} {ย่อมปรากฏ} {ปาตุภวติ}
\entry{ปานกถา} {อิตฺ} {[ปา + กถา]} {คำพูดที่ไร้สาระเกี่ยวกับน้ำปานะ} {ปานกถํ}
\entry{ปานสนฺนิธิ} {ปุ\, อิตฺ} {[ปาน + สนฺนิธิ]} {การเก็บสะสมน้ำปานะ} {ปานสนฺนิธิํ}
\entry{ปิฏฺฐิโต} {พฺยย} {ไม่มีการประกอบคำ} {โดยตามหลังมาติดๆ} {ปิฏฺฐิโต}
\entry{ปิตุ} {ปุ} {ไม่มีการประกอบคำ} {บิดา} {ปิตา}
\entry{ปิสุณ} {ติ} {[ปิยสุญฺญ + กรณ\, ปิสุณ + กรณ. ปิส + อุณ]} {ที่ส่อเสียด[วาจาที่ก่อให้เกิดความแตกแยก]} {ปิสุณํ, ปิสุณาย}
\entry{ปีติ} {อิตฺ} {[ปี + ติ\, ปีนยติ กายจิตฺตํ ตปฺเปติ วฑฺเฒตีติ วา ปีติ. วิภาวินี. ๙๒]} {ปีติ} {ปีติยา}
\entry{ปีติคต} {ติ} {[ปีติ + คต]} {ความอิ่มเอิบใจ} {ปีติคตํ}
\entry{ปีติภกฺข} {ติ} {[ปีติ + ภิกฺข]} {ผู้มีปีติเป็นภักษาหาร} {ปีติภกฺขา, ปีติภกฺโข}
\entry{ปีติสุข} {นปุ} {[ปีติสมฺปยุตฺต + สุข]} {อันเปี่ยมไปด้วยปีติและสุข} {ปีติสุขํ}
\entry{ปุจฺฉสิ} {ขฺยา} {[ปุจฺฉ + อ + สิ]} {ย่อมถาม} {ปุจฺฉสิ}
\entry{ปุญฺญกฺขย} {ปุ} {[ปุญฺญ + ขย]} {การสิ้นบุญเป็นเหตุ} {ปุญฺญกฺขยา}
\entry{ปุฏฺฐ} {ติ} {[ปุจฺฉ + ต]} {เมื่อถูกถาม} {ปุฏฺฐา, ปุฏฺโฐ}
\entry{ปุถุชฺชน} {ปุ} {[ปุถุ + ชน]} {ปุถุชน} {ปุถุชฺชโน}
\entry{ปุพฺพ} {ปุ} {ไม่มีการประกอบคำ} {ส่วนแห่งขันธ์ ๕ ที่เป็นอดีต} {ปุพฺเพ}
\entry{ปุพฺพนฺตํ} {ปุ\, นปุ} {[ปุพฺพ + อนฺต]} {ส่วนแห่งขันธ์ ๕ ที่เป็นอดีต} {ปุพฺพนฺตํ}
\entry{ปุพฺพนฺตกปฺปิก} {ติ} {[ปุพฺพนฺตกปฺป + อิก]} {ผู้มีความเห็นผิดคิดเรื่องที่เกี่ยวกับส่วนแห่งขันธ์ ๕ ที่เป็นอดีตไปตามอำนาจแห่งตัณหาและทิฏฐิของตนด้วย} {ปุพฺพนฺตกปฺปกา, ปุพฺพนฺตกปฺปิกา}
\entry{ปุพฺพนฺตานุทิฏฺฐิ} {อิตฺ} {[ปุพฺพนฺต + อนุทิฏฺฐิ]} {ผู้มีความเห็นผิดคิดเรื่องที่เกี่ยวกับส่วนแห่งขันธ์ ๕ ที่เป็นอดีตอยู่เนืองนิตย์} {ปุพฺพนฺตานุทิฏฺฐิโน}
\entry{ปุพฺพนฺตาปรนฺต} {ติ} {[ปุพฺพนฺต + อปรนฺต]} {ส่วนแห่งขันธ์ ๕ ที่เป็นอดีตและส่วนแห่งขันธ์ ๕ ที่เป็นอนาคต} {ปุพฺพนฺตาปรนฺตํ}
\entry{ปุพฺพนฺตาปรนฺตกปฺปิก} {ติ} {[ปุพฺพนฺตาปรนฺตกปฺป + อิก]} {ผู้มีความเห็นผิดคิดเรื่องที่เกี่ยวกับส่วนแห่งขันธ์ ๕ ที่เป็นอดีตและส่วนแห่งขันธ์ ๕ ที่เป็นอนาคตด้วย} {ปุพฺพนฺตาปรนฺตกปฺปกา, ปุพฺพนฺตาปรนฺตกปฺปิกา}
\entry{ปุพฺพนฺตาปรนฺตานุทิฏฺฐิ} {อิตฺ} {[ปุพฺพนฺตาปรนฺต + อนุทิฏฺฐิ]} {ผู้มีความเห็นผิดคิดเรื่องที่เกี่ยวกับส่วนแห่งขันธ์ ๕ ที่เป็นอดีตและส่วนแห่งขันธ์ ๕ ที่เป็นอนาคตอยู่เนืองนิตย์} {ปุพฺพนฺตาปรนฺตานุทิฏฺฐิโน}
\entry{ปุพฺพเปตกถา} {อิตฺ} {[ปุพฺพเปต + กถา]} {คำพูดที่ไร้สาระเกี่ยวกับญาติที่เสียชีวิตไปแล้ว} {ปุพฺพเปตกถํ}
\entry{ปุพฺเพนิวาส} {ติ} {[ปุพฺเพ + นิวาส]} {อดีตชาติ} {ปุพฺเพนิวาสํ}
\entry{ปุริสกถา} {ปุ} {[ปุริส + กถา]} {คำพูดที่ไร้สาระเกี่ยวกับบุรุษ} {ปุริสกถํ}
\entry{ปุริสลกฺขณ} {นปุ} {[ปุริส + ลกฺขณ]} {ศาสตร์ว่าด้วยการทำนายลักษณะของบุรุษ} {ปุริสลกฺขณํ}
\entry{ปุเร} {พฺยย} {ไม่มีการประกอบคำ} {ไว้ก่อน} {ปุเร}
\entry{เปกฺข} {อิตฺ\, นปุ} {[ป + อิกฺข + อ]} {วงเต้นรำ} {เปกฺขํ}
\entry{เปมนีย} {ติ} {[เปม + อิย]} {ก่อให้เกิดความรัก} {เปมนียา}
\entry{โปรี} {อิตฺ} {[ปุร + ณ + อี]} {วาจาของชาวเมือง(หรือเป็นคำพูดของผู้ดี)} {โปรี}
\end{multicols}
\newpage
\needspace{5cm} \section*{\Huge ผ}
\noindent\rule{\textwidth}{0.4pt}
\begin{multicols}{2}
\entry{ผรุส} {ติ} {[ผร + อุส + อ]} {หยาบคาย} {ผรุสํ, ผรุสาย}
\entry{ผล} {นปุ} {[ผล + อ]} {ผล} {ผลํ}
\entry{ผลุพีช} {นปุ} {[ผลุ + พีช]} {ไม่มีคำแปล} {ผลุพีชํ}
\entry{ผสฺส} {ติ} {[ผสฺส + ณ]\, ผสฺส (ปุ)  [ผุส + ณ]} {ผัสสะ} {ผสฺสา}
\entry{ผสฺสปจฺจย} {ปุ} {[ผสฺส + ปจฺจย]} {ผัสสะเป็นปัจจัย} {ผสฺสปจฺจยา}
\entry{ผสฺสปจฺจยวาร} {ปุ} {[ผสฺส + ปจฺจย + วาร]} {ผัสสปัจจยาวาระ [วาระว่าด้วยผัสสะเป็นปรัมปรปัจจัยให้เกิดมิจฉาทิฏฐิ]} {ผสฺสปจฺจยาวาร}
\entry{ผสฺสายตน} {ติ} {[ผสฺสายตน + ณ]\, ผสฺสายตน (น) [ผสฺส + อายตน]} {ผัสสายตนะ} {ผสฺสายตนานํ, ผสฺสายตเนหิ}
\entry{ผุสติ} {ขฺยา} {[ผุส + ย + เต]\, ผุสติ (กฺริ) [ผุส + อ + ติ]} {ย่อมสัมผัส} {ผุสติ}
\entry{ผุสามิ} {ขฺยา} {[ผุส + อ + มิ]} {ย่อมสัมผัส} {ผุสามิ}
\entry{ผุสฺส} {พฺยย} {[ผุส + ตฺวา]\, ผุสฺส (ปุ) [ผุส + สก]\, ผุสฺส (ปุ) [ผุสฺส + ณ]} {กระทบแล้ว} {ผุสฺส}
\end{multicols}
\newpage
\needspace{5cm} \section*{\Huge พ}
\noindent\rule{\textwidth}{0.4pt}
\begin{multicols}{2}
\entry{พลคฺค} {นปุ} {[พล + อคฺค]} {การตรวจนับพลสวนสนาม} {พลคฺคํ}
\entry{พหิทฺธา} {พฺยย} {[สตฺตมฺยนฺตนิปาต]} {อันอื่น} {พหิทฺธา}
\entry{พหุชนกนฺต} {ติ} {[พหุชน + กนฺต]} {เป็นวาจาอันเป็นที่ชอบใจของชนจำนวนมาก} {พหุชนกนฺตา}
\entry{พหุชนมนาป} {ติ} {[พหุชน + มนาป]} {เป็นวาจาเป็นที่ชื่นใจของคนจำนวนมาก} {พหุชนมนาปา}
\entry{พาหิร} {ติ} {[พหิ + อิรณ\, ธาน\, ฏี]} {ผู้อยู่ภายนอกพระนครหรืออาณาประเทศ} {พาหิรานํ}
\entry{พีชคามภูตคามสมารมฺภ} {ปุ} {[พีชคามภูตคาม + สมารมฺภ]} {การทำลายพีชคาม[พันธุ์พืชมีเมล็ด เป็นต้น] และภูตคาม[ต้นพืชที่งอกออกเป็นลำต้นแล้ว]} {พีชคามภูตคามสมารมฺภํ, พีชคามภูตคามสมารมฺภา}
\entry{พีชพีช} {นปุ} {[พีช + พีช]} {พันธุ์ไม้จำพวกเมล็ด} {พีชพีชํ}
\entry{พุทฺธ} {ปุ\, นปุ} {[พุธ + ต]\, พุทฺธ (ปุ) [พุทฺธ + ปจฺเจกพุทฺธ + อนุพุทฺธ]\, พุทฺธ (ปุ) [พุทฺธ + ณ]} {พระพุทธเจ้า} {พุทฺธสฺส}
\entry{พฺยากเรยฺยํ} {ขฺยา} {[วิ + อา + กร + เอยฺยํ]} {พึงตอบ} {พฺยากเรยฺยํ}
\entry{พฺยากโรติ} {ขฺยา} {[วิ + อา + กร + โอ + ติ]} {ย่อมพยากรณ์\, ย่อมกระทำให้แจ่มแจ้ง} {พฺยากโรติ}
\entry{พฺรหฺม} {ปุ} {[พฺรูห + ม\, พห + ม. พฺรห + ม. นีติ.ธา. ๑๙๘. รู. ๔๐๖. ธานฺ.ฏี. ๑๕]} {พระญาณอันประเสริฐ[พระสัพพัญญุตญาณ]} {พฺรหฺมา, พฺรหฺมุนา}
\entry{พฺรหฺมจารี} {ปุ} {[พฺรหฺม + จร +  ณี]} {ผู้มีปกติประพฤติธรรมอันประเสริฐ} {พฺรหฺมจารี}
\entry{พฺรหฺมชาล} {ติ} {[พฺรหฺม + ชาล]} {ข่ายกล่าวคือพระสัพพัญญุตญาณอันประเสริฐ} {พฺรหฺมชาลํ}
\entry{พฺรหฺมทตฺต} {ติ} {[พฺรหฺม + ทินฺน]} {ผู้มีนามว่าพรหมทัต} {พฺรหฺมทตฺเตน, พฺรหฺมทตฺโต}
\entry{พฺรหฺมวิมาน} {ปุ\, นปุ} {[พฺรหฺม + วิมาน]} {วิมานของพรหม} {พฺรหฺมวิมานํ}
\entry{พฺราหฺมณ} {ปุ} {[พฺราหฺม + ณ]} {พราหมณ์} {พฺราหฺมณา, พฺราหฺมณานํ, พฺราหฺมโณ}
\end{multicols}
\newpage
\needspace{5cm} \section*{\Huge ภ}
\noindent\rule{\textwidth}{0.4pt}
\begin{multicols}{2}
\entry{ภคว} {ปุ} {[ภา + วน + กฺวิ]} {พระผู้มีพระภาค} {ภควา}
\entry{ภควํเนตฺติก} {ติ} {[ภควนฺตุ + เนตฺติ + ก]} {ไม่มีคำแปล} {ภวเนตฺติโก}
\entry{ภควนฺตุ} {ปุ} {[ภค + วนฺตุ]} {พระผู้มีพระภาค} {ภควโต, ภควนฺตํ}
\entry{ภควา} {ปุ} {[รุฬฺหีนามปุท]\, ภควา (ปุ) [ภา + วนฺตุ + คมน]} {พระผู้มีพระภาคเจ้า} {ภควตา}
\entry{ภญฺญมาน} {ติ} {[ภณ + ย + มาน]} {ทรงแสดงอยู่} {ภญฺญมาเน}
\entry{ภติกมฺม} {นปุ} {[ภติ + กมฺม]} {ไม่มีคำแปล} {ภูติกมฺมํ}
\entry{ภนฺต} {ติ} {[ภมุ + ต]} {ข้าแต่พระองค์ผู้เจริญ} {ภนฺเต}
\entry{ภย} {นปุ} {[ภี + ณ]} {ภัยพิบัติ} {ภยํ}
\entry{ภยกถา} {อิตฺ} {[ภย + กถา]} {คำพูดที่ไร้สาระเกี่ยวกับภยันตราย} {ภยกถํ}
\entry{ภว} {ปุ} {[ภู + ณ]\, ภว (ปุ) [ภว + ทิฏฺฐิ]\, ภว (กฺริ) [ภู + อ + หิ]} {ผู้เป็นเจ้า} {ภวํ, ภโว}
\entry{ภวนฺติ} {ขฺยา} {[ภู + อ + อนฺติ]} {ย่อมเป็น} {ภวนฺติ}
\entry{ภวปจฺจย} {ปุ} {[ภว + ปจฺจย]} {มีภพเป็นอุปนิสสยปัจจัย} {ภวปจฺจยา}
\entry{ภวิสฺสติ} {ขฺยา} {[ภู + สฺสติ]\, ภวิสฺสติ (ติ) [ภู + สนฺตุ (= ภวิสฺสนฺตุ) + สฺมึ]} {จักเกิดขึ้น} {ภวิสฺสติ}
\entry{ภาสติ} {ขฺยา} {[ภาสุ + อ + ติ]\, ภาสติ (กฺริ) [ภาส + อ + ติ]} {ก็ยังกล่าว} {ภาสติ}
\entry{ภาสิต} {นปุ} {[ภาส + ต]} {กล่าวแล้ว} {ภาสิตํ, ภาสิตา}
\entry{ภาเสยฺยุํ} {ขฺยา} {[ภาส + เอยฺยุํ]} {พึงกล่าว} {ภาเสยฺยุํ}
\entry{ภิกฺขุ} {ปุ} {[ภิกฺข + รู]} {ภิกษุ (หรือโยคีนักปฏิบัติ)} {ภิกฺขเว, ภิกฺขุ, ภิกฺขู, ภิกฺขูนํ}
\entry{ภิกฺขุสํฆ} {ปุ} {[ภิกฺขุ + สํฆ]} {หมู่แห่งภิกษุ} {ภิกฺขุสํฆํ, ภิกฺขุสํเฆน}
\entry{ภิกฺขุสต} {นปุ} {[ภิกฺขุ + สต]} {ภิกษุ ๑๐๐ รูป} {ภิกฺขุสเตหิ}
\entry{ภินฺทนฺต} {ติ} {[ภิทิ + อนฺต]} {ทำลายอยู่} {ภินฺทนฺตา}
\entry{ภินฺน} {ปุ} {[ภิทิ + ต]} {ผู้แตกแยกกัน} {ภินฺนานํ}
\entry{ภุญฺชิตฺวา} {พฺยย} {[ภุช + ตฺวา]} {บริโภคใช้สอยแล้ว} {ภุญฺชิตฺวา}
\entry{ภูต} {ปุ\, นปุ} {[ภู + ต]} {คำพูดที่เป็นจริง} {ภูตํ}
\entry{ภูตโต} {พฺยย} {ไม่มีการประกอบคำ} {โดยเป็นคำพูดที่เป็นจริง} {ภูตโต}
\entry{ภูตภพฺย} {ปุ} {[ภูต + ภพฺย]} {สัตว์ที่เกิดแล้วและที่กำลังเกิด} {ภูตภพฺยานํ}
\entry{ภูตวาที} {ติ} {[ภูต + วท + ณี]} {ผู้มีปกติกล่าวเฉพาะความจริง} {ภูตวาที}
\entry{ภูตวิชฺชา} {อิตฺ} {[ภูต + วิชฺชา]} {ศาสตร์ว่าด้วยเวทมนต์ไล่ผี} {ภูตวิชฺชา}
\entry{ภูมิจาล} {ปุ} {[ภูมิ + จาล]} {การไหวแห่งแผ่นดิน} {ภูมิจาโล}
\entry{ภูริกมฺม} {นปุ} {[ภูรี + กมฺม]} {การเรียนมนต์ในห้องใต้ดินหรือเรือนที่สร้างด้วยดินเหนียว} {ภูริกมฺมํ}
\entry{ภูริวิชฺชา} {อิตฺ} {[ภูรี + วิชฺชา]} {ศาสตร์ว่าด้วยเวทมนต์ที่ต้องเรียนภายในพื้นที่บ้านเท่านั้น} {ภูริวิชฺชา}
\entry{เภท} {ปุ} {[ภิทิ + ณ]} {การแตกดับ} {เภทา, เภทาย}
\entry{โภ} {พฺยย} {[อาลุปอนก]} {ผู้เจริญ} {โภ}
\entry{โภชน} {นปุ} {[โภชน + โภชน]\, โภชน (น) [ภุช + ยุ]} {โภชนะ} {โภชนานิ}
\entry{โภต} {ปุ} {ไม่มีการประกอบคำ} {ผู้เป็นเจ้า} {โภตา}
\entry{โภนฺโต} {ปุ} {[ภวนฺต + โย]} {ผู้เจริญ} {โภนฺโต}
\end{multicols}
\newpage
\needspace{5cm} \section*{\Huge ม}
\noindent\rule{\textwidth}{0.4pt}
\begin{multicols}{2}
\entry{มชฺฌิมสีล} {นปุ} {[มชฺฌิม + สีล]} {มัชฌิมศีล [พระพุทธดำรัสเรื่องศีลที่มีเนื้อหาปานกลาง]} {มชฺฌิมสีล, มชฺฌิมสีลํ}
\entry{มญฺเญ} {ขฺยา} {[มน + ย + ม]\, มญฺเญ (กฺริ) [มน + ย + เอยฺยํ]\, มญฺเญ (พฺย)  มญฺเญ (กฺริ) [มน + ย + มิ]} {เห็นจะ} {มญฺเญ}
\entry{มณิ} {ปุ\, อิตฺ} {[มณ + อิ]} {การปักปิ่นเพชร} {มณิํ}
\entry{มณิลกฺขณ} {นปุ} {[มณิ + ลกฺขณ]} {ศาสตร์ว่าด้วยการทำนายลักษณะของแก้วมณี[เพชร]} {มณิลกฺขณํ}
\entry{มณฺฑนวิภูสนฏฺฐานานุโยค} {ปุ} {[มณฺฑน + วิภูสน + ฐาน + อนุโยค]} {การขวนขวายในการประดับและตกแต่งร่างกายอันเป็นเหตุนำมาซึ่งความประมาท} {มณฺฑนวิภูสนฏฺฐานานุโยคํ, มณฺฑนวิภูสนฏฺฐานานุโยคา}
\entry{มณฺฑลมาฬ} {ปุ} {[มณฺฑล + มาฬ]} {ศาลาอันเป็นที่นั่งประชุม} {มณฺฑลมาเฬ, มณฺฑลมาโฬ}
\entry{มน} {ปุ} {[มน + อ]} {มนะ} {มโน}
\entry{มเนสิกา} {อิตฺ} {[มน + เอสิกา]} {การพนันทายใจ} {มเนสิกํ}
\entry{มโนปณิธิ} {ติ} {[มน + ปณิธิ]} {ความปรารถนา} {มโนปณิธิ}
\entry{มโนปโทสิก} {ติ} {[มโนปโทสี + ก]} {เทวดาจำพวกมโนปโทสิกะ} {มโนปโทสิกา}
\entry{มโนมย} {ติ} {[มน + มย]} {อันสำเร็จแต่ฌานจิต [อันถือปฏิสนธิด้วยฌานจิต]} {มโนมยา, มโนมโย}
\entry{มนฺท} {ติ} {[มน + ทก]\, มนฺท (กฺริ\,วิ) [มนฺทตญฺญกฺริยาวิเสสนอนก]\, มนฺท (ปุ) [มทิ + อ]} {คนมีปัญญาน้อย} {มนฺโท}
\entry{มนฺทรตฺต} {ติ} {[มนฺท + รตฺต]} {ความเป็นผู้มีปัญญาน้อย} {มนฺทตฺตา}
\entry{มรณ} {นปุ} {[มร + ยุ]} {ความตาย} {มรณา}
\entry{มหติ} {ขฺยา} {[มห + อ + ติ]} {หมู่ใหญ่} {มหติ}
\entry{มหนฺต} {ปุ} {[มห + อนฺต]} {หมู่ใหญ่} {มหตา}
\entry{มหึสยุทฺธ} {นปุ} {[มหึส + ยุทฺธ]} {การแข่งขันชนกระบือ} {มหิสยุทฺธํ}
\entry{มหึสลกฺขณ} {นปุ} {[มหึส + ลกฺขณ]} {ศาสตร์ว่าด้วยการทำนายลักษณะของกระบือ} {มหิสลกฺขณํ}
\entry{มหาพฺรหฺม} {ปุ} {[มหนฺต + พฺรหฺม]} {ท้าวมหาพรหม} {มหาพฺรหฺมา}
\entry{มหามตฺตกถา} {อิตฺ} {[มหามตฺต + กถา]} {คำพูดที่ไร้สาระเกี่ยวกับมหาอำมาตย์} {มหามตฺตกถํ}
\entry{มหาสีล} {นปุ} {[มหนฺต + สีล]} {มหาศีล [พระพุทธดำรัสเรื่องศีลที่มีเนื้อหายาว]} {มหาสีล, มหาสีลํ}
\entry{มเหสกฺขตร} {ติ} {[มเหสกฺข + ตร]} {เป็นผู้มีฤทธิ์เดชมากกว่าพรหมผู้เกิดภายหลัง} {มเหสกฺขตโร}
\entry{มาณว} {ปุ} {[มนุ + ณ]} {เด็กหนุ่ม} {มาณเวน, มาณโว}
\entry{มาตาเปตฺติกสมฺภว} {ติ} {[มาตาเปตฺติก + สมฺภาว]} {อันเกิดจากเลือดเนื้อเชื้อไขของมารดาบิดา} {มาตาเปตฺติกสมฺภโว}
\entry{มาลากถา} {อิตฺ} {[มาลา + กถา]} {คำพูดที่ไร้สาระเกี่ยวกับดอกไม้} {มาลากถํ}
\entry{มาลาวิเลปน} {นปุ} {[มาลา + วิ + ลิป + ยุ]} {การทัดดอกไม้ การอบน้ำหอม และการใช้เครื่องประทินผิว} {มาลาวิเลปนํ}
\entry{มิคจกฺก} {นปุ} {[มิค + จกฺก]} {ศาสตร์ว่าด้วยการทำนายเสียงสัตว์ร้อง[สัตว์ทุกชนิดที่เหลือจากที่ระบุแล้วข้างต้นโดยเฉพาะสัตว์ ๔ เท้า]} {มิคจกฺกํ}
\entry{มิคลกฺขณ} {นปุ} {[มิค + ลกฺขณ]} {ศาสตร์ว่าด้วยการทำนายลักษณะของมฤค[สัตว์ ๔ เท้าที่เหลือ]} {มิคลกฺขณํ}
\entry{มิจฺฉาชีว} {ปุ} {[มิจฺฉา + อา + ชีว + ณ\, มิจฺฉาอาชีวนฺติ เอเตนาติ มิจฺฉชีโว\, สี\,ฏี\,ใหม่ ๒\,๙๘.]} {อันเป็นการเลี้ยงชีพในทางที่ผิด} {มิจฺฉาชีวา, มิจฺฉาชีเวน}
\entry{มิจฺฉาปฏิปนฺน} {ติ} {[มิจฺฉา + ปฏิปนฺน]} {เป็นผู้ปฏิบัติผิด} {มิจฺฉาปฏิปนฺโน}
\entry{มุขจุณฺณก} {ปุ\, นปุ} {[มุขจุณฺณ + ก]} {การผัดหน้า} {มุขจุณฺณกํ}
\entry{มุขโหม} {นปุ} {[มุข + โหม]} {ศาสตร์ว่าด้วยวิธีการพ่นเครื่องเซ่น มีเมล็ดผักกาด เป็นต้น} {มุขโหมํ}
\entry{มุขาเลปน} {นปุ} {[มุข + อาเลปน]} {การทาหน้า} {มุขาเลปนํ}
\entry{มุฏฺฐิยุทฺธ} {ปุ} {[มุฏฺฐิ + ยุทฺธ]} {การแข่งขันชกมวย} {มุฏฺฐิยุทฺธํ}
\entry{มุทฺทา} {อิตฺ} {[มุท + ท + อา]} {วิชาคำนวณนับด้วยการหักข้อนิ้วมือ} {มุทฺทา}
\entry{มุสา} {พฺยย} {ไม่มีการประกอบคำ} {เป็นคำเท็จ} {มุสา}
\entry{มุสาวาท} {ปุ} {[มุสา + วาท]} {การกล่าวคำเท็จ} {มุสาวาทํ, มุสาวาทา}
\entry{มุสาวาทปริเชคุจฺฉา} {อิตฺ} {[มุสาวาท + ปริเชคุจฺฉา]} {รังเกียจคำเท็จ} {มุสาวาทปริเชคุจฺฉา}
\entry{มูลพีช} {นปุ} {[มูล + พีช]} {พันธุ์ไม้จำพวกราก} {มูลพีชํ}
\entry{มูลเภสชฺช} {นปุ} {[มูล + เภสชฺช]} {ยาหลักในการรักษาโรคที่เกิดขึ้นแรกเริ่ม [ที่มิใช่โรคแทรกซ้อน] ที่เกิดขึ้นบริเวณกาย} {มูลเภสชฺชานํ}
\entry{มูสิกจฺฉินฺน} {ติ} {[มูสิก + ฉินฺน]} {ศาสตร์ว่าด้วยการทำนายหนูกัดผ้า} {มูสิกจฺฉินฺนํ}
\entry{มูสิกวิชฺชา} {อิตฺ} {[มูสิก + วิชฺชา]} {ศาสตร์ว่าด้วยการรักษาพิษหนูกัด} {มูสิกวิชฺชา}
\entry{เมณฺฑยุทฺธ} {นปุ} {[เมณฺฑ + ยุทฺธ]} {ศาสตร์ว่าด้วยการทำนายลักษณะของแกะ} {เมณฺฑยุทฺธํ}
\entry{เมณฺฑลกฺขณ} {นปุ} {[เมณฺฑ + ลกฺขณ]} {ศาสตร์ว่าด้วยการทำนายลักษณะของแกะ} {เมณฺฑลกฺขณํ}
\entry{เมถุน} {ปุ\, นปุ} {[มิถุน + ณ]} {เมถุน} {เมถุนา}
\entry{โมกฺขจิกา} {อิตฺ} {[โมกฺข + จิ + ณฺวุ + อา]} {การพนันหกคะเมน} {โมกฺขจิกํ}
\entry{โมมูห} {ปุ} {[โมห + มูห]} {คนหลงงมงาย} {โมมูโห}
\entry{โมมูหตฺต} {นปุ} {[โมมูห + ตฺต]} {ความเป็นผู้หลงงมงาย} {โมมูหตฺตา}
\end{multicols}
\newpage
\needspace{5cm} \section*{\Huge ย}
\noindent\rule{\textwidth}{0.4pt}
\begin{multicols}{2}
\entry{ย} {อิตฺ} {ไม่มีการประกอบคำ} {ใด} {ยํ, ยา, เย, เยน, เยหิ, โย}
\entry{ยโต} {พฺยย} {[ยนฺต + ส]\, ยโต (พฺย) [ย + โต]\, ยโต (ปุ) [ยต + สี]} {ในการใด} {ยโต}
\entry{ยตฺถ} {พฺยย} {[ย + ถ\, ตฺถ-หุ อสทิสทฺเวะโภปฺรุ]} {ใด} {ยตฺถ}
\entry{ยถา} {พฺยย} {[ย + ถา]} {ด้วยเหตุใด} {ยถา}
\entry{ยถาภุจฺจ} {ติ} {[ยถา + ณฺย]} {ตามความเป็นจริง} {ยถาภุจฺจํ}
\entry{ยถาภูต} {ติ} {[ยถา + ภูต]} {ตามความเป็นจริง} {ยถาภูตํ}
\entry{ยถาวชฺช} {นปุ} {[ยถา + วชฺช]} {การพนันแข่งขันล้อเลียนคนพิการ} {ยถาวชฺชํ}
\entry{ยถาสมาหิต} {ติ} {[ยถา + สมิ (มี) หิต]} {ตั้งมั่นดีแล้ว} {ยถาสมาหิเต}
\entry{ยานกถา} {อิตฺ} {[ยาน + กถา]} {คำพูดที่ไร้สาระเกี่ยวกับยานพาหนะ} {ยานกถํ}
\entry{ยานสนฺนิธิ} {ปุ} {[ยาน + สนฺนิธิ]} {การเก็บสะสมยานพาหนะ} {ยานสนฺนิธิํ}
\entry{ยาว} {พฺยย} {[รู]} {สิ้นกาลเพียงไร} {ยาว}
\entry{ยาวญฺจ} {พฺยย} {ไม่มีการประกอบคำ} {อย่างไม่จำกัด} {ยาวญฺจ}
\entry{ยุทฺธกถา} {อิตฺ} {[ยุทฺธ + กถา]} {คำพูดที่ไร้สาระเกี่ยวกับการรบ} {ยุทฺธกถํ}
\entry{เยภุยฺเยน} {พฺยย} {[เยภุยฺเยนิจฺจาทโย วิภตฺยนฺตปติรูปกา]} {โดยส่วนมาก} {เยภุยฺเยน}
\end{multicols}
\newpage
\needspace{5cm} \section*{\Huge ร}
\noindent\rule{\textwidth}{0.4pt}
\begin{multicols}{2}
\entry{รญฺญํ} {ปุ} {[ราช + นํ]} {พระราชา} {รญฺญํ}
\entry{รตฺติ} {อิตฺ} {[รมุ + ติ]} {ราตรี} {รตฺติยา}
\entry{รตฺตูปรต} {ติ} {[รตฺต + อุปรต]} {เป็นผู้งดเว้นจากการฉันอาหารในเวลากลางคืน} {รตฺตูปรโต}
\entry{รถก} {ปุ} {[รถ + ก]} {การพนันของเล่นที่เป็นรถเล็กๆ} {รถกํ}
\entry{รถตฺถร} {ปุ\, นปุ} {[รถ + อตฺถร]} {เครื่องลาดสำหรับปูบนรถ} {รถตฺถรํ}
\entry{ราค} {ปุ} {[รค + ณ]} {ราคะ[ความกำหนัด]} {ราโค}
\entry{ราชกถา} {อิตฺ} {[ราช + กถา]} {คำพูดที่ไร้สาระเกี่ยวกับพระราชา} {ราชกถํ}
\entry{ราชคห} {นปุ} {[ราช + คห]} {กรุงราชคฤห์} {ราชคหํ}
\entry{ราชมหามตฺต} {ปุ} {[ราช + มหามตฺต]} {ราชมหาอำมาตย์} {ราชมหามตฺตานํ}
\entry{ราชาคารก} {ติ} {[ราชาคาร + ก]} {ที่ศาลาอันเป็นที่เกษมสำราญของพระราชา} {ราชาคารเก}
\entry{รูปสญฺญา} {อิตฺ} {[รูป + สญฺญา]\,  รูปสญฺญา (ถี) [รูปสญฺญา + รูปสญฺญ]\, รูปสญฺญา (ถี) [รูป + สญฺญา]} {รูปสัญญา} {รูปสญฺญานํ}
\entry{รูปี} {ติ} {[รูป + อี]} {เป็นธรรมชาติที่มีความเสื่อมสลาย(หรือเป็นธรรมชาติที่มีรูป)} {รูปี}
\entry{โรค} {ปุ} {[รุช + ณ]} {โรคระบาด} {โรโค}
\end{multicols}
\newpage
\needspace{5cm} \section*{\Huge ล}
\noindent\rule{\textwidth}{0.4pt}
\begin{multicols}{2}
\entry{ลกฺขณ} {นปุ} {[ลกฺข + ยุ]\, ลกฺขณ (น) [ลกฺขณ + อภินิเวส]\, ลกฺขณ (น) [ลกฺขณ + กถน]\, ลกฺขณ (ติ) [ลกฺขณ + ณ]\, ลกฺขณ (ติ) [ลกฺขี + ณ]\, ลกฺขณ (น) [ลกฺขณ + ปฏิลาภ]\, ลกฺขณ} {ศาสตร์ว่าด้วยการทำนายลักษณะ} {ลกฺขณํ}
\entry{ลชฺชี} {ติ} {[ลชฺช + ณี]} {เป็นผู้มีความละอายจากกายทุจริตเป็นต้น} {ลชฺชี}
\entry{ลปก} {ติ} {[ลป + ณฺวุ]} {เป็นคนพูดเอาดีใส่ตัว} {ลปกา}
\entry{ลาภ} {ปุ} {[ลภ + ณ]\, ลาภ (ติ) [ลาภ + เหตุ]\, ลาภ (ปุ) [ลภ + ณาเป + ณ]} {ลาภ} {ลาภํ, ลาเภน}
\entry{โลก} {ปุ} {[โลก + อ]} {โลก} {โลกํ, โลกสฺมิํ, โลกสฺส, โลเก, โลโก}
\entry{โลกกฺขายิกา} {อิตฺ} {[โลก + อกฺขายิกา]} {คำพูดที่ไร้สาระเกี่ยวกับโลก} {โลกกฺขายิกํ}
\entry{โลกธาตุ} {อิตฺ} {[โลก + ธาตุ]} {โลกธาตุ} {โลกธาตุ}
\entry{โลกายต} {นปุ} {[โลก + อายต- (=น + ยต + อ]} {วิชา โลกายตะ} {โลกายตํ}
\entry{โลหิตโหม} {นปุ} {[โลหิต + โหม]} {ศาสตร์ว่าด้วยวิธีการบูชาไฟด้วยเลือดจากไหล่และเข่าด้านขวา เป็นต้น} {โลหิตโหมํ}
\end{multicols}
\newpage
\needspace{5cm} \section*{\Huge ว}
\noindent\rule{\textwidth}{0.4pt}
\begin{multicols}{2}
\entry{ว} {ปุ} {[วา + อ]} {ไม่มีคำแปล} {ว, วา, โว}
\entry{วํส} {ปุ} {[วํส + ณ]} {การเล่นโดยการโหนลำไม้ไผ่} {วํสํ}
\entry{วงฺกก} {ปุ\, นปุ} {[วงฺก + ก]} {การพนันด้วยคันไถ} {วงฺกกํ}
\entry{วจนีย} {ติ} {[วจ + อนีย]} {คำที่ควรพูด} {วจนียํ}
\entry{วญฺฌ} {ติ} {[วญฺฌ + ก]\, วญฺฌ (ติ) [วน + ฌก]} {เป็นธรรมชาติที่ไม่ให้ผล[เสมือนสิ่งที่เป็นหมัน]} {วญฺโฌ}
\entry{วฏฺฏกยุทฺธ} {นปุ} {[วฏฺฏก + ยุทฺธ]} {การแข่งขันชนนกคุ่ม} {วฏฺฏกยุทฺธํ}
\entry{วฏฺฏกลกฺขณ} {นปุ} {[วฏฺฏก + ลกฺขณ]} {ศาสตร์ว่าด้วยการทำนายลักษณะของนกคุ่ม} {วฏฺฏกลกฺขณํ}
\entry{วณฺฏจฺฉินฺน} {ติ} {[วณฺฏ + ฉินฺน]} {ถูกตัดขั้วแล้ว} {วณฺฏจฺฉินฺนาย}
\entry{วณฺฏูปนิพนฺธน} {ติ} {[วณฺฏ + อุปนิพนฺธน]} {ไม่มีคำแปล} {วณฺฏปฏิพนฺธานิ}
\entry{วณฺณ} {ติ} {[วณฺณ + ภาสน]\, วณฺณ (ติ) [วณฺณ + ณ]\, วณฺณ (ปุ\,น) [วณฺณ + ณ]} {คำสรรเสริญ} {วณฺณํ}
\entry{วณฺณวนฺตตร} {ติ} {[วณฺณวนฺตุ + ตร]} {เป็นผู้มีผิวพรรณเลิศกว่า} {วณฺณวนฺตตโร}
\entry{วต} {นปุ} {[วต + อ]} {อย่างแน่นอน} {วต}
\entry{วตฺถ} {นปุ} {ไม่มีการประกอบคำ} {ผ้า} {วตฺถานิ}
\entry{วตฺถลกฺขณ} {นปุ} {[วตฺถ + ลกฺขณ]} {ศาสตร์ว่าด้วยการทำนายลักษณะของผ้า} {วตฺถลกฺขณํ}
\entry{วตฺถสนฺนิธิ} {ปุ} {[วตฺถ + สนฺนิธิ]} {การเก็บสะสมผ้ามีจีวร เป็นต้น} {วตฺถสนฺนิธิํ}
\entry{วตฺถุ} {นปุ} {[วตฺถุ + ก (กน)]\, วตฺถุ (น) [วส + ถุ]} {เหตุ} {วตฺถูหิ}
\entry{วตฺถุกมฺม} {นปุ} {[วตฺถุ + กมฺม]} {การปลูกเรือนในพื้นที่ใหม่} {วตฺถุกมฺมํ}
\entry{วตฺถุปริกมฺม} {นปุ} {[วตฺถุ + ปริกมฺม]} {การบวงสรวงพื้นที่} {วตฺถุปริกรณํ}
\entry{วตฺถุวิชฺชา} {อิตฺ} {[วตฺถุ + วิชฺชา]} {ศาสตร์ว่าด้วยการทำนายลักษณะพื้นที่สำหรับสร้างบ้าน เป็นต้น} {วตฺถุวิชฺชา}
\entry{วทมาน} {ติ} {[วท + มาน]} {เมื่อประสงค์จะกล่าว} {วทมานา, วทมาโน}
\entry{วทามิ} {ขฺยา} {[วท + อ + มิ]} {ย่อมกล่าว} {วทามิ}
\entry{วเทยฺย} {ขฺยา} {[วท + อ + เอยฺย]} {ก็คงได้แค่กล่าว} {วเทยฺย}
\entry{วเทยฺยํ} {ขฺยา} {[วท + อ + เอยฺยํ]} {ชื่อว่ากล่าวได้} {วเทยฺยุํ}
\entry{วเทสิ} {ขฺยา} {[วท + อ + สิ.]\, วเทสิ (กฺริ) [วท + อี]} {ย่อมกล่าว} {วเทสิ}
\entry{วมน} {นปุ} {[วมน + ก.]\, วมน (น) [วมุ + ยุ]} {การทำให้อาเจียน[การปรุงยาสำรอก]} {วมนํ}
\entry{วสวตฺตี} {ติ} {[วสวตฺต + อี]} {เป็นผู้ยังชนทั้งปวงให้เป็นไปตามอำนาจของตน} {วสวตฺตี}
\entry{วสี} {ขฺยา} {[วส + อี]\, วสี (ถี) [วส + อ +  อี]\, วสี (ติ) [วส + สี]} {เป็นผู้มีอำนาจ[เป็นผู้ช่ำชองในฌานสมาบัติ]} {วสี}
\entry{วสฺสกมฺม} {นปุ} {[วสฺส + กมฺม]} {การทำบัณเฑาะก์ให้เป็นชาย} {วสฺสกมฺมํ}
\entry{วาจา} {อิตฺ} {ไม่มีการประกอบคำ} {วาจา} {วาจํ, วาจา, วาจาย}
\entry{วาจาวิกฺเขป} {ปุ} {[วาจา + วิกฺเขป]} {ความซัดส่ายแห่งวาจา} {วาจาวิกฺเขปํ}
\entry{วาท} {ปุ} {[วท + ณ]\, วาท (ติ) [วาท + ณ]} {คำพูด} {วาโท}
\entry{วาทปฺปโมกฺข} {ปุ} {[วาท + ปโมกฺข]} {พ้นจากการถูกตำหนิ} {วาทปฺปโมกฺขาย}
\entry{วาทิต} {นปุ} {[วท + เณ + ต]} {การดีดสีตีเป่า[ประโคม]} {วาทิตํ}
\entry{วายสวิชฺชา} {อิตฺ} {[วายส + วิชฺชา]} {ศาสตร์ว่าด้วยการทำนายเสียงการ้อง} {วายสวิชฺชา}
\entry{วาลพีชนี} {อิตฺ} {[วาล + พีชนี]} {วาลพีชนีธารณํ การใช้พัดหางจามรี} {วาลวีชนิ}
\entry{วาลเวธิรูป} {ติ} {[วาลเวธี + รูป]} {ผู้เชี่ยวชาญแม่นยำในหลักการดุจนายขมังธนูผู้สามารถยิงธนูให้ถูกแม้กระทั่งขนหางของนางเนื้อป่าได้} {วาลเวธิรูปา}
\entry{วิกติกา} {อิตฺ} {[วิกติ + ก]} {เครื่องลาดขนแกะอันวิจิตรด้วยรูปสัตว์ร้ายมีรูปสิงห์และรูปเสือ เป็นต้น} {วิกติกํ}
\entry{วิกาลโภชน} {นปุ} {[วิกาล + โภชน + อชฺโฌหรณ]} {การฉันอาหารในเวลาวิกาล\&nbsp;} {วิกาลโภชนา}
\entry{วิกิรณ} {ติ} {[วิ + กร + ยุ]\, วิกิรณ (ถี\,น) [วิ + กิร +  ยุ]} {การดูฤกษ์จ่ายทรัพย์} {วิกิรณํ}
\entry{วิคตูปกฺกิเลส} {ติ} {[วิคต + อุปกฺกิเลส]} {ปราศจากอุปกิเลสแล้ว} {วิคตูปกฺกิเลเส}
\entry{วิคฺคาหิกกถา} {อิตฺ} {[วิคฺคาหิกา + กถา]} {คำพูดที่ขัดแย้งอันทำให้ทะเลาะเบาะแว้งกัน} {วิคฺคาหิกกถํ, วิคฺคาหิกกถาย}
\entry{วิฆาต} {ติ} {[วิฆาต + ณ]\, วิฆาต (ปุ) [วิ + หน + ณ]} {เป็นสิ่งนำมาซึ่งความลำบาก} {วิฆาโต}
\entry{วิจาริต} {ติ} {[วิ + จร + เณ + ต]} {วิจาร} {วิจาริตํ}
\entry{วิจฺฉิกวิชฺชา} {อิตฺ} {[วิจฺฉิก + วิชฺชา]} {ศาสตร์ว่าด้วยการรักษาพิษแมงป่อง} {วิจฺฉิกวิชฺชา}
\entry{วิชฺชติ} {ขฺยา} {[วิท + ย + เต]\, วิชฺชติ (กฺริ) [วิท + ย +ติ]} {ย่อมมี} {วิชฺชติ}
\entry{วิญฺญาณ} {นปุ} {[วิญฺญาณ + ราค]\, วิญฺญาณ (น) [วิญา + ยุ]} {วิญญาณ} {วิญฺญาณํ}
\entry{วิญฺญาณญฺจายตน} {นปุ} {[วิญฺญาณญฺจ + อายตน]} {วิญญาณัญจายตนฌาน} {วิญฺญาณญฺจายตนํ}
\entry{วิญฺญาณญฺจายตนูปค} {ติ} {[วิญฺญาณญฺจยตน + อุป + คมุ + กฺวิ.]} {เข้าถึงวิญญาณัญจายตนภูมิ} {วิญฺญาณญฺจายตนูปโค}
\entry{วิตกฺกวิจาร} {ปุ} {[วิตกฺก + วิจาร]} {วิตกและวิจาร} {วิตกฺกวิจารานํ}
\entry{วิตกฺกิต} {ติ} {[วิ + ตกฺก + ต]} {ยังมีวิตก} {วิตกฺกิตํ}
\entry{วิทิต} {ติ} {[วิท + กฺต]} {รู้แล้ว} {วิทิตา}
\entry{วิทิตฺวา} {พฺยย} {[วิท + ตฺวา]} {รู้แล้ว} {วิทิตฺวา}
\entry{วินยวาที} {ติ} {[วินย + วท + ณี]} {ผู้มีปกติกล่าวเฉพาะคำพูดที่เกี่ยวกับการสำรวมและการละกิเลส} {วินยวาที}
\entry{วินสฺสติ} {ขฺยา} {[วิ + นส + ย + ติ]} {ย่อมพินาศไป} {วินสฺสติ}
\entry{วินาส} {ปุ} {[วิ + นส + ณ]} {การพินาศไป} {วินาสํ}
\entry{วิปราวตฺต} {ติ} {[วิ + ปรา + วตุ + เณ + อ (ต)]\, วิปราวตฺต (ติ) [วิ + ปรา + วตุ + อ (ต)]} {สวนทางหรือขัดแย้ง} {วิปราวตฺตํ}
\entry{วิปริณามญฺญถาภาว} {ปุ} {[วิปริณาม + อญฺญถาภาว]} {ความเปลี่ยนแปลงเป็นอย่างอื่น} {วิปริณามญฺญถาภาวา}
\entry{วิปริณามธมฺม} {ติ} {[วิปริณาม + ธมฺม]} {ธรรมชาติที่มีการเปลี่ยนแปลงตลอดเวลา} {วิปริณามธมฺมา, วิปริณามธมฺโม}
\entry{วิปาก} {ติ} {[วิ + ปจ + ณ]\, วิปาก (ติ) [วิปาก + ณ]\, วิปาก (ปุ) [วิ + ปจ + เณ + ณ]} {วิบาก} {วิปาโก}
\entry{วิปฺปกต} {ติ} {[วิ + ป + กร + เณ + ต]\, วิปฺปกต (ติ) [วิ + กร + ต]} {ยังทำไม่เสร็จ[ยังพูดไม่จบ]} {วิปฺปกตา}
\entry{วิปฺผนฺทิต} {ติ} {[วิ + ผทิ + ต]} {เป็นธรรมชาติที่เป็นไปด้วยอำนาจของตัณหาและทิฏฐิ จึงยังหวั่นไหวไม่มั่นคงนั่นเทียว} {วิปฺผนฺทิตํ}
\entry{วิภว} {ปุ} {[วิภว + ตณฺหา]\, วิภว (ปุ) [วิภว + ทสฺสน]\, วิภว (ปุ) [วิ + ภู + อ]} {การไม่มีการสืบต่อแห่งภพชาติ} {วิภวํ}
\entry{วิรต} {ติ} {[วิมุ + ต]} {ผู้งดเว้น} {วิรโต}
\entry{วิราค} {ปุ} {[วิ + รนฺช + ณ]\, วิราค (ปุ) [วิราค + ณ]\, วิราค (ปุ) [วิราค + อนุปสฺสนา]} {ความเบื่อหน่าย} {วิราคา}
\entry{วิรุทฺธคพฺภกรณ} {นปุ} {[วิรุทฺธ + คพฺภ + กรณ]} {การให้ยาผดุงครรภ์} {วิรุทฺธคพฺภกรณํ}
\entry{วิเรจน} {นปุ} {[วิเรจน + ณ]\, วิเรจน (น) [วิ+ ริจ + ยุ]} {การทำให้ถ่าย[การปรุงยาถ่าย]} {วิเรจนํ}
\entry{วิเลปนสนฺนิธิ} {ปุ} {[วิเลปน + สนฺนิธิ]} {ไม่มีคำแปล} {วิเลปนสนฺนิธิํ}
\entry{วิวฏฺฏกถาทิ} {ติ} {[วิวฏฺฏ + กถา + อาทิ]} {วิวัฏฏกถาทิ [ตอนว่าด้วยวิวัฏฏะ เป็นต้น อันเป็นจุดหมายของผู้เจริญวิปัสสนา]} {วิวฏฺฏกถาทิ}
\entry{วิวฏฺฏติ} {ขฺยา} {[วิ + วตุ + อ + ติ]} {ย่อมก่อตัวตั้งขึ้น} {วิวฏฺฏติ}
\entry{วิวฏฺฏมาน} {ติ} {[วิ + วตุ + อ + มาน]} {ก่อตัวตั้งขึ้นอยู่} {วิวฏฺฏมาเน}
\entry{วิวทน} {นปุ} {[วิ + วท + ยุ]} {ไม่มีคำแปล} {วิวทนํ}
\entry{วิวาหน} {นปุ} {[วิ + วห + ยุ]} {การดูฤกษ์วิวาหมงคล[พิธีส่งตัวเจ้าบ่าวไปบ้านเจ้าสาว]} {วิวาหนํ}
\entry{วิวิจฺจ} {พฺยย} {[วิ + วิจ + ตฺวา]\, วิวิจฺจ (ติ)  [วิ + วิจ + ย + อ]\, วิวิจฺจ (กฺริ) [วิ + วิจ + ย + หิ]} {สงัดแล้ว} {วิวิจฺจ}
\entry{วิเวกช} {ติ} {[วิเวก + ชน + กฺวิ]} {อันเกิดแต่วิเวก[ความสงัดจากนิวรณ์ (หรืออันบังเกิดขึ้นในสภาวธรรมที่สงัดจากนิวรณ์)]} {วิเวกชํ}
\entry{วิสวิชฺชา} {อิตฺ} {[วิส + วิชฺชา]} {ศาสตร์ว่าด้วยการแก้พิษหรือการควบคุมพิษมิให้พิษกระจายหรือศาสตร์ว่าด้วยการถอนพิษงู} {วิสวิชฺชา}
\entry{วิสิขากถา} {อิตฺ} {[วิสิขา + กถา]} {คำพูดที่ไร้สาระเกี่ยวกับถนน\, ตรอก\, ซอย} {วิสิขากถํ}
\entry{วิสูกทสฺสน} {ติ} {[วิสูก + ทสฺสน]} {การดูการละเล่นอันเป็นข้าศึกต่อคำสอน} {วิสูกทสฺสนํ, วิสูกทสฺสนา}
\entry{วิหรติ} {ขฺยา} {[วิ + หร + อ + ติ]} {ย่อมประทับอยู่} {วิหรติ}
\entry{วิหรตุ} {ขฺยา} {[วิ + หร + อ + ตุ]} {จงอยู่} {วิหรตํ}
\entry{วิหรนฺติ} {ขฺยา} {[วิ + หร + อ + อนฺติ]} {ย่อมอยู่} {วิหรนฺติ}
\entry{วิหรามิ} {ขฺยา} {[วิ + หร + อ + มิ]} {ย่อมอยู่} {วิหรามิ}
\entry{วิหริมฺหา} {ขฺยา} {[วิ + หร + มฺหา]} {อยู่แล้ว} {วิหริมฺหา}
\entry{วีมํสานุจริต} {ติ} {[วีมํสา + อนุจริต]} {การตามพิจารณา} {วีมํสานุจริตํ}
\entry{วีมํสี} {ติ} {[วีมํสา + อี\, มาน + ส + ณี]} {ผู้มีการศึกษาค้นคว้า[นักปรัชญา]} {วีมํสี}
\entry{วีสํ} {อิตฺ} {[ทส + ทส + ทส + โย วี + อีสํ]} {๒๐} {วีสํ}
\entry{วุจฺจติ} {ขฺยา} {[วจ + ย + เต]} {อันถูกเรียก} {วุจฺจติ}
\entry{วุตฺต} {ติ} {[วป + ต\, นีติ.ธาตุ. ๔๐-๑\,๑๒๑.]} {ตรัสแล้ว} {วุตฺเต}
\entry{วูปสม} {ปุ} {[วิ + อุป + สมุ + เณ  + อ]} {การดับ} {วูปสมา}
\entry{เวตาฬ} {ปุ} {[ว + อิ + ต. โว วายุ อิโต คโต ยสฺมาติ เวโต. เวต + อล + อ. เวตํ อลติ ภูเสตีติ เวตาโล. อถวา เว วายุมฺหิ ตาโล ปติฏฺฐา ยสฺสาติ เวตาโล.]} {การตีฆ้องหรือการปลุกผีให้ฟื้นคืนชีพด้วยเวทมนต์คาถา} {เวตาฬํ}
\entry{เวทนา} {อิตฺ} {[วิท + ยุ. นีติ.ธาตุ. ๓๑๒. เวทยตีติ เวทนา. อารมฺมณรสํ เวทยนฺติ อนุภวนฺตีติ เวทนา.]} {เวทนา} {เวทนานํ}
\entry{เวทนาปจฺจย} {ปุ} {[เวทนา + ปจฺจย]} {มีเวทนาเป็นอุปนิสสยปัจจัย} {เวทนาปจฺจยา}
\entry{เวทยิต} {ติ} {[วิท + ณย + ต]} {ความสุขโสมนัสยินดี} {เวทยิตํ}
\entry{เวยฺยากรณ} {ติ} {[พฺยากรณ + ณ. รู. ๒๙. พฺยากรณมธีเต ชานาติ วา เวยฺยากรโณ. นีติ.สุตฺต. ๘๕๐.]} {ไวยากรณ์} {เวยฺยากรณสฺมิํ}
\entry{โวทาน} {ปุ\, นปุ} {[วิ + อว + ทา + ยุ]} {ความจรัสจ้า} {โวทานํ}
\entry{โวสฺสกมฺม} {นปุ} {[โวสฺส + กมฺม]} {การทำชายให้เป็นบัณเฑาะก์} {โวสฺสกมฺมํ}
\end{multicols}
\newpage
\needspace{5cm} \section*{\Huge ส}
\noindent\rule{\textwidth}{0.4pt}
\begin{multicols}{2}
\entry{สอุตฺตรจฺฉท} {ติ} {[สห + อุตฺตรจฺฉท]} {เครื่องลาดอันมีผ้าเพดานสีแดงขึงไว้ข้างบน} {สอุตฺตรจฺฉทํ}
\entry{สอุทฺเทส} {ติ} {[สห + อุทฺเทส]} {พร้อมด้วยชื่อเสียงเรียงนาม} {สอุทฺเทสํ}
\entry{สํกิรณ} {นปุ} {[สํ + กิร + เณ + ยุ]} {การดูฤกษ์เก็บรวบรวมทรัพย์} {สํกิรณํ}
\entry{สํกิเลส} {ปุ\, นปุ} {[สํ + กิลิส + ณ.  สํ + กิลิส + เณ + ณ.]} {ความขมุกขมัว} {สํกิเลสํ}
\entry{สํขิยธมฺม} {ปุ} {[สํขิย + ธมฺม]} {คำสนทนา} {สํขิยธมฺโม}
\entry{สํฆ} {ปุ\, อิตฺ} {[สํ + หน + ร]} {พระสงฆ์} {สํฆสฺส}
\entry{สํวฏฺฏติ} {ขฺยา} {[สํ + วฏฺฏ + อ + ติ]} {ย่อมพินาศ} {สํวฏฺฏติ}
\entry{สํวฏฺฏมาน} {ติ} {[สํวฏฺฏ + อ + มาน]} {พินาศอยู่} {สํวฏฺฏมาเน}
\entry{สํวฏฺฏวิวฏฏ} {ปุ} {[สํวฏฺฏ + วิวฏฺฏ]} {ซึ่งสังวัฏฏกัปและวิวัฏฏกัป} {สํวฏฺฏวิวฏฏํ}
\entry{สํวฏฺฏวิวฏฺฏ} {ปุ} {[สํวฏฺฏ + วิวฏฺฏ]} {ซึ่งสังวัฏฏกัปและวิวัฏฏกัป} {สํวฏฺฏวิวฏฺฏานิ}
\entry{สํวทน} {นปุ} {[สํ + วท + ยุ]} {ไม่มีคำแปล} {สํวทนํ}
\entry{สํวิชฺชติ} {ขฺยา} {[สํ + วิช + ย + ติ]} {ย่อมมีปรากฏ} {สํวิชฺชติ}
\entry{สํสรนฺติ} {ขฺยา} {[สํ + สร + อ + อนฺติ]} {ย่อมวนเวียน} {สํสรนฺติ}
\entry{สกุณวิชฺชา} {อิตฺ} {[สกุณ + วิชฺชา]} {ศาสตร์ว่าด้วยการทำนายเสียงนกร้อง} {สกุณวิชฺชา}
\entry{สงฺขาน} {นปุ} {[สํ + ขา + ยุ]} {วิชาคำนวณนับด้วยการใช้อุปกรณ์มีลูกคิด เป็นต้น} {สํขานํ}
\entry{สงฺขิยธมฺม} {ปุ} {[สงฺขิย + ธมฺม]} {สังขิยธรรม} {สงฺขิยธมฺมํ}
\entry{สงฺคามวิชย} {ปุ} {[สงฺคาม + วิชย]} {สังคามวิชัย} {สงฺคามวิชโย}
\entry{สเจ} {พฺยย} {[ส + เจ]} {หากว่า} {สเจ}
\entry{สจฺจวาที} {ติ} {[สจฺจ + วท + ณี]} {ผู้มีปกติกล่าวคำจริง} {สจฺจวาที}
\entry{สจฺจสนฺธ} {ติ} {[สจฺจ + สนฺธ]} {ผู้สืบต่อคำจริงด้วยคำจริง[เป็นผู้กล่าวคำจริงอย่างต่อเนื่อง} {สจฺจสนฺโธ}
\entry{สจฺฉิกตฺวา} {พฺยย} {[สจฺฉิ + กตฺวา]} {ประจักษ์แล้ว} {สจฺฉิกตฺวา}
\entry{สชิต} {ติ} {[สชฺช + ตุ]} {เป็นผู้จัดการ} {สชฺชิตา}
\entry{สญฺญมมตฺต} {นปุ} {[สญฺญม + มตฺต]} {ไม่มีคำแปล} {สญฺญิมตฺตานํ}
\entry{สญฺญี} {ติ} {[สญฺญา + อี]} {เป็นธรรมชาติที่มีสัญญา} {สญฺญี}
\entry{สญฺญีวาท} {ติ} {[สญฺญีวาท + ณ]} {ผู้มีความเห็นผิดคิดว่า ชีวิตหลังความตายมีสัญญา} {สญฺญีวาท, สญฺญีวาทา}
\entry{สญฺญุปฺปาท} {ปุ} {[สญฺญา + อุปฺปาท]} {การเกิดขึ้นแห่งปฏิสนธิสัญญา} {สญฺญุปฺปาทํ, สญฺญุปฺปาทา}
\entry{สติ} {อิตฺ} {[สร + ติ]} {สติ} {สติ, สติยา}
\entry{สติม} {ติ} {[สติ + ม]} {มีสติสมบูรณ์} {สติมา}
\entry{สโต} {พฺยย} {[ส + โต]} {มีสติสมบูรณ์} {สโต}
\entry{สตฺต} {ติ} {[สป + ตนินฺ.]} {สัตว์} {สตฺตสฺส, สตฺตหิ, สตฺตา, สตฺตานํ, สตฺโต}
\entry{สตฺถลกฺขณ} {นปุ} {[สตฺถ + ลกฺขณ]} {ศาสตร์ว่าด้วยการทำนายลักษณะของหอก[ทวน\,หลาว]} {สตฺถลกฺขณํ}
\entry{สทฺธาเทยฺย} {ติ} {[สทฺธา + ตพฺพ]} {บุคคลน้อมถวายด้วยจิตศรัทธา} {สทฺธาเทยฺยานิ}
\entry{สทฺธิํ} {พฺยย} {[สห + ริธ + อมุ]} {พร้อม} {สทฺธิํ}
\entry{สนฺต} {ติ} {อส + อนฺต. นีติ.สุตฺต. ๑๐๑๙. นีติ.ปท. ๒๓๕-๖.} {เป็นธรรมชาติที่สงบจากความเร่าร้อนทั้งปวง} {สนฺตํ, สนฺตา}
\entry{สนฺตต} {ติ} {[สํ + ตนุ + ต. สมนฺตโต  ปุนปฺปุนํ วา ตโนตีติ สนฺตตํ. ธานฺ.ฏี. ๔๓. ]} {การเกิดปรากฏ} {สนฺตตาย}
\entry{สนฺติ} {อิตฺ} {[สมุ + ติ. นีติ.สุตฺต. ๑๑๘๘.]} {มีอยู่} {สนฺติ}
\entry{สนฺติก} {ติ} {[สห + อนฺต= สนฺต + อิก. สห อนฺเตน สนฺติกํ\, สกตฺเถ อิโก. ธานฺ.ฏี. ๔๖๒.]} {การพนันหมากไหว} {สนฺติกํ}
\entry{สนฺติกมฺม} {นปุ} {[สนฺติ + กมฺม]} {การทำพิธีบนบานหรือการทำพิธีบูชาเทพยดา} {สนฺติกมฺมํ}
\entry{สนฺธาต} {ติ} {[สํ + ธา + ต]} {เป็นผู้ประสาน} {สนฺธาตา}
\entry{สนฺธาวนฺติ} {ขฺยา} {[สํ + ธาวุ + อ + อนฺติ]} {ย่อมโลดแล่นไป} {สนฺธาวนฺติ}
\entry{สนฺนิธิการปริโภค} {ปุ} {[สนฺนิธิการ + ปริโภค]} {การบริโภคใช้สอยปัจจัย ๔ ที่เก็บสะสมไว้} {สนฺนิธิการกปริโภคํ, สนฺนิธิการกปริโภคา}
\entry{สนฺนิปติต} {ติ} {[สํ + นิ + ปต + ต]} {ผู้ชุมนุมกันแล้ว} {สนฺนิปติตา, สนฺนิปติตานํ}
\entry{สนฺนิสินฺน} {ติ} {[สํ + นิ + สท + ต]} {ผู้นั่งแล้ว} {สนฺนิสินฺนา, สนฺนิสินฺนานํ}
\entry{สปฺปิโหม} {ปุ} {[สปฺปิ + โหม]} {ศาสตร์ว่าด้วยการทำนายตามลักษณะของเนยใสที่ใช้ในการบูชาไฟ} {สปฺปิโหมํ}
\entry{สพฺพ} {ติ} {ไม่มีการประกอบคำ} {ทั้งปวง} {สพฺพานิ, สพฺเพ, สพฺเพหิ}
\entry{สพฺพงฺคปจฺจงฺคี} {ติ} {[สพฺพงฺคปจฺจงฺค + อี]} {อันบริบูรณ์ด้วยอวัยวะองค์น้อยองค์ใหญ่} {สพฺพงฺคปจฺจงฺคี}
\entry{สพฺพปาณภูตหิตานุกมฺปี} {ติ} {[สพฺพปาณภูต + หิต + อนุกมฺปี]} {เป็นผู้มีจิตอนุเคราะห์ด้วยการบำเพ็ญประโยชน์แก่เหล่าสัตว์ทั้งปวง} {สพฺพปาณภูตหิตานุกมฺปี}
\entry{สพฺพโส} {พฺยย} {[สพฺพ + โส]} {โดยประการทั้งปวง} {สพฺพโส}
\entry{สมคฺคกรณี} {ติ} {[สมคฺค + กรณ + อี]} {ที่ก่อให้บุคคลเกิดความสามัคคี} {สมคฺคกรณิํ}
\entry{สมคฺคนนฺที} {ติ} {[สมคฺค+ นนฺที]} {ผู้ชื่นชมในความสามัคคี} {สมคฺคนนฺที}
\entry{สมคฺครต} {ติ} {[สมคฺค + รต]} {ผู้ยินดีในความสามัคคี} {สมคฺครโต}
\entry{สมคฺคาราม} {ติ} {[สมคฺค + อาราม]} {ผู้มีความยินดีในความสามัคคี} {สมคฺคาราโม}
\entry{สมงฺคีภูต} {ติ} {[สมงฺคี + ภูต]} {มีความเพียบพร้อม} {สมงฺคีภูโต}
\entry{สมณ} {ปุ} {[สมุ + (เณ) + ยุ]} {สมณะ} {สมณา, สมโณ}
\entry{สมณพฺราหฺมณ} {ปุ} {[สมณ + พฺราหฺมณ]} {สมณะและพราหมณ์} {สมณพฺราหฺมณ, สมณพฺราหฺมณา, สมณพฺราหฺมณานํ}
\entry{สมณาพฺราหฺมณ} {ปุ} {[สมณ + พฺราหฺมณ]} {สมณะและพราหมณ์} {สมณาพฺราหฺมณา}
\entry{สมติกฺกม} {ปุ} {[สํ + อติ + กมุ + อ ]} {หลังจากที่ได้ข้ามพ้นแล้ว} {สมติกฺกมา}
\entry{สมติกฺกมฺม} {พฺยย} {[สํ + อติ + กมุ + ตฺวา]} {หลังจากที่ได้ข้ามพ้นแล้ว} {สมติกฺกมฺม}
\entry{สมนุคาเหยฺยุํ} {ขฺยา} {[สํ + อนุ + คห + เณ + เอยฺยุํ]} {พึงสอบสวน} {สมนุคาเหยฺยุํ}
\entry{สมนุภาเสยฺยุํ} {ขฺยา} {[สํ + อนุ + ภาส + เอยฺยุํ]} {พึงว่ากล่าวตักเตือน} {สมนุภาเสยฺยุํ}
\entry{สมนุยุญฺเชยฺยุํ} {ขฺยา} {[สํ + อนุ + ยุช + เอยฺยุํ]} {พึงไต่ถาม} {สมนุยุญฺเชยฺยุํ}
\entry{สมปฺปิต} {ติ} {[สํ + อปฺป + ต. สํ + อปิ + ต]} {สมบูรณ์} {สมปฺปิโต}
\entry{สมย} {ปุ\, นปุ} {[สํ + อิ + อ.  สม + ยา + อ. สํ + อย + อ.]} {กาลเวลา} {สมยํ, สมโย}
\entry{สมาธิช} {ติ} {[สมาธิ + ชน + กฺวิ]} {ที่บังเกิดแต่สมาธิ} {สมาธิชํ}
\entry{สมาน} {ติ} {[อส + มาน]} {มีอยู่} {สมานา, สมาโน}
\entry{สมุจฺฉินฺน} {ติ} {[สํ + อุ + ฉิทิ + ต]} {เป็นธรรมชาติที่ขาดสูญ} {สมุจฺฉินฺโน}
\entry{สมุทย} {ปุ} {[สํ + อุ + อิ + ณ]} {การเกิด(หรือเหตุแห่งการเกิด)} {สมุทยํ}
\entry{สมุทฺทกฺขายิก} {อิตฺ} {[สมุทฺท + อา + ขา + ณฺวุ]} {คำพูดที่ไร้สาระเกี่ยวกับทะเล} {สมุทฺทกฺขายิกํ}
\entry{สมฺปชาน} {ติ} {[สํ + ป + ญา + นา + อ]} {ถึงพร้อมด้วยปัญญา} {สมฺปชาโน}
\entry{สมฺปสาทน} {ติ} {[สํ + ป + สีท + เณ + ยุ. นีติ.ธาตุ. ๙๕. ป + สท + เณ + ยุ. นิรุตฺติ. ๔๓๑.]} {อันยังจิตให้ผ่องใส} {สมฺปสาทนํ}
\entry{สมฺปาเยยฺยํ} {ขฺยา} {[สํ + ปท + ณย + เอยฺยํ.  สํ + ป + ยา + เอยฺยํ]} {ยังคำตอบให้ถึงพร้อม} {สมฺปาเยยฺยํ}
\entry{สมฺผปฺปลาป} {ปุ\, นปุ} {[สมฺผ + ป + ลป + ณ]} {คำพูดที่เพ้อเจ้อ} {สมฺผปฺปลาปํ, สมฺผปฺปลาปา}
\entry{สมฺพหุล} {ติ} {[สํ + พล + กุลจฺ]} {จำนวนมาก} {สมฺพหุลานํ}
\entry{สมฺพาหน} {นปุ} {[สํ + พาห + ยุ.  สํ + วาห + ยุ สํปุพฺโพ วาห ปยตเน มทฺทเน วา. ยุ. ธานฺ.ฏี. ๗๖๙.]} {การเพาะกาย} {สมฺพาหนํ}
\entry{สมฺภวนฺติ} {ขฺยา} {[สํ + ภู + อ + อนฺติ]} {ย่อมบังเกิด} {สมฺภวนฺติ}
\entry{สมฺมา} {อิตฺ} {[สมุ + ม. ณฺวาทิ. ๑๓๖. สมนฺติ ยาย สมฺมา. ธานฺ.ฏี. ๔๔๙.  สมฺม เต อิมายาติ สมฺมา.]} {โดยถูกต้อง} {สมฺมา}
\entry{สมฺมาปฏิปนฺน} {ติ} {[สมฺมา + ปฏิ + ปท + ต]} {ผู้ปฏิบัติถูก} {สมฺมาปฏิปนฺโน}
\entry{สมฺมามนสิการ} {ปุ} {[สมฺมา + มนสิการ]} {การพิจารณา ในใจด้วยเหตุผลที่ถูกต้อง} {สมฺมามนสิการํ}
\entry{สมฺมาสมฺพุทฺธ} {ปุ} {[สมฺมา + สํ + พุธ + ต. นิรุตฺติ. ๗๕๐. ธานฺ.ฏี. ๔.  สมฺมา + สมฺมาสมฺพุทฺธ. มณิมญฺชู. ๗๔\,๘๐.๘๒.]} {ผู้ตรัสรู้สรรพธรรมด้วยพระองค์เองโดยถูกต้องตามสภาพที่เป็นจริง} {สมฺมาสมฺพุทฺเธน}
\entry{สมฺมุสฺสติ} {ขฺยา} {สํ + มุส + ย + ติ} {ย่อมหลงลืม} {สมฺมุสฺสติ}
\entry{สมฺโมส} {ปุ} {[สํ + มุส + ณ]} {การหลงลืม} {สมฺโมสา}
\entry{สยํ} {พฺยย} {[สุ + อย + อมุ]} {ด้วยตนเอง} {สยํ}
\entry{สยํปฏิภาณ} {ติ} {[สยํ + ปฏิภาน]} {ตามปฏิภาณของตน} {สยํปฏิภาณํ}
\entry{สยํปฏิภาน} {ติ} {[สยํ + ปฏิภาน]} {ตามปฏิภาณของตน} {สยํปฏิภานํ}
\entry{สยํปภา} {อิตฺ} {[สยํ + ปภา]} {ผู้มีแสงสว่างออกจากร่างกายของตนเอง} {สยํปภา}
\entry{สยนกถา} {อิตฺ} {[สยน + กถา]} {คำพูดที่ไร้สาระเกี่ยวกับที่นอนหมอนเสื่อ} {สยนกถํ}
\entry{สยนสนฺนิธิ} {ปุ} {[สยน + สนฺนิธิ]} {การเก็บสะสมที่นอน} {สยนสนฺนิธิํ}
\entry{สยมฺปภ} {ติ} {[สยํ + ปภ + อ]} {เป็นผู้มีแสงสว่างออกจากร่างกายของตนเอง} {สยํปโภ}
\entry{สรปริตฺตาณ} {นปุ} {[สร + ปริตฺตาณ]} {ศาสตร์ว่าด้วยการป้องกันลูกศร[ศาสตร์คงกระพัน]} {สรปริตฺตาณํ}
\entry{สลากหตฺถ} {ปุ} {[สลาก + หตฺถ]} {การพนันกำทาย[ทายด้วยไม้เซียมซี]} {สลากหตฺถํ}
\entry{สลฺลกตฺติย} {นปุ} {[สลฺลกตฺต + ณฺย]} {การผ่าตัด[การทำให้ศรหลุดจากร่างกาย]} {สลฺลกตฺติยํ}
\entry{สวิจาร} {ติ} {[สห + วิจาร]} {อันเป็นไปกับด้วยวิจาร[อันถึงพร้อมด้วยวิจาร]} {สวิจารํ}
\entry{สวิตกฺก} {ติ} {[สห + วิตกฺก]} {อันเป็นไปกับด้วยวิตก[อันถึงพร้อมด้วยวิตก]} {สวิตกฺกํ}
\entry{สโสมนสฺส} {ติ} {[สห + โสมนสฺส]} {ความโสมนัส} {โสมนสฺสํ}
\entry{สสฺสต} {ติ} {[สสฺสติ + อ]} {เป็นสิ่งเที่ยงแท้} {สสฺสตํ, สสฺสตา, สสฺสโต}
\entry{สสฺสตวาท} {ติ} {[สสฺสต + วาท]} {ผู้มีความเห็นผิดคิดว่า อัตตาและโลกเป็นสิ่งเที่ยงแท้} {สสฺสตวาท, สสฺสตวาทา}
\entry{สสฺสติสม} {ติ} {[สสฺสติ + สม]} {เปรียบเหมือนกับวัตถุอมตะ เช่น แผ่นดิน ขุนเขาสิเนรุ ดวงอาทิตย์ และดวงจันทร์} {สสฺสติสมํ}
\entry{สหพฺยตา} {อิตฺ} {[สหพฺย + ตา]} {ความเป็นสหายร่วมภพ} {สหพฺยตํ}
\entry{สหิต} {ติ} {[สห + อิ + ต]} {ผู้สามัคคีกัน} {สหิตํ, สหิตานํ}
\entry{สาการ} {ติ} {[สห + อาการ]} {พร้อมด้วยลักษณะอาการ} {สาการํ}
\entry{สาปเทส} {ติ} {[สห + อปเทส]} {มีการอ้างอิงหลักฐาน มีอุปมาอุปไมย ไม่เลื่อนลอย} {สาปเทสํ}
\entry{สาลากิย} {ปุ\, นปุ} {[สาลากา + ณฺย]} {การรักษาโรคตาต้อ} {สาลากิยํ}
\entry{สิขาพนฺธ} {ปุ} {[สิขา + พนฺธ]} {การสวมเกี้ยว} {สิขาพนฺธํ}
\entry{สิต} {ติ} {[สิ + ต]} {ติดอยู่} {สิตา}
\entry{สิริวฺหายน} {นปุ} {[สิริ + อวฺหายน]} {การทำพิธีเรียกขวัญ} {สิริวฺหายนํ}
\entry{สิววิชฺชา} {อิตฺ} {[สิว + วิชฺชา]} {ศาสตร์ที่ต้องศึกษาอยู่ในป่าช้า เช่น ทำเสน่ห์ หรือศาสตร์ที่ล่วงรู้เสียงของสัตว์มีสุนัขจิ้งจอก เป็นต้น)} {สิววิชฺชา}
\entry{สีลมตฺตก} {นปุ} {[สีล + มตฺต + ก]} {เพียงข้อปฏิบัติเล็กน้อย} {สีลมตฺตกํ}
\entry{สีสวิเรจน} {นปุ} {[สีส + วิเรจน]} {การทำให้เชื้อโรคหลุดออกจากศีรษะ} {สีสวิเรจนํ}
\entry{สุกตทุกฺกฏ} {นปุ} {[สุกต + ทุกฺกฏ]} {อันบุคคลทำไว้ดีและไม่ดี} {สุกตทุกฺกฏานํ}
\entry{สุข} {นปุ} {[สุข + อ. สุ + ขาท + กฺวิ. สุ + ขนุ + กฺวิ. สุ + ขมุ + กฺวิ. สุ + ข.]} {สุข} {สุขํ, สุขสฺส}
\entry{สุขทุกฺขี} {ติ} {[สุขทุกฺข + อี]} {ผู้มีทั้งสุขและทุกข์} {สุขทุกฺขี}
\entry{สุขวิหารี} {ติ} {[สุขวิหาร + ณี]} {อยู่อย่างมีความสุข} {สุขวิหารี}
\entry{สุขุมจฺฉิก} {ติ} {[สุขุม + อจฺฉิก]} {อันมีตาถี่} {สุขุมจฺฉิเกน}
\entry{สุจิภูต} {ติ} {[สุจิ + ภูต]} {สะอาดบริสุทธิ์} {สุจิภูเตน}
\entry{สุญฺญ} {ติ} {[สุน + ย. สุนสฺส หิตํ สุญฺญํ. โย. นฺยสฺส ญฺโญ. สุน คติยํ วา. โย. ธานฺ.ฏี. ๖๙๘. สุน + ย.]} {ไม่มีคำแปล} {สุญฺญํ}
\entry{สุต} {ติ} {[สุ + ต. สวนํ สุตํ. อสุยิตฺถาติ วา สุตํ. นีติ.ธาตุ. ๒๔๑-๒.]} {ได้สดับจดจำมา} {สุตํ}
\entry{สุตฺวา} {พฺยย} {[สุ + ตฺวา]} {ฟังแล้ว} {สุตฺวา}
\entry{สุทํ} {พฺยย} {[อีกนัยหนึ่ง สุ + อิทํ]} {(๑) เป็นเพียงปทปูรณนิบาต หมายถึง เป็นนิบาตที่ไม่มีความหมาย ใส่เข้ามาเพื่อประดับคำให้เกิดเสียงสละสลวย(วาจาสิลิฏฺฐ) เท่านั้น [สุทนฺติ ปทปูรณมตฺเต นิปาโต.(ม.อฏฺ. ๑/๓๕๒) สุทนฺติ นิปาตมตฺตํ. บทว่า สุทํ เป็นเพียงนิบาต (ที.อฏฺ. ๑/๓๗]
(๒) เช่นนี้\, อย่างนี้\, อย่างนี้นั่นเทียว
สุทนฺติ สุทํ. สนฺธิวเสน อิการโลโป เวทิตพฺโพ. จกฺขุนฺทฺริยํ... กึสูธวิตฺตนฺติ อาทีสุ วิย. (บทว่า สุทํ ตัดบทสนธิ เป็น สุ+อิทํ  เมื่อจะเข้าสนธิ ให้ลบ อิ หลัง เหมือนในข้อความเป็นต้นว่า จกฺขุนฺทฺริยํ... กุสุธวิตฺตํ.  ซึ่ง \`กึสุธ\` นี้  ตัดสนธิว่า กึสุ+อิธ  ลบ อิ หลัง แล้วทีฆะ อุ เป็น อู ได้รูปว่า \`กึสูธ\` (วิ.อฏฺ. ๑/๑๕๗)} {สุทํ}
\entry{สุปิน} {ปุ\, นปุ} {[สุป + อิน. สุปนฺติ เอเตนาติ สุปินํ.]} {ศาสตร์ว่าด้วยการทำนายฝัน} {สุปินํ}
\entry{สุปฺปฏิวิทิต} {ติ} {[สุ + ปฏิ + วิท + ต]} {ทรงรู้แจ้งแทงตลอดแล้วซึ่งอัธยาศัยที่ต่างกันของบุคคลทั้งสองเหล่านั้นเช่นไร} {สุปฺปฏิวิทิตา}
\entry{สุปฺปิย} {ติ} {[สุ + ปี + อ]} {ผู้มีนามว่าสุปปิยะ} {สุปฺปิยสฺส, สุปฺปิโย}
\entry{สุภคกรณ} {นปุ} {[สุภค + กรณ]} {การดูฤกษ์ประกอบงานที่เป็นสิริมงคลหรือการทำให้คนรัก[ทำเสน่ห์]} {สุภคกรณํ}
\entry{สุภฏฺฐายี} {ติ} {[สภ + ฐายี- ฐา + ณย + ณี]} {ผู้อาศัยอยู่ในอุทยาน วิมาน อันงดงาม} {สุภฏฺฐายิโน, สุภฏฺฐายี}
\entry{สุภาสิต} {ติ} {[สุ + ภาส + ต]} {ความหมายของคำที่เป็นสุภาษิต[คำพูดที่ดี]} {สุภาสิตํ}
\entry{สุภิกฺข} {ติ} {[สุ + ภิกฺขา. สุลภา ภิกฺขา ยสฺมึ ชนปเท โสยํ สุภิกฺโข. นิรุตฺติ. ๒๔๘-๙.]} {อาหารอุดมสมบูรณ์} {สุภิกฺขํ}
\entry{สุมน} {ติ} {[สุ + มน. สุนฺทรํ มนํ จิตฺตํ ยสฺส โส สุมโน. อป.อฏฺ. ๒/๙๖. โสภณํ  มโน ยสฺมาติ.]} {ผู้ดีใจ} {สุมนา}
\entry{สุวุฏฺฐิกา} {อิตฺ} {[สุวุฏฺฐิ + ก + อา]} {ฝนดี} {สุวุฏฺฐิกา}
\entry{สูรกถา} {อิตฺ} {[สูร + กถา]} {คำพูดที่ไร้สาระเกี่ยวกับวีรบุรุษ[ผู้กล้า]} {สูรกถํ}
\entry{สูริยคฺคาห} {ปุ} {[สูริย + คห + ณ]} {สุริยคราส} {สูริยคฺคาโห}
\entry{เสฏฺฐ} {ติ} {[เสฏฺฐ + ณ]} {ผู้ทรงเกียรติอันยิ่งใหญ่} {เสฏฺโฐ}
\entry{เสนากถา} {อิตฺ} {[เสนา + กถา]} {คำพูดที่ไร้สาระเกี่ยวกับกองทัพ} {เสนากถํ}
\entry{เสนาพฺยูห} {ปุ} {[เสนา + พฺยูห =วิ + อูห + อ]} {การจัดกระบวนทัพ} {เสนาพฺยูหํ}
\entry{เสยฺยถา} {พฺยย} {[เสยฺย + ถา]} {เป็นเช่นไร} {เสยฺยถา}
\entry{เสยฺยถิทํ} {พฺยย} {[เสยฺย + อิทํ]} {ได้แก่} {เสยฺยถิทํ}
\entry{โสกปริเทวทุกฺขโทมนสฺสุปายาส} {ปุ} {[โสก + ปริเทว + ทุกฺข + โทมนสฺส + อุปายาส]} {ความเศร้าโศกเสียใจ ความร้องไห้คร่ำครวญ ความทุกข์กายทุกข์ใจ และความคับแค้นใจ} {โสกปริเทวทุกฺขโทมนสฺสุปายาสา}
\entry{โสต} {นปุ} {[สุต + ณฺวุ]} {โสตะ} {โสตํ}
\entry{โสภนครก} {นปุ} {[โสภนคร + ก]} {การแต่งตัวให้นักแสดง} {โสภนครกํ}
\entry{โสมนสฺสโทมนสฺส} {นปุ} {[โสมนสฺส + โทมนสฺส]} {โสมนัสและโทมนัสเวทนา} {โสมนสฺสโทมนสฺสานํ}
\entry{โสฬส} {ติ} {ไม่มีการประกอบคำ} {๑๖ ประการ} {โสฬสหิ}
\end{multicols}
\newpage
\needspace{5cm} \section*{\Huge ห}
\noindent\rule{\textwidth}{0.4pt}
\begin{multicols}{2}
\entry{หตฺถตฺถร} {นปุ} {[หตฺถ + อตฺถร]} {เครื่องลาดบนหลังช้าง} {หตฺถตฺถรํ}
\entry{หตฺถพนฺธ} {ปุ} {[หตฺถ + พนฺธ]} {การสวมเครื่องประดับข้อมือหรือการสวมกำไลข้อมือ} {หตฺถพนฺธํ}
\entry{หตฺถาภิชปฺปน} {นปุ} {[หตฺถ + อภิ+ ชปฺป + ยุ]} {การร่ายมนต์ทำให้มือมีอาการผิดปกติไม่อยู่ในรูปทรงเดิม} {หตฺถาภิชปฺปนํ}
\entry{หตฺถิควสฺสวฬวาปฏิคฺคหณ} {นปุ} {[หตฺถิ + คว + อสฺส + วฬว + ปฏิคฺคหณ]} {การรับช้าง โค ม้า และลา} {หตฺถิควสฺสวฬวาปฏิคฺคหณา}
\entry{หตฺถิยุทฺธ} {นปุ} {[หตฺถิ + ยุทฺธ]} {การแข่งขันชนช้าง} {หตฺถิยุทฺธํ}
\entry{หตฺถิลกฺขณ} {นปุ} {[หตฺถิ + ลกฺขณ]} {ศาสตร์ว่าด้วยการทำนายลักษณะของช้าง} {หตฺถิลกฺขณํ}
\entry{หทยงฺคม} {ติ} {[หทย + คมุ + อ]} {ตรึงใจ} {หทยงฺคมา}
\entry{หนุชปฺปน} {นปุ} {[หนุ + ชปฺปน]} {การร่ายมนต์ทำให้คางมีอาการผิดปกติ} {หนุชปฺปนํ}
\entry{หนุสํหนน} {นปุ} {[หนุ + สํหนน]} {การร่ายมนต์ทำให้คางแข็ง} {หนุสํหนนํ}
\entry{หร} {ขฺยา} {[หร + อ + หิ]} {จงนำไป} {หร}
\entry{หสขิฑฺฑารติธมฺมสมาปนฺน} {ติ} {[หส + ขิฑฺฑา + รติ + ธมฺม + สมาปนฺน]} {เป็นผู้หมกมุ่นในความเพลิดเพลินกล่าวคือความสนุกสนานและการละเล่น} {หสขิฑฺฑารติธมฺมสมาปนฺนา, หสขิฑฺฑารติธมฺมสมาปนฺนานํ}
\entry{หิ} {พฺยย} {ไม่มีการประกอบคำ} {ด้วยว่า} {หิ}
\entry{เหตุ} {ปุ} {[เหตุ + ณ]} {เหตุ} {เหตุ}
\entry{โหติ} {ขฺยา} {[หู + อ + ติ]} {ย่อมเป็น} {โหติ}
\entry{โหนฺติ} {ขฺยา} {[หู + อ + อนฺติ]} {ย่อมเป็น} {โหนฺติ}
\end{multicols}

    \newpage
    
\end{document}
